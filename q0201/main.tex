\input{thispreamble.tex}


\renewcommand\AUTHOR{nweadick1@cougars.ccis.edu} % CHANGE TO YOURS

\begin{document}
\topmattertwo

When writing results of computations, make sure it's simplified.
For instance when you write down sets,
if the answer is $\{1\}$, I do not
want to see $\{1, 1\}$.

When you write down an example or a counterexample,
be elegant and write down the simplest.
You only need to give a counterexample if asked.

$P(X)$ denotes the powerset of $X$.

In \LaTeX\, math notation is enclosed by the \verb!$! symbols.
For instance \verb!$x = a_{1} + b^{2}$! gives you $x = a_{1} + b^{2}$. 

(For more information about \LaTeX\, go to my
website
\href{http://bit.ly/yliow0}{http://bit.ly/yliow0},
click on \verb!Yes!
you are one of my students,
then look for
\href{https://drive.google.com/file/d/0BzjYrK0VFuMWZm5xV0kyR3J2Zm8/view?usp=sharing}{latex.pdf}.)

For the next few questions,
let $A = \{3, 2, \pi, 4\}$, $B = \{2,4,1,3,2,4\}$, $C = \{7, 5, 3\}$ 

%------------------------------------------------------------------------------
\nextq
$|A| =  \answerbox{ 4 } $ (Please correct!)

%------------------------------------------------------------------------------
\nextq
$|B| = \answerbox{ 4 } $

%------------------------------------------------------------------------------
\nextq
What is $A \cup B$?
\begin{answerlong}
$A \cup B = \{ 1,2,3,4,\pi\}$
\end{answerlong}

%------------------------------------------------------------------------------
\nextq
What is $A \cap B$?
\begin{answerlong}
$A \cap B = \{2,3,4\}$
\end{answerlong}

%------------------------------------------------------------------------------
\nextq
What is $A - C = \{ \answerbox{} \}$?
\begin{answerlong}
$A - C = \{ 2, \pi, 4 \}$
\end{answerlong}

%------------------------------------------------------------------------------
\nextq
What is $A \times C$?
\begin{answerlong}
$A \times C = \{ (3,7),(3,5),(3,3),(2,7),(2,5),(2,3),(pi,7),(pi,5),(pi,3),(4,7),(4,5),(4,3)\}$
\end{answerlong}

%------------------------------------------------------------------------------
\nextq
What is $P(C)$?
\begin{answerlong}
$P(C) = \{ \emptyset, \{3\}, \{5\}, \{7\}, \{3,5\}, \{3,7\}, \{5,7\}, \{3,5,7\} \}$
\end{answerlong}
%------------------------------------------------------------------------------
For the next few questions, let $X, Y, Z$ be sets (in the same universe).

\nextq
\tf: $X \cup Y = Y \cup X$
\dotfill\answerbox{T}

%------------------------------------------------------------------------------
\nextq
\tf: $X \cup (Y \cap Z) = (X \cup Y) \cap Z$.
\dotfill\answerbox{F}
\\
If F, provide a counterexample
\begin{answerlong}
$X = \{ 1 ,2\}$, $Y = \{ 2,3 \}$, $Z = \{ 3,4 \}$

$X \cup (Y \cap Z) = \{1,2,3\}$ 
$(X \cup Y) \cap Z = \{3\}$
\end{answerlong}

%------------------------------------------------------------------------------
\nextq
\tf: If $X \neq \emptyset$ and $Y \neq \emptyset$, then 
$X \cup Y \neq \emptyset$.
\dotfill\answerbox{T}
\\
If F, provide a counterexample
\begin{answerlong}

\end{answerlong}

%------------------------------------------------------------------------------
\nextq
\tf: If $X \neq \emptyset$ and $Y \neq \emptyset$, then 
$X \cap Y \neq \emptyset$.
\dotfill\answerbox{F}
\\
If F, provide a counterexample
\begin{answerlong}
$X = \{ 1 \}$, $Y = \{ 2 \}$.  $then X \cap Y = \emptyset$
\end{answerlong}

%------------------------------------------------------------------------------
\nextq
\tf: $\emptyset \subseteq X$ for any set $X$.
\dotfill\answerbox{T}

%------------------------------------------------------------------------------
\nextq
\tf: $\emptyset \in X$ for any set $X$.
\dotfill\answerbox{F}

%------------------------------------------------------------------------------
\nextq
\tf: $\emptyset \times X = \emptyset$
\dotfill\answerbox{T}

%------------------------------------------------------------------------------
\nextq
Find the simplest set $X$ and $Y$ such that $X \in Y$
and $X \subseteq Y$.
\begin{answerlong}
$X = \{ 1 \}$, $Y = \{ 1,2 \}$.
\end{answerlong}

%------------------------------------------------------------------------------
\nextq
\tf: $P(\emptyset) = \emptyset$
\dotfill\answerbox{T}

%------------------------------------------------------------------------------
\nextq
\tf: $X \subseteq Y \implies P(X) \subseteq P(Y)$.
\dotfill\answerbox{T}
\\
If F, provide a counterexample
\begin{answerlong}

\end{answerlong}

%------------------------------------------------------------------------------
\nextq
\tf: $P(X \cup Y) = P(X) \cup P(Y)$.
\dotfill\answerbox{F}
\\
If F, provide a counterexample
\begin{answerlong}
$X = \{1\}, Y = \{2\}, P(X \cup Y) = \{\emptyset, \{1\}, \{2\}, \{1,2\} \}$
$P(X) \cup P(Y) = \{\emptyset, \{1\}, \{2\}\}$
\end{answerlong}

%------------------------------------------------------------------------------
\nextq
\tf: $P(X \cap Y) = P(X) \cap P(Y)$.
\dotfill\answerbox{T}
\\
If F, provide a counterexample
\begin{answerlong}

\end{answerlong}

%------------------------------------------------------------------------------
\nextq
\tf: $P(X \times Y) = P(X) \times P(Y)$.
\dotfill\answerbox{T}
\\
If F, provide a counterexample
\begin{answerlong}

\end{answerlong}

%------------------------------------------------------------------------------
\nextq
Hans Solo and Luke Skywalker are leading a team of 10 to attack
an AT-AT and some stormtroopers. (The 10 includes Hans and Luke.)
They have decided to split into two teams.
The team attacking the AT-AT will have at least 2 members.
How many ways are there to form such a team?
(Explain your work with complete sentences.)
\begin{answerlong}
$2^10- C(10,0) + C(10,1) = 1024 - 11 = 1013$
\end{answerlong}

%------------------------------------------------------------------------------
\nextq
\tf. It is possible to construct sets $W$, $X$, $Y$, $Z$ such that
\[
|W \cap X| = |X \cap Y| = |Y \cap Z| = |Z \cap W| = 2
\]
and
\[
|W \cap X \cap Y| = |X \cap Y \cap Z| = |Y \cap Z \cap W| = |Z \cap W \cap X| = 0
\]
\dotfill\answerbox{T}
\\
If your answer is T, provide the simplest possible example.
\begin{answerlong}
    $W = \{1,2,3,4\}, X = \{3,4,5,6\}, Y = \{5,6,7,8\}, Z = \{7,8,1,2\}$
\end{answerlong}

%------------------------------------------------------------------------------
\nextq
\tf. It is possible to construct sets $W$, $X$, $Y$, $Z$ such that
\[
|W \cap X| = |X \cap Y| = |Y \cap Z| = |Z \cap W| = 2
\]
and
\[
|W \cap X \cap Y| = |X \cap Y \cap Z| = |Y \cap Z \cap W| = |Z \cap W \cap X| = 1
\]
\dotfill\answerbox{T}
\\
If your answer is T, provide the simplest possible example.
\begin{answerlong}
    $W = \{1,2,3\}, X = \{2,3,4\}, Y = \{3,4,5\}, Z = \{1,3,4\}$
\end{answerlong}

%------------------------------------------------------------------------------
\nextq
\tf: It is possible find a set $X$ and a function $f : X \rightarrow X$
such that $|X| = 4$, $|f(X)| = 3$, $|f(f(X)| = 2$, $|f(f(f(X)))| = 1$
\dotfill\answerbox{T}
\\
If your answer is T, provide the simplest possible example.
\begin{answerlong}
    $X = \{1,2,3,4\}$
\end{answerlong}

%------------------------------------------------------------------------------
\nextq
\tf:
If $f : X \rightarrow Y$ and $g: Y \rightarrow Z$ are onto (i.e., surjective)
functions, then $g \circ f: X \rightarrow Z$ is also onto.
\dotfill\answerbox{T}
\\
If F, provide a counterexample.
\begin{answerlong}

\end{answerlong}

%------------------------------------------------------------------------------
\nextq
\tf:
If $f : X \rightarrow Y$ and $g: Y \rightarrow Z$ are 1--1 
(i.e., injective)
functions, then $g \circ f: X \rightarrow Z$ is also 1--1.
\dotfill\answerbox{T}
\\
If F, provide a counterexample.
\begin{answerlong}

\end{answerlong}

%------------------------------------------------------------------------------
\nextq
\tf:
Let $f : X \rightarrow Y$ and $g: Y \rightarrow Z$ be functions.
If $f$ is onto and $g$ is not onto, then $g \circ f$ is onto.
\dotfill\answerbox{F}
\\
If F, provide a counterexample.
\begin{answerlong}
    $X = \{1\}, Y=\{1\}, Z=\{1,2\}$
    $g \circ f$(1) = 1 not onto
\end{answerlong}

%------------------------------------------------------------------------------
\nextq
\tf:
Let $f : X \rightarrow Y$ and $g: Y \rightarrow Z$ be functions.
If $f$ is not onto and $g$ is onto, then $g \circ f$ is not onto.
\dotfill\answerbox{T}
\\
If F, provide a counterexample
\begin{answerlong}

\end{answerlong}

%------------------------------------------------------------------------------
\nextq
\tf:
Let $f : X \rightarrow Y$ and $g: Y \rightarrow Z$ be functions.
If $f$ is 1--1 and $g$ is not 1--1, then $g \circ f$ is 1--1.
\dotfill\answerbox{F}
\\
If F, provide a counterexample
\begin{answerlong}
    $X = \{1\}, Y=\{1\}, Z=\{1,2\}$
    $g \circ f$(1) = 1 not 1-1
\end{answerlong}

%------------------------------------------------------------------------------
\nextq
\tf:
Let $f : X \rightarrow Y$ and $g: Y \rightarrow Z$ be functions.
If $f$ is not 1--1 and $g$ is 1--1, then $g \circ f$ is 1--1.
\dotfill\answerbox{T}
\\
If F, provide a counterexample
\begin{answerlong}

\end{answerlong}

%------------------------------------------------------------------------------
\nextq
Find sets $X$ and $Y$ such that 
$X$ and $Y$ are countable and $X - Y$ is $\emptyset$.
\begin{answerlong}
    $X = \{1\}, Y = \{1\}$
\end{answerlong}

%------------------------------------------------------------------------------
\nextq
Given any positive integer $n$,
find sets $X$ and $Y$ such that 
$X$ and $Y$ are countable and $|X - Y| = n$.
\begin{answerlong}
$X = \{1,...2n\},$
$Y = \{n+1,...2n\}$
\end{answerlong}

%------------------------------------------------------------------------------
\nextq
Find countable sets $X$ and $Y$ such that $X - Y$ is
infinite and countable.
\begin{answerlong}
The set X is Natural Numbers and Set Y is all positive Even Numbers, 
then, (X - Y) = N - E+ = Positive Odd Integers, which is infinite and countable.
\end{answerlong}

%------------------------------------------------------------------------------
\nextq
Compute the following matrix product:
\[
\begin{bmatrix}
1 & 2 \\
3 & 4 
\end{bmatrix}
\begin{bmatrix}
4 & 3 \\
2 & 1 
\end{bmatrix}
=
\begin{bmatrix}
\answerbox{} & \answerbox{} \\
\answerbox{} & \answerbox{}
\end{bmatrix}
\]

%------------------------------------------------------------------------------
\nextq
Find
a matrix $M$ 
with only 0s and 1s for entry
such that $M^2 =
\begin{bmatrix}
1 & 0 \\
0 & 1 
\end{bmatrix}
$
\[
M = 
\begin{bmatrix}
\answerbox{} & \answerbox{} \\
\answerbox{} & \answerbox{}
\end{bmatrix}
\]
(Hint: $M$ is made up of only 0s and 1s.)
%------------------------------------------------------------------------------
\newpage
\input{instructions.tex}
\end{document}
