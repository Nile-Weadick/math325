\newcommand\COURSE{math325}
\newcommand\ASSESSMENT{q080403}
\newcommand\ASSESSMENTTYPE{Quiz}
\newcommand\POINTS{\textwhite{xxx/xxx}}

\makeatletter
\DeclareOldFontCommand{\rm}{\normalfont\rmfamily}{\mathrm}
\DeclareOldFontCommand{\sf}{\normalfont\sffamily}{\mathsf}
\DeclareOldFontCommand{\tt}{\normalfont\ttfamily}{\mathtt}
\DeclareOldFontCommand{\bf}{\normalfont\bfseries}{\mathbf}
\DeclareOldFontCommand{\it}{\normalfont\itshape}{\mathit}
\DeclareOldFontCommand{\sl}{\normalfont\slshape}{\@nomath\sl}
\DeclareOldFontCommand{\sc}{\normalfont\scshape}{\@nomath\sc}
\makeatother

\input{myquizpreamble}
\input{yliow}
\input{\COURSE}
\textwidth=6in

\renewcommand\TITLE{\ASSESSMENTTYPE \ \ASSESSMENT}

\newcommand\topmattertwo{
\topmatter
\score \\ \\
Open \texttt{main.tex} and enter answers (look for
\texttt{answercode}, \texttt{answerbox}, \texttt{answerlong}).
Turn the page for detailed instructions.
To rebuild and view pdf, in bash shell execute \texttt{make}.
To build a gzip-tar file, in bash shell execute \texttt{make s} and
you'll get \texttt{submit.tar.gz}.
}

\newcommand\tf{T or F or M}
\newcommand\answerbox[1]{\textbox{#1}}
\newcommand\codebox[1]{\begin{console}#1\end{console}}

\usepackage{pifont}
\newcommand{\cmark}{\textred{\ding{51}}}
\newcommand{\xmark}{\textred{\ding{55}}}

\newcounter{qc}
\newcommand\nextq{
%\newpage
\addtocounter{qc}{1}
Q{\theqc}.
}

\DefineVerbatimEnvironment%
 {answercode}{Verbatim}
 {frame=single,fontsize=\footnotesize}

\newenvironment{largebox}[1]{%
 \boxparone{#1}
}
{}

\usepackage{environ}
\let\oldquote=\quote
\let\endoldquote=\endquote
\let\quote\relax
\let\endquote\relax

% ADDED 2021/09/09
\renewcommand\boxpar[1]{
 \[
  \framebox[\textwidth][c] {
   \parbox[]{\dimexpr\textwidth - 0.25cm} {#1}
  }
 \]
}

\NewEnviron{answerlong}%
  {\vspace{-1mm} \global\let\tmp\BODY\aftergroup\doboxpar}

\newcommand\doboxpar{%
  \let\quote=\oldquote
  \let\endquote=\endoldquote
  \boxpar{\tmp}
}

\newenvironment{mcq}[7]%
{% begin code
#1 \dotfill{#2}
 \begin{tightlist}
 \item[(A)] #3
 \item[(B)] #4
 \item[(C)] #5
 \item[(D)] #6
 \item[(E)] #7 
 \end{tightlist}
}%
{% end code
} 

\renewcommand\EMAIL{}
\newcommand\score{%
\vspace{-0.6in}
\begin{flushright}
Score: \answerbox{\POINTS}
\end{flushright}
\vspace{-0.4in}
\hspace{0.7in}\AUTHOR
\vspace{0.2in}
}

\newcommand\blankline{\mbox{}\\ }

\newcommand\ANSWER{\textsc{Answer:}\vspace{-2mm}}



\renewcommand\AUTHOR{nweadick1@cougars.ccis.edu} % CHANGE TO YOURS

\begin{document}
\topmattertwo

When writing results of computations, make sure it's simplified.
For instance when you write down sets,
if the answer is $\{1\}$, I do not
want to see $\{1, 1\}$.

When you write down an example or a counterexample,
be elegant and write down the simplest.
You only need to give a counterexample if asked.

$P(X)$ denotes the powerset of $X$.

In \LaTeX\, math notation is enclosed by the \verb!$! symbols.
For instance \verb!$x = a_{1} + b^{2}$! gives you $x = a_{1} + b^{2}$. 

(For more information about \LaTeX\, go to my
website
\href{http://bit.ly/yliow0}{http://bit.ly/yliow0},
click on \verb!Yes!
you are one of my students,
then look for
\href{https://drive.google.com/file/d/0BzjYrK0VFuMWZm5xV0kyR3J2Zm8/view?usp=sharing}{latex.pdf}.)

For the next few questions,
let $A = \{3, 2, \pi, 4\}$, $B = \{2,4,1,3,2,4\}$, $C = \{7, 5, 3\}$ 

%------------------------------------------------------------------------------
\nextq
$|A| =  \answerbox{ 4 } $ (Please correct!)

%------------------------------------------------------------------------------
\nextq
$|B| = \answerbox{ 4 } $

%------------------------------------------------------------------------------
\nextq
What is $A \cup B$?
\begin{answerlong}
$A \cup B = \{ 1,2,3,4,\pi\}$
\end{answerlong}

%------------------------------------------------------------------------------
\nextq
What is $A \cap B$?
\begin{answerlong}
$A \cap B = \{2,3,4\}$
\end{answerlong}

%------------------------------------------------------------------------------
\nextq
What is $A - C = \{ \answerbox{} \}$?
\begin{answerlong}
$A - C = \{ 2, \pi, 4 \}$
\end{answerlong}

%------------------------------------------------------------------------------
\nextq
What is $A \times C$?
\begin{answerlong}
$A \times C = \{ (3,7),(3,5),(3,3),(2,7),(2,5),(2,3),(pi,7),(pi,5),(pi,3),(4,7),(4,5),(4,3)\}$
\end{answerlong}

%------------------------------------------------------------------------------
\nextq
What is $P(C)$?
\begin{answerlong}
$P(C) = \{ \emptyset, \{3\}, \{5\}, \{7\}, \{3,5\}, \{3,7\}, \{5,7\}, \{3,5,7\} \}$
\end{answerlong}
%------------------------------------------------------------------------------
For the next few questions, let $X, Y, Z$ be sets (in the same universe).

\nextq
\tf: $X \cup Y = Y \cup X$
\dotfill\answerbox{T}

%------------------------------------------------------------------------------
\nextq
\tf: $X \cup (Y \cap Z) = (X \cup Y) \cap Z$.
\dotfill\answerbox{F}
\\
If F, provide a counterexample
\begin{answerlong}
$X = \{ 1 ,2\}$, $Y = \{ 2,3 \}$, $Z = \{ 3,4 \}$

$X \cup (Y \cap Z) = \{1,2,3\}$ 
$(X \cup Y) \cap Z = \{3\}$
\end{answerlong}

%------------------------------------------------------------------------------
\nextq
\tf: If $X \neq \emptyset$ and $Y \neq \emptyset$, then 
$X \cup Y \neq \emptyset$.
\dotfill\answerbox{T}
\\
If F, provide a counterexample
\begin{answerlong}

\end{answerlong}

%------------------------------------------------------------------------------
\nextq
\tf: If $X \neq \emptyset$ and $Y \neq \emptyset$, then 
$X \cap Y \neq \emptyset$.
\dotfill\answerbox{F}
\\
If F, provide a counterexample
\begin{answerlong}
$X = \{ 1 \}$, $Y = \{ 2 \}$.  $then X \cap Y = \emptyset$
\end{answerlong}

%------------------------------------------------------------------------------
\nextq
\tf: $\emptyset \subseteq X$ for any set $X$.
\dotfill\answerbox{T}

%------------------------------------------------------------------------------
\nextq
\tf: $\emptyset \in X$ for any set $X$.
\dotfill\answerbox{F}

%------------------------------------------------------------------------------
\nextq
\tf: $\emptyset \times X = \emptyset$
\dotfill\answerbox{T}

%------------------------------------------------------------------------------
\nextq
Find the simplest set $X$ and $Y$ such that $X \in Y$
and $X \subseteq Y$.
\begin{answerlong}
$X = \{ 1 \}$, $Y = \{ 1,2 \}$.
\end{answerlong}

%------------------------------------------------------------------------------
\nextq
\tf: $P(\emptyset) = \emptyset$
\dotfill\answerbox{T}

%------------------------------------------------------------------------------
\nextq
\tf: $X \subseteq Y \implies P(X) \subseteq P(Y)$.
\dotfill\answerbox{T}
\\
If F, provide a counterexample
\begin{answerlong}

\end{answerlong}

%------------------------------------------------------------------------------
\nextq
\tf: $P(X \cup Y) = P(X) \cup P(Y)$.
\dotfill\answerbox{F}
\\
If F, provide a counterexample
\begin{answerlong}
$X = \{1\}, Y = \{2\}, P(X \cup Y) = \{\emptyset, \{1\}, \{2\}, \{1,2\} \}$
$P(X) \cup P(Y) = \{\emptyset, \{1\}, \{2\}\}$
\end{answerlong}

%------------------------------------------------------------------------------
\nextq
\tf: $P(X \cap Y) = P(X) \cap P(Y)$.
\dotfill\answerbox{T}
\\
If F, provide a counterexample
\begin{answerlong}

\end{answerlong}

%------------------------------------------------------------------------------
\nextq
\tf: $P(X \times Y) = P(X) \times P(Y)$.
\dotfill\answerbox{T}
\\
If F, provide a counterexample
\begin{answerlong}

\end{answerlong}

%------------------------------------------------------------------------------
\nextq
Hans Solo and Luke Skywalker are leading a team of 10 to attack
an AT-AT and some stormtroopers. (The 10 includes Hans and Luke.)
They have decided to split into two teams.
The team attacking the AT-AT will have at least 2 members.
How many ways are there to form such a team?
(Explain your work with complete sentences.)
\begin{answerlong}
$2^10- C(10,0) + C(10,1) = 1024 - 11 = 1013$
\end{answerlong}

%------------------------------------------------------------------------------
\nextq
\tf. It is possible to construct sets $W$, $X$, $Y$, $Z$ such that
\[
|W \cap X| = |X \cap Y| = |Y \cap Z| = |Z \cap W| = 2
\]
and
\[
|W \cap X \cap Y| = |X \cap Y \cap Z| = |Y \cap Z \cap W| = |Z \cap W \cap X| = 0
\]
\dotfill\answerbox{T}
\\
If your answer is T, provide the simplest possible example.
\begin{answerlong}
    $W = \{1,2,3,4\}, X = \{3,4,5,6\}, Y = \{5,6,7,8\}, Z = \{7,8,1,2\}$
\end{answerlong}

%------------------------------------------------------------------------------
\nextq
\tf. It is possible to construct sets $W$, $X$, $Y$, $Z$ such that
\[
|W \cap X| = |X \cap Y| = |Y \cap Z| = |Z \cap W| = 2
\]
and
\[
|W \cap X \cap Y| = |X \cap Y \cap Z| = |Y \cap Z \cap W| = |Z \cap W \cap X| = 1
\]
\dotfill\answerbox{T}
\\
If your answer is T, provide the simplest possible example.
\begin{answerlong}
    $W = \{1,2,3\}, X = \{2,3,4\}, Y = \{3,4,5\}, Z = \{1,3,4\}$
\end{answerlong}

%------------------------------------------------------------------------------
\nextq
\tf: It is possible find a set $X$ and a function $f : X \rightarrow X$
such that $|X| = 4$, $|f(X)| = 3$, $|f(f(X)| = 2$, $|f(f(f(X)))| = 1$
\dotfill\answerbox{T}
\\
If your answer is T, provide the simplest possible example.
\begin{answerlong}
    $X = \{1,2,3,4\}$
\end{answerlong}

%------------------------------------------------------------------------------
\nextq
\tf:
If $f : X \rightarrow Y$ and $g: Y \rightarrow Z$ are onto (i.e., surjective)
functions, then $g \circ f: X \rightarrow Z$ is also onto.
\dotfill\answerbox{T}
\\
If F, provide a counterexample.
\begin{answerlong}

\end{answerlong}

%------------------------------------------------------------------------------
\nextq
\tf:
If $f : X \rightarrow Y$ and $g: Y \rightarrow Z$ are 1--1 
(i.e., injective)
functions, then $g \circ f: X \rightarrow Z$ is also 1--1.
\dotfill\answerbox{T}
\\
If F, provide a counterexample.
\begin{answerlong}

\end{answerlong}

%------------------------------------------------------------------------------
\nextq
\tf:
Let $f : X \rightarrow Y$ and $g: Y \rightarrow Z$ be functions.
If $f$ is onto and $g$ is not onto, then $g \circ f$ is onto.
\dotfill\answerbox{F}
\\
If F, provide a counterexample.
\begin{answerlong}
    $X = \{1\}, Y=\{1\}, Z=\{1,2\}$
    $g \circ f$(1) = 1 not onto
\end{answerlong}

%------------------------------------------------------------------------------
\nextq
\tf:
Let $f : X \rightarrow Y$ and $g: Y \rightarrow Z$ be functions.
If $f$ is not onto and $g$ is onto, then $g \circ f$ is not onto.
\dotfill\answerbox{T}
\\
If F, provide a counterexample
\begin{answerlong}

\end{answerlong}

%------------------------------------------------------------------------------
\nextq
\tf:
Let $f : X \rightarrow Y$ and $g: Y \rightarrow Z$ be functions.
If $f$ is 1--1 and $g$ is not 1--1, then $g \circ f$ is 1--1.
\dotfill\answerbox{F}
\\
If F, provide a counterexample
\begin{answerlong}
    $X = \{1\}, Y=\{1\}, Z=\{1,2\}$
    $g \circ f$(1) = 1 not 1-1
\end{answerlong}

%------------------------------------------------------------------------------
\nextq
\tf:
Let $f : X \rightarrow Y$ and $g: Y \rightarrow Z$ be functions.
If $f$ is not 1--1 and $g$ is 1--1, then $g \circ f$ is 1--1.
\dotfill\answerbox{T}
\\
If F, provide a counterexample
\begin{answerlong}

\end{answerlong}

%------------------------------------------------------------------------------
\nextq
Find sets $X$ and $Y$ such that 
$X$ and $Y$ are countable and $X - Y$ is $\emptyset$.
\begin{answerlong}
    $X = \{1\}, Y = \{1\}$
\end{answerlong}

%------------------------------------------------------------------------------
\nextq
Given any positive integer $n$,
find sets $X$ and $Y$ such that 
$X$ and $Y$ are countable and $|X - Y| = n$.
\begin{answerlong}
$X = \{1,...2n\},$
$Y = \{n+1,...2n\}$
\end{answerlong}

%------------------------------------------------------------------------------
\nextq
Find countable sets $X$ and $Y$ such that $X - Y$ is
infinite and countable.
\begin{answerlong}
The set X is Natural Numbers and Set Y is all positive Even Numbers, 
then, (X - Y) = N - E+ = Positive Odd Integers, which is infinite and countable.
\end{answerlong}

%------------------------------------------------------------------------------
\nextq
Compute the following matrix product:
\[
\begin{bmatrix}
1 & 2 \\
3 & 4 
\end{bmatrix}
\begin{bmatrix}
4 & 3 \\
2 & 1 
\end{bmatrix}
=
\begin{bmatrix}
\answerbox{} & \answerbox{} \\
\answerbox{} & \answerbox{}
\end{bmatrix}
\]

%------------------------------------------------------------------------------
\nextq
Find
a matrix $M$ 
with only 0s and 1s for entry
such that $M^2 =
\begin{bmatrix}
1 & 0 \\
0 & 1 
\end{bmatrix}
$
\[
M = 
\begin{bmatrix}
\answerbox{} & \answerbox{} \\
\answerbox{} & \answerbox{}
\end{bmatrix}
\]
(Hint: $M$ is made up of only 0s and 1s.)
%------------------------------------------------------------------------------
\newpage

\textsc{Instructions}

In \verb!main.tex! change the email address in
\begin{console}
\renewcommand\AUTHOR{jdoe5@cougars.ccis.edu} 
\end{console}
yours.
In the bash shell, execute \lq\lq \verb!make!" to recompile \verb!main.pdf!.
Execute \lq\lq \verb!make v!" to view \verb!main.pdf!.
Execute \lq\lq \verb!make s!" to create \verb!submit.tar.gz! for submission.

For each question, you'll see boxes for you to fill.
You write your answers in \verb!main.tex! file.
For small boxes, if you see
\begin{console}[frame=single=single,fontsize=\small]
1 + 1 = \answerbox{}.
\end{console}
you do this:
\begin{console}[frame=single=single,fontsize=\small]
1 + 1 = \answerbox{2}.
\end{console}
\verb!answerbox! will also appear in
\lq\lq true/false" and \lq\lq multiple-choice"
questions.

For longer answers that needs typewriter font, if you see
\begin{console}[frame=single=single, fontsize=\small]
Write a C++ statement that declares an integer variable name x.
\begin{answercode}
\end{answercode}
\end{console}
you do this:
\begin{console}[frame=single=single, fontsize=\small]
Write a C++ statement that declares an integer variable name x.
\begin{answercode}
int x;
\end{answercode}
\end{console}
\verb!answercode! will appear in questions asking for
code, algorithm, and program output.
In this case, indentation and spacing is significant.
For program output, I do look at spaces and newlines.

For long answers (not in typewriter font) if you see
\begin{console}[frame=single=single, fontsize=\small]
What is the color of the sky?
\begin{answerlong}
\end{answerlong}
\end{console}
you can write
\begin{console}[frame=single=single, fontsize=\small]
What is the color of the sky?
\begin{answerlong}
The color of the sky is blue.
\end{answerlong}
\end{console}
For students beyond 245: You can put \LaTeX\ commands in
\verb!answerbox! and 
\verb!answerlong!.

A question that begins with \lq\lq T or F or M"
requires you to identify whether it is true or
false, or meaningless.
\lq\lq Meaningless" means something's wrong with the statement and
it is not well-defined.
Something like \lq\lq $1 +_2$" or \lq\lq $\{2\}^{\{3\}}$" is not
well-defined.
Therefore a question such as
\lq\lq Is $42 = 1 +_2$ true or false?" or
\lq\lq Is $42 = \{2\}^{\{3\}}$ true or false?"
does not make sense.
\lq\lq Is $P(42) = \{42\}$ true or false?" is meaningless because $P(X)$
is only defined if $X$ is a set.
For \lq\lq Is 1 + 2 + 3 true or false?", \lq\lq 1 + 2 + 3" is well--defined but
as a
\lq\lq numerical expression", not as a \lq\lq proposition", i.e.,
it cannot be true or false.
Therefore \lq\lq Is 1 + 2 + 3 true or false?" is also not a well-defined
question.

When writing results of computations, make sure it's simplified.
For instance write $2$ instead of $1 + 1$.
When you write down sets,
if the answer is $\{1\}$, I do not
want to see $\{1, 1\}$.

When writing a counterexample, always write the simplest.

Here are some examples (see \verb!instructions.tex! for details):

\begin{enumerate}

 \item \tf: 1 + 1 = 2 \dotfill\answerbox{T}
 
 \item \tf: 1 + 1 = 3 \dotfill\answerbox{F}
 
 \item \tf: $1 +^2 =$ \dotfill\answerbox{M}
 
 \item $1 + 2 =$ \answerbox{3}
 
 \item Write a C++ statement to declare an integer variable named
 \verb!x!.
 \begin{answercode}
int x;
 \end{answercode}

 \item Solve $x^2 - 1 = 0$.
 \begin{answerlong}
 Since $x^2 - 1 = (x-1)(x+1)$, $x^2 - 1 = 0$ implies $(x-1)(x+1)=0$.
 Therefore $x - 1 = 0$ or $x = -1$.
 Hence $x = 1$ or $x = -1$.
 \end{answerlong}

 \item
 \begin{mcq}
 {Which is true?}{\answerbox{C}}
 {$1+1=0$}
 {$1+1=1$}
 {$1+1=2$}
 {$1+1=3$}
 {$1+1=4$}
 \end{mcq}


\end{enumerate}

\end{document}
