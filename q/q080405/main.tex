\input{thispreamble.tex}

\renewcommand\AUTHOR{nweadick1@cougars.ccis.edu} % CHANGE TO YOURS

\begin{document}
\topmattertwo

This is a 3-min quiz.
If you cannot complete this quiz in 10 minutes, you are strongly
advised to study my notes carefully.

For all question, you only need to write down the answers you have
computed. Do not show your justification/computation.
Some \LaTeX\ for you:
\begin{enumerate}[nosep]
\li Product: \verb!$a \cdot b$! gives you $a \cdot b$
\li Superscript: \verb!$a^{n}$! gives you $a^{n}$
\li Fraction: \verb!$\frac{a}{b}$! gives you $\frac{a}{b}$ or
\verb!$a/b$! gives you $a/b$ 
\li Binomial: \verb!$binom{a}{b}$! gives you $\binom{a}{b}$
\li Large parentheses:
\verb!$\left( a \right)^b$! gives you $\left( a \right)^b$ 
\end{enumerate}
It's a common practice to wrote $(-2)^n$ as $(-1)^n 2^n$, i.e.,
so that $(-1)^n$ is factored as a \lq\lq sign".
It's also a common practice to simplify binomial coefficents
such as $\binom{n + 20}{n + 1}$ to
$\binom{n + 20}{19}$,
i.e., no $n$ in the bottom.
This is from $\binom{n}{r} = \binom{n}{n - r}$.

%------------------------------------------------------------------------------
\nextq
Write down the coefficient of $x^n$ for 
$(1 - 7x)^5$.
\\
\ANSWER
\begin{answerlong}
\[
\begin{cases}
$$\binom{5}{n} \cdot (-7)^n$$ \text{ for $0 \leq n \leq 5$} \\
$0$ \text{ otherwise}
\end{cases}
\]
\end{answerlong}

%------------------------------------------------------------------------------
\nextq
Write down the coefficient of $x^n$ for 
$(9x - 2)^8$.
\\
\ANSWER

\begin{answerlong}
\[
\begin{cases}
$$\binom{8}{n} \cdot (-2)^8 \cdot \frac{-9}{2}^n$$ \text{ for $0 \leq n \leq 8$} \\
$0$ \text{ otherwise}
\end{cases}
\]
\end{answerlong}

%------------------------------------------------------------------------------
\nextq
Write down the coefficient of $x^n$ for 
$(9x^2 + 2)^8$.
\\
\ANSWER
\begin{answerlong}
\[
\begin{cases}
$$\binom{8}{\frac{n}{2}} \cdot (2)^8 \cdot (\frac{9}{n})^\frac{n}{2}$$ \text{ for $0 \leq n \leq 16$, where n must be even} \\
$0$ \text{ otherwise}
\end{cases}
\]
\end{answerlong}

%------------------------------------------------------------------------------
\newpage
\input{instructions.tex}
\end{document}
