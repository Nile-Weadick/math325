\input{thispreamble.tex}

\renewcommand\AUTHOR{nweadick1@cougars.ccis.edu} % CHANGE TO YOURS

\begin{document}
\topmattertwo

This is a 10-min quiz.
If you cannot complete this quiz in 10 minutes, you are strongly
advised to study my notes carefully.

For all question, you only need to write down the answers you have
computed. Do not show your justification/computation.
Some \LaTeX\ for you:
\begin{enumerate}[nosep]
\li Product: \verb!$a \cdot b$! gives you $a \cdot b$
\li Superscript: \verb!$a^{n}$! gives you $a^{n}$.
\li Square root: \verb!$\sqrt{a}$! gives you $\sqrt{a}$.
\li Fraction: \verb!$\frac{a}{b}$! gives you $\frac{a}{b}$ or
\verb!$a/b$! gives you $a/b$ 
\li Binomial: \verb!$binom{a}{b}$! gives you $\binom{a}{b}$
\li Large parentheses:
\verb!$\left( a \right)^b$! gives you $\left( a \right)^b$ 
\end{enumerate}
It's a common practice to wrote $(-2)^n$ as $(-1)^n 2^n$, i.e.,
so that $(-1)^n$ is factored as a \lq\lq sign".
It's also a common practice to simplify binomial coefficents
such as $\binom{n + 20}{n + 1}$ to
$\binom{n + 20}{19}$,
i.e., no $n$ in the bottom.
This is from $\binom{n}{r} = \binom{n}{n - r}$.

%------------------------------------------------------------------------------
Consider the recurrence
\[
a_n = 4a_{n-1} + a_{n-2}
\]
for $n \geq 1$ and $a_0 = 1, a_1 = 2$.
\\

\nextq
What is the characteristic equation for the $a_n$ recurrence?
\\
\ANSWER
\begin{answerlong}
$ x^2-4x-1 = 0 $
\end{answerlong}
(No explanation needed.)

\nextq
What are the roots of the characteristic equation?
\\
\ANSWER
\begin{answerlong}
$r_1 = 2+\sqrt{5}, r_2 = 2-\sqrt{5}$
\end{answerlong}
(No explanation needed.)

\nextq
What is the general form of the closed form for $a_n$?
\\
\ANSWER
\begin{answerlong}
$a_n = C_1 \cdot (2+\sqrt{5})^n + C_2 \cdot (2-\sqrt{5})^n$, where $C_1, C_2$ are constants to be determined.
\end{answerlong}
(No explanation needed.)

\nextq
Using the method of characteristic equation, find a closed form for $a_n$.
\\
\ANSWER
\begin{answerlong}
    $a_n = \frac{1}{2} \cdot (2+\sqrt{5})^n + \frac{1}{2} \cdot (2-\sqrt{5})^n$
\end{answerlong}
(No explanation needed.)

\textsc{Note.}
Just because the method of characteristic equation makes solving recurrences
easier, it does not mean that generating functions are useless.
Generating functions \textit{explains} why the roots of the characteristic
equation can be used to find closed.

%------------------------------------------------------------------------------
\newpage
\input{instructions.tex}
\end{document}
