
(a)
If $a_n = -2^{n+1}$, then
\begin{align*}
3a_{n-1} + 2^n
&= 3(-2^{(n-1)+1}) + 2^n \\
&= 3(-2^n) + 2^n \\
&= (3(-1) + 1) \cdot 2^n \\
&= -2 \cdot 2^n \\
&= -2^{n + 1} \\
&= a_n
\end{align*}
Hence
\[
a_n^{(p)} = -2^{n+1}
\]
is a (particular) solution of $a_n = 3a_{n - 1} + 2^n$.


(b)
The characteristic equation of
$a_n = 3a_{n - 1}$
is
\[
x - 3
\]
The root of the charactistic equation is $r_1 = 3$.
Therefore the general solution to
$a_n = 3a_{n - 1}$ is $C_1 3^n$, i.e.,
\[
a_n^{(h)} = C_1 3^n
\]
where $C_1$ is a constant.
Hence the general solution to
$a_n = 3a_{n - 1} + 2^n$ is
\[
a_n = a_n^{(h)} + a_n^{(p)} = C_1 3^n -2^{n+1}
\]
where $C_1$ is a constant.

(c)
Using the base case,
\[
1 = a_0 = C_1 3^0 -2^{0+1} = C_1 - 2
\]
Hence $C_1 = 3$.
Therefore
\[
a_n = 3 \cdot 3^n -2^{n+1}
\]
\qed

Check:
\begin{Verbatim}[frame=single,fontsize=\small]
def an(n):
    if n == 0:
        return 1
    else:
        return 3 * an(n - 1) + 2**n

def bn(n):
    return 3 * 3**n - 2**(n + 1)

for n in range(10):
    print(n, an(n), bn(n))     
\end{Verbatim}
has output
\begin{Verbatim}[frame=single,fontsize=\small]
0 1 1
1 5 5
2 19 19
3 65 65
4 211 211
5 665 665
6 2059 2059
7 6305 6305
8 19171 19171
9 58025 58025
\end{Verbatim}
