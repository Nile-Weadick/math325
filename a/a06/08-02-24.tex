
(a)
If $a_n = n 2^n$, then
\begin{align*}
  2 a_{n - 1} + 2^n
  &= 2 (n - 1) 2^{n - 1} + 2^n \\
  &= (n - 1) 2^{n} + 2^n \\
  &= n 2^{n} \\
  &= a_n
\end{align*}
Hence
\[
a_n^{(p)} = n 2^n
\]
is a (particular) solution of $a_n = 2 a_{n - 1} + 2^n$.

(b)
The characteristic equation of
$a_n = 2 a_{n - 1}$
is
\[
x - 2
\]
The root of the charactistic equation is $r_1 = 2$.
Therefore the general solution to
$a_n = 3a_{n - 1}$ is $C_1 2^n$, i.e.,
\[
a_n^{(h)} = C_1 2^n
\]
where $C_1$ is a constant.
Hence the general solution to
$a_n = 3a_{n - 1} + 2^n$ is
\[
a_n = a_n^{(h)} + a_n^{(p)} = C_1 2^n + n2^n
\]
where $C_1$ is a constant.

(c)
Using the base case,
\[
2 = a_0 = C_1 2^0 + 0 = C_1
\]
Hence $C_1 = 2$.
Therefore
\[
a_n = 2 \cdot 2^n + n2^n = (2 + n) 2^n 
\]
\qed

Check:
\begin{Verbatim}[frame=single,fontsize=\small]
def an(n):
    if n == 0:
        return 2
    else:
        return 2 * an(n - 1) + 2**n

def bn(n):
    return (2 + n) * 2**n

for n in range(10):
    print(n, an(n), bn(n))     
\end{Verbatim}
has output
\begin{Verbatim}[frame=single,fontsize=\small]
0 2 2
1 6 6
2 16 16
3 40 40
4 96 96
5 224 224
6 512 512
7 1152 1152
8 2560 2560
9 5632 5632
\end{Verbatim}
