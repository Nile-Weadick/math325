
(a)
If $a_n = An + B$ is a particular solution of $a_n = 2 a_{n - 1} + n + 5$, then
\begin{align*}
  An + B
  &= 2 (A(n - 1) + B) + n + 5 \\ 
  &= (2A + 1) n + (-2A + 2B + 5) \\ 
\end{align*}
Hence
\begin{align*}
  A &= 2A + 1 \\
  B &= -2A + 2B + 5 
\end{align*}
i.e.,
\begin{align*}
  A &= - 1 \\
  B &= 2A - 5 
\end{align*}
Hence $A = -1$ and $B = -7$.
Therefore
\[
a^{(p)} = -n -7
\]
is a particular solution of $a_n = 2 a_{n - 1} + n + 5$.


(b)
The characteristic equation of $a_n = 2 a_{n - 1}$ is
\[
x - 2
\]
The root of the chracteristic equation is $r_1 = 2$.
Therefore the general solution of $a_n = 2a_{n - 1}$ is
\[
a^{(h)} = C_1 2^n
\]
Hence the general solution of $a_n = 2 a_{n - 1} + n + 5$ is
\[
a_n = a^{(h)} +  a^{(p)} = C_1 2^n - n - 7
\]


(c)
From $a_0 = 4$,
\[
4 = a_0 = C_1 2^0 - 0 - 7 = C_1 - 7
\]
i.e., $C_1 = 11$.
Hence the solution is
\[
a_n = 11 \cdot 2^n - n - 7
\]

Check:
\begin{Verbatim}[frame=single,fontsize=\small]
def an(n):
    if n == 0:
        return 4
    else:
        return 2 * an(n - 1) + n + 5

def bn(n):
    return 11 * 2**n - n - 7

for n in range(10):
    print(n, an(n), bn(n))     
\end{Verbatim}
has output
\begin{Verbatim}[frame=single,fontsize=\small]
0 4 4
1 14 14
2 35 35
3 78 78
4 165 165
5 340 340
6 691 691
7 1394 1394
8 2801 2801
9 5616 5616
\end{Verbatim}
