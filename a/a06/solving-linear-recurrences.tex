\textsc{Solving linear recurrences}

The first method is always by generating functions and is the best.

The second method is to use characteristic equation.
For instance if the linear recurrence is
\[
a_n = 5a_{n-1} - 6a_{n-2}
\]
then the characteristic equation is
\[
x^2 - 5x + 6
\]
which has roots $r_1 = 2, r_2 = 3$.
Therefore the general form for $a_n$ is
\[
a_n = C_1 2^n + C_2 3^n
\]
If the linear recurrence in
\[
b_n = 4b_{n-1} - 4b_{n-2}
\]
then the characteristic equation is
\[
x^2 - 4x + 4
\]
which has roots $r_1 = r_2 = 2$.
Therefore the general form for $b_n$ is
\[
b_n = C'_1 2^n + C'_2 n2^n
\]
In both cases, to find the constants ($C_i$ or $C'_i$), you need
two base conditions.

What about the nonhomogeneous case?
For instance
\[
a_n = 5a_{n-1} - 6a_{n-2} + \underline{3n + 1}
\]
If you use the generating function, you will find the closed form
for $a_n$. (See question below.)
But what if you are using the characteristic equation method?

\textsc{Step 1.} You ignore the nonhomogeneous term and solve
\[
a_n = 5a_{n-1} - 6a_{n-2}
\]
This is sometimes called the homogeneous part of the linear recurrence.
Sometimes to differentiate this and the original,
you can write
\[
a_n^{(h)} = 5a_{n-1}^{(h)} - 6a_{n-2}^{(h)}
\]
i.e., the sequence is $a_0^{(h)}, a_1^{(h)}, a_2^{(h)}, ...$.
So
\[
a_n^{(h)} = C_1 2^n + C_2 3^n
\]

\textsc{Step 2.}
You make an educated guess that involves the
nonhomogeneous term.
(This is the part I don't like about characteristic equation method.)
The nonhomogeneous part of
\[
a_n = 5a_{n-1} - 6a_{n-2} + \underline{3n + 1}
\]
is a polynomial $3n+1$ of degree 1.
You try and see if $a_n^{(p)} = An + B$ is a solution of 
\[
a_n = 5a_{n-1} - 6a_{n-2} + \underline{3n + 1}
\]
Substituting $a_n^{(p)} = An + B$ into the linear recurrence, you get
\[
An + B = 5(A(n-1) + B) - 6(A(n-2) + B) + 3n + 1
\]
i.e.,
\begin{align*}
An + B
&= 5(A(n-1) + B) - 6(A(n-2) + B) + 3n + 1 \\
&= 5(An - A + B) - 6(An - 2A + B) + 3n + 1 \\
&= (5A - 6A + 3)n + (-5A + 5B + 12A - 6B + 1) \\
&= (3 - A)n + (7A - B + 1)
\end{align*}
which means $A = 3 - A$ and $B = 7A - B + 1$, i.e.,
$A = 3/2$ and $B = 23/4$.
Therefore
\[
a_n^{(p)} = An + B = \frac{3}{2}n + \frac{23}{4}
\]

\textsc{Step 3}.
You add the homogeneous and particular solutions:
\[
a_n = a_n^{(h)} + a_n^{(p)} = C_1 2^n + C_2 3^n + \frac{3}{2}n + \frac{23}{4}  
\]
Suppose the base cases are $a_0 = 2, a_1 = 3$.
You can then solve for $C_1$ and $C_2$ with the base cases:
\begin{align*}
2 &= a_0 = C_1 + C_2 + \frac{23}{4} \\ 
3 &= a_1 = 2C_1 + 3C_2 + \frac{3}{2} + \frac{23}{4}
\end{align*}
i.e.,
\begin{align*}
C_1 + C_2 &= 2 - \frac{23}{4} = -\frac{15}{4} \\ 
2C_1 + 3C_2 &= 3 - \frac{29}{4} = -\frac{17}{4}
\end{align*}
This yields
$C_1 = -15/4 - C_2 = -15/4 - 13/4 = -7$ and $C_2 = 13/4$.
Hence
\[
a_n = -7 \cdot 2^n + \frac{13}{4} 3^n + \frac{3}{2}n + \frac{23}{4}  
\]
Here's a quick check:
\begin{console}[frame=single,fontsize=\footnotesize]
def an(n):
    if n == 0: return 2
    elif n == 1: return 3
    else: return 5 * an(n - 1) - 6 * an(n - 2) + 3 * n + 1

def bn(n):
    return -7 * 2**n + (13/4.0) * 3**n + (1.5) * n + (23.0/4)

for n in range(10):
    print(n, an(n), bn(n))
\end{console}
The output is
\begin{console}[frame=single,fontsize=\footnotesize]
0 2 2.0
1 3 3.0
2 10 10.0
3 42 42.0
4 163 163.0
5 579 579.0
6 1936 1936.0
7 6228 6228.0
8 19549 19549.0
9 60405 60405.0
\end{console}

The problem with the characteristic equation method is that you have to guess
the particular solution.
\begin{itemize}
\item Suppose the recurrence relation is $a_n = 5a_{n-1} - 6a_{n-2} + 7n^2$.
Then you guess $a_n^{(p)} = A + Bn + Cn^2$.

\item Suppose the recurrence relation is $a_n = 5a_{n-1} - 6a_{n-2} + 2 + 7n^5$.
Then you guess $a_n^{(p)} = A + Bn + Cn^2 + Dn^3 + En^4 + Fn^5$.

\item Suppose the recurrence relation is $a_n = 5a_{n-1} - 6a_{n-2} + 7 \cdot 5^n$.
Then you guess $a_n^{(p)} = A \cdot 5^n$.

\item Suppose the recurrence relation is $a_n = 5a_{n-1} - 6a_{n-2} + (7 + 10n) \cdot 5^n$.
Then you guess $a_n^{(p)} = (A + Bn) \cdot 5^n$.

\item Suppose the recurrence relation is $a_n = 5a_{n-1} - 6a_{n-2} + 7 \cdot 2^n$.
Then you cannot guess $a_n^{(p)} = A \cdot 2^n$. Why?
Because the homogeneous solution has a $C_12^n$ and this overlaps with your
guess.
You have to guess $a_n^{(p)} = C \cdot n2^n$

\item Suppose the recurrence relation is $a_n = 5a_{n-1} - 6a_{n-2} + (7 + 9n)\cdot 2^n$.
Then you cannot guess $a_n^{(p)} = (A + Bn) \cdot 2^n$. Why?
Because the homogeneous solution has a $C_12^n$
and your guess contains $A 2^n$ with overlaps with $C_12^n$.
You have to guess $a_n^{(p)} = (A + Bn) \cdot n2^n$

\item Suppose the recurrence relation is $a_n = 4a_{n-1} - 4a_{n-2} + (7 + 9n)\cdot 2^n$.
The homogeneous solution is $a_n^{(h)} = C_1 2^n + C_2 n2^n$.
Then you cannot guess $a_n^{(p)} = (A + Bn) \cdot 2^n$. Why?
Because the homogeneous solution has a $C_12^n$
and your guess contains $A 2^n$ with overlaps with $C_12^n$.
Your guess also cannot be $a_n^{(p)} = (A + Bn) \cdot n2^n$. Why?
Because the homogeneous solution has a $C_2 n2^n$
and your guess contains $Bn 2^n$ with overlaps with $C_2 n2^n$.
You have to guess $a_n^{(p)} = (A + Bn) \cdot n^2 2^n$.

\item So in general if the nonhomogeneous part looks like
$\text{(polynomial of degree d)} \cdot r^n$,
you write down 
$\text{(polynomial of degree d)} \cdot r^n$,
where the polynomial has constants $A,B,C,...$ for the coefficients
for the polynomial.
You check check if you need to change $r^n$ to
$nr^n$, $n^2r^n$, etc. until your guess does not have any term
that overlaps with the any term of the homogeneous solution.

\item See theorem 6 in Rosen.
\end{itemize}

The problem is that you have memorize the above cases or memorize
Theorem 6 of Rosen.
And theorem 6 goes not even cover all cases.
On the other hand, generating functions always work, as long as
when $a_n$'s are converted into a power series, the corresponding
power series for the 
the nonhomogenous part becomes a power series that is one of the
standard forms or can be manipulated into one of the standard forms.
That's why I prefer to simply work with power series.
