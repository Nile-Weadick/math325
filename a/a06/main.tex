\input{thispreamble.tex}

\renewcommand\AUTHOR{jdoe5@cougars.ccis.edu} % CHANGE TO YOURS

\begin{document}
\topmattertwo

\textsc{Objectives}
\begin{enumerate}[nosep]
\item Use generating functions to compute closed forms for linear recurrences.
\item Use generating functions to compute/prove identities.
\item Use characteristic equation to compute closed forms for
linear recurrences.
\end{enumerate}
\vspace{1cm}



The following are practice problems for self-study:
\begin{enumerate}[nosep]
\li Rosen 8th edition, section 8.2: Odd numbered problems 1-53 except 41,49,51.
\end{enumerate}
For this assignment the problems you need to solve are
\begin{enumerate}[nosep]
\li Q1 and Q2 and 
\li Rosen 8th edition, section 8.2:
\underline{questions 4, 12, 28, 30}.
\end{enumerate}
Explain your work completely using the solutions provided as
guide and examples on how to write math properly.

In \LaTeX\, math notation is enclosed by the \verb!$! symbols.
For instance \verb!$x = a_{1} + b^{2}$! gives you $x = a_{1} + b^{2}$.
For emphasis you can write also write it as
\verb!\[ x = a_{1} + b^{2} \]!
to center your math:
\[
x = a_{1} + b^{2}
\]
For binomial coefficients, \verb!$\binom{5}{2}$! will give you 
$\binom{5}{2}$.

If you are writing a series of computations you can align them like this:
\begin{Verbatim}[frame=single,fontsize=\footnotesize]
\begin{align*}
a(x) &= \sum_{n=0}^\infty a_n x^n \\
     &= a_0 + a_1 x + \sum_{n=2}^\infty a_n x^n \\
     &= 123 + 234 x + \sum_{n=2}^\infty (a_{n-1} + 1) x^n \\
\end{align*}
\end{Verbatim}
gives you this:
\begin{align*}
a(x) &= \sum_{n=0}^\infty a_n x^n \\
     &= a_0 + a_1 x + \sum_{n=2}^\infty a_n x^n \\
     &= 123 + 234 x + \sum_{n=2}^\infty (a_{n-1} + 1) x^n \\
\end{align*}

For more information about \LaTeX\, go to my
website
\href{http://bit.ly/yliow0}{http://bit.ly/yliow0},
click on \verb!Yes!
you are one of my students,
then look for
\href{https://drive.google.com/file/d/0BzjYrK0VFuMWZm5xV0kyR3J2Zm8/view?usp=sharing}{latex.pdf}.)
Even easier: ask questions in CCCS discord.

Draw a box around your final answer if the question required
an explicit answer (i.e., if it's not a proof question).

\newpage
\textsc{Solving linear recurrences}

The first method is always by generating functions and is the best.

The second method is to use characteristic equation.
For instance if the linear recurrence is
\[
a_n = 5a_{n-1} - 6a_{n-2}
\]
then the characteristic equation is
\[
x^2 - 5x + 6
\]
which has roots $r_1 = 2, r_2 = 3$.
Therefore the general form for $a_n$ is
\[
a_n = C_1 2^n + C_2 3^n
\]
If the linear recurrence in
\[
b_n = 4b_{n-1} - 4b_{n-2}
\]
then the characteristic equation is
\[
x^2 - 4x + 4
\]
which has roots $r_1 = r_2 = 2$.
Therefore the general form for $b_n$ is
\[
b_n = C'_1 2^n + C'_2 n2^n
\]
In both cases, to find the constants ($C_i$ or $C'_i$), you need
two base conditions.

What about the nonhomogeneous case?
For instance
\[
a_n = 5a_{n-1} - 6a_{n-2} + \underline{3n + 1}
\]
If you use the generating function, you will find the closed form
for $a_n$. (See question below.)
But what if you are using the characteristic equation method?

\textsc{Step 1.} You ignore the nonhomogeneous term and solve
\[
a_n = 5a_{n-1} - 6a_{n-2}
\]
This is sometimes called the homogeneous part of the linear recurrence.
Sometimes to differentiate this and the original,
you can write
\[
a_n^{(h)} = 5a_{n-1}^{(h)} - 6a_{n-2}^{(h)}
\]
i.e., the sequence is $a_0^{(h)}, a_1^{(h)}, a_2^{(h)}, ...$.
So
\[
a_n^{(h)} = C_1 2^n + C_2 3^n
\]

\textsc{Step 2.}
You make an educated guess that involves the
nonhomogeneous term.
(This is the part I don't like about characteristic equation method.)
The nonhomogeneous part of
\[
a_n = 5a_{n-1} - 6a_{n-2} + \underline{3n + 1}
\]
is a polynomial $3n+1$ of degree 1.
You try and see if $a_n^{(p)} = An + B$ is a solution of 
\[
a_n = 5a_{n-1} - 6a_{n-2} + \underline{3n + 1}
\]
Substituting $a_n^{(p)} = An + B$ into the linear recurrence, you get
\[
An + B = 5(A(n-1) + B) - 6(A(n-2) + B) + 3n + 1
\]
i.e.,
\begin{align*}
An + B
&= 5(A(n-1) + B) - 6(A(n-2) + B) + 3n + 1 \\
&= 5(An - A + B) - 6(An - 2A + B) + 3n + 1 \\
&= (5A - 6A + 3)n + (-5A + 5B + 12A - 6B + 1) \\
&= (3 - A)n + (7A - B + 1)
\end{align*}
which means $A = 3 - A$ and $B = 7A - B + 1$, i.e.,
$A = 3/2$ and $B = 23/4$.
Therefore
\[
a_n^{(p)} = An + B = \frac{3}{2}n + \frac{23}{4}
\]

\textsc{Step 3}.
You add the homogeneous and particular solutions:
\[
a_n = a_n^{(h)} + a_n^{(p)} = C_1 2^n + C_2 3^n + \frac{3}{2}n + \frac{23}{4}  
\]
Suppose the base cases are $a_0 = 2, a_1 = 3$.
You can then solve for $C_1$ and $C_2$ with the base cases:
\begin{align*}
2 &= a_0 = C_1 + C_2 + \frac{23}{4} \\ 
3 &= a_1 = 2C_1 + 3C_2 + \frac{3}{2} + \frac{23}{4}
\end{align*}
i.e.,
\begin{align*}
C_1 + C_2 &= 2 - \frac{23}{4} = -\frac{15}{4} \\ 
2C_1 + 3C_2 &= 3 - \frac{29}{4} = -\frac{17}{4}
\end{align*}
This yields
$C_1 = -15/4 - C_2 = -15/4 - 13/4 = -7$ and $C_2 = 13/4$.
Hence
\[
a_n = -7 \cdot 2^n + \frac{13}{4} 3^n + \frac{3}{2}n + \frac{23}{4}  
\]
Here's a quick check:
\begin{console}[frame=single,fontsize=\footnotesize]
def an(n):
    if n == 0: return 2
    elif n == 1: return 3
    else: return 5 * an(n - 1) - 6 * an(n - 2) + 3 * n + 1

def bn(n):
    return -7 * 2**n + (13/4.0) * 3**n + (1.5) * n + (23.0/4)

for n in range(10):
    print(n, an(n), bn(n))
\end{console}
The output is
\begin{console}[frame=single,fontsize=\footnotesize]
0 2 2.0
1 3 3.0
2 10 10.0
3 42 42.0
4 163 163.0
5 579 579.0
6 1936 1936.0
7 6228 6228.0
8 19549 19549.0
9 60405 60405.0
\end{console}

The problem with the characteristic equation method is that you have to guess
the particular solution.
\begin{itemize}
\item Suppose the recurrence relation is $a_n = 5a_{n-1} - 6a_{n-2} + 7n^2$.
Then you guess $a_n^{(p)} = A + Bn + Cn^2$.

\item Suppose the recurrence relation is $a_n = 5a_{n-1} - 6a_{n-2} + 2 + 7n^5$.
Then you guess $a_n^{(p)} = A + Bn + Cn^2 + Dn^3 + En^4 + Fn^5$.

\item Suppose the recurrence relation is $a_n = 5a_{n-1} - 6a_{n-2} + 7 \cdot 5^n$.
Then you guess $a_n^{(p)} = A \cdot 5^n$.

\item Suppose the recurrence relation is $a_n = 5a_{n-1} - 6a_{n-2} + (7 + 10n) \cdot 5^n$.
Then you guess $a_n^{(p)} = (A + Bn) \cdot 5^n$.

\item Suppose the recurrence relation is $a_n = 5a_{n-1} - 6a_{n-2} + 7 \cdot 2^n$.
Then you cannot guess $a_n^{(p)} = A \cdot 2^n$. Why?
Because the homogeneous solution has a $C_12^n$ and this overlaps with your
guess.
You have to guess $a_n^{(p)} = C \cdot n2^n$

\item Suppose the recurrence relation is $a_n = 5a_{n-1} - 6a_{n-2} + (7 + 9n)\cdot 2^n$.
Then you cannot guess $a_n^{(p)} = (A + Bn) \cdot 2^n$. Why?
Because the homogeneous solution has a $C_12^n$
and your guess contains $A 2^n$ with overlaps with $C_12^n$.
You have to guess $a_n^{(p)} = (A + Bn) \cdot n2^n$

\item Suppose the recurrence relation is $a_n = 4a_{n-1} - 4a_{n-2} + (7 + 9n)\cdot 2^n$.
The homogeneous solution is $a_n^{(h)} = C_1 2^n + C_2 n2^n$.
Then you cannot guess $a_n^{(p)} = (A + Bn) \cdot 2^n$. Why?
Because the homogeneous solution has a $C_12^n$
and your guess contains $A 2^n$ with overlaps with $C_12^n$.
Your guess also cannot be $a_n^{(p)} = (A + Bn) \cdot n2^n$. Why?
Because the homogeneous solution has a $C_2 n2^n$
and your guess contains $Bn 2^n$ with overlaps with $C_2 n2^n$.
You have to guess $a_n^{(p)} = (A + Bn) \cdot n^2 2^n$.

\item So in general if the nonhomogeneous part looks like
$\text{(polynomial of degree d)} \cdot r^n$,
you write down 
$\text{(polynomial of degree d)} \cdot r^n$,
where the polynomial has constants $A,B,C,...$ for the coefficients
for the polynomial.
You check check if you need to change $r^n$ to
$nr^n$, $n^2r^n$, etc. until your guess does not have any term
that overlaps with the any term of the homogeneous solution.

\item See theorem 6 in Rosen.
\end{itemize}

The problem is that you have memorize the above cases or memorize
Theorem 6 of Rosen.
And theorem 6 goes not even cover all cases.
On the other hand, generating functions always work, as long as
when $a_n$'s are converted into a power series, the corresponding
power series for the 
the nonhomogenous part becomes a power series that is one of the
standard forms or can be manipulated into one of the standard forms.
That's why I prefer to simply work with power series.


%------------------------------------------------------------------------------
\newpage
\nextq Find a closed form for $0^3 + 1^3 + 2^3 + \cdots + n^3$.

\SOLUTION

\input{sum-of-cubes.tex}

%------------------------------------------------------------------------------
\newpage
\nextq
Using generating function method, find a closed form for $a_n$ where
\[
a_n = 5a_{n-1} - 6a_{n-2} + \underline{3n + 1}
\]
where $a_0 = 2$ and $a_1 = 3$.

\SOLUTION

\input{gen-func-nonhomogeneous.tex}

%------------------------------------------------------------------------------
\newpage
\nextq Rosen 8th edition, section 8.2, question 1.

\SOLUTION


(a)
$a_n = 3a_{n - 1} + 4a_{n - 2} + 5a_{n - 3}$: Linear. Homogeneous. Degree 3.

(b)
$a_n = 2 \underline{n} a_{n - 1} + a_{n - 2}$: Not linear. Homogeneous. Degree 2.

(c) $a_n = a_{n - 1} + a_{n - 4}$: Linear. Homogeneous. Degree 4.

(d) $a_n = a_{n-1} + \underline{2}$: Linear. Not homogeneous. Degree 1.

(e) $a_n = \underline{a^2_{n-1}} + a_{n-2}$: Not linear. Homogeneous. Degree 2.

(f) $a_n = a_{n-2}$: Linear. Homogeneous. Degree 2.

(g)
$a_n = a_{n-2}$: Linear. Homogeneous. Degree 2.




%------------------------------------------------------------------------------
\newpage
\nextq Rosen 8th edition, section 8.2, question 3.

\SOLUTION


(a)
The characteristic equation of
\[
a_n = 
\begin{cases}
  2 a_{n-1}  &\text{for $n \geq 1$} \\
  3         &\text{for $n \geq 0$}
\end{cases}
\]
is
\[
x - 2
\]
The root is $r_1 = 2$.
Therefore the general closed form for $a_n$ is
\[
a_n = C_1 2^n
\]
From the base condition
\[
3 = C_1 2^0 = C_1
\]
Hence
\[
a_n = 3 \cdot 2^n = 3^{n + 1}
\]
for $n \geq 0$.
\qed

(Check: $a_0 = 3, a_1 = 3 \cdot 2^1 = 6 = 2 \cdot 3 = 2 \cdot a_0, a_2 = 3 \cdot 2^2 = 12 = 2 \cdot 6 = 2 \cdot a_1$, $a_3 = 3 \cdot 2^3 = 24 = 2 \cdot 12 = 2 \cdot a_2$.)


(b)
The characteristic equation of
\[
a_n = 
\begin{cases}
  a_{n-1}  &\text{if } n \geq 1 \\
  2       &\text{if } n = 0
\end{cases}
\]
is
\[
x - 1
\]
The root is $r_1 = 1$.
Therefore the general closed form for $a_n$ is
\[
a_n = C_1 1^n = C_1
\]
From the base condition
\[
2 = a_0 = C_1
\]
Hence
\[
a_n = 2 
\]
for $n \geq 0$.
\qed

(Check: $a_0 = 2, a_1 = 2 = a_0, a_2 = 2 = a_1$, $a_3 = 2 = a_2$.)


(c)
The characteristic equation of
\[
a_n = 
\begin{cases}
  5 a_{n-1} - 6 a_{n-2} &\text{if } n \geq 2 \\
  1                   &\text{if } n = 0  \\
  0                   &\text{if } n = 1 
\end{cases}
\]
is
\[
x^2 - 5x + 6
\]
The roots are $r_1 = 2$, $r_2 = 3$.
Therefore the general closed form for $a_n$ is
\[
a_n = C_1 2^n + C_2 3^n
\]
From the base cases
\begin{align*}
1 &= a_0 = C_1 + C_2 \\
0 &= a_1 = 2C_1 + 3C_2 \\
\end{align*}
Therefore $C_1 = 3, C_2 = -2$
Hence
\[
a_n = 3 \cdot 2^n - 2 \cdot 3^n
\]
for $n \geq 0$.
\qed

(Check:
\begin{console}[fontsize=\footnotesize]
def an(n):
    if n == 0: return 1
    elif n == 1: return 0
    else: return 5 * an(n - 1) - 6 * an(n - 2)

def bn(n):
    return 3 * 2**n - 2 * 3**n

for n in range(10):
    print(n, an(n), bn(n))
\end{console}
has output
\begin{console}[fontsize=\footnotesize]
0 1 1
1 0 0
2 -6 -6
3 -30 -30
4 -114 -114
5 -390 -390
6 -1266 -1266
7 -3990 -3990
8 -12354 -12354
9 -37830 -37830
\end{console}
)


(d)
The characteristic equation of
\[
a_n = 
\begin{cases}
  4 a_{n-1} - 4 a_{n-2} &\text{if } n \geq 2 \\
  6                   &\text{if } n \geq 0 \\
  8                   &\text{if } n \geq 1
\end{cases}
\]
is
\[
x^2 - 4x + 4
\]
The roots are $r_1 = 2$, $r_2 = 2$.
Therefore the general closed form for $a_n$ is
\[
a_n = C_1 2^n + C_2 n2^n
\]
From the base cases
\begin{align*}
6 &= a_0 = C_1 \\
8 &= a_1 = 2C_1 + 2C_2 \\
\end{align*}
Therefore $C_1 = 6, C_2 = -2$
Hence
\[
a_n = 6 \cdot 2^n - 2 \cdot n 2^n = (6 - 2n) 2^n
\]
for $n \geq 0$.
\qed

Check:
\begin{console}[fontsize=\footnotesize]
def an(n):
    if n == 0: return 6
    elif n == 1: return 8
    else: return 4 * an(n-1) - 4 * an(n - 2)

def bn(n):
    return 6 * 2**n - 2 * n * 2**n

for n in range(10):
    print(n, an(n), bn(n))
\end{console}
\begin{console}[fontsize=\footnotesize]
0 6 6
1 8 8
2 8 8
3 0 0
4 -32 -32
5 -128 -128
6 -384 -384
7 -1024 -1024
8 -2560 -2560
9 -6144 -6144
\end{console}


(e)
The characteristic equation of
\[
a_n = 
\begin{cases}
 - 4 a_{n-1} - 4 a_{n-2} &\text{if } n \geq 2 \\
  0                    &\text{if } n = 0 \\
  1                    &\text{if } n = 1
\end{cases}
\]
is
\[
x^2 + 4x + 4
\]
The roots are $r_1 = -2$, $r_2 = -2$.
Therefore the general closed form for $a_n$ is
\[
a_n = C_1 (-2)^n + C_2 n(-2)^n
= C_1 (-1)^n 2^n + C_2 n(-1)^n 2^n
= (C_1 + C_2 n) (-1)^n 2^n
\]
From the base cases
\begin{align*}
0 &= a_0 = C_1 \\
1 &= a_1 = -2C_1 - 2C_2 
\end{align*}
Therefore $C_1 = 0, C_2 = -1/2$
Hence
\[
a_n = (0 + (-1/2)n) (-1)^n 2^n = (-1)^{n+1} n 2^{n - 1}
\]
for $n \geq 0$.
\qed

Check:
\begin{console}[fontsize=\footnotesize]
def an(n):
    if n == 0: return 0
    elif n == 1: return 1
    else: return -4 * an(n - 1) - 4 * an(n - 2)

def bn(n):
    return (-1)**(n + 1) * n * 2**(n - 1)

for n in range(10):
    print(n, an(n), bn(n))
\end{console}
\begin{console}[fontsize=\footnotesize]
0 0 0.0
1 1 1
2 -4 -4
3 12 12
4 -32 -32
5 80 80
6 -192 -192
7 448 448
8 -1024 -1024
9 2304 2304
\end{console}


(f)
The characteristic equation of
\[
a_n = 
\begin{cases}
  4 a_{n-2} &\text{if } n \geq 2 \\
  0        &\text{if } n = 0 \\
  4        &\text{if } n = 1
\end{cases}
\]
is
\[
x^2 - 4
\]
The roots are $r_1 = -2$, $r_2 = 2$.
Therefore the general closed form for $a_n$ is
\[
a_n = C_1 (-2)^n + C_2 2^n
= C_1 (-1)^n 2^n + C_2 2^n
= ((-1)^n C_1 + C_2) 2^n
\]
From the base cases
\begin{align*}
0 &= a_0 = C_1 + C_2 \\
4 &= a_1 = -2C_1 + 2C_2 
\end{align*}
Therefore $C_1 = -1, C_2 = 1$
Hence
\[
a_n = ((-1)^n (-1) + 1) 2^n = ((-1)^{n+1} + 1) 2^n
\]
for $n \geq 0$.
\qed

Check:
\begin{console}[fontsize=\footnotesize]
def an(n):
    if n == 0: return 0
    elif n == 1: return 4
    else: return  4 * an(n - 2)

def bn(n):
    return ((-1)**(n+1) + 1) * 2**n

for n in range(10):
    print(n, an(n), bn(n))
\end{console}
\begin{console}[fontsize=\footnotesize]
0 0 0
1 4 4
2 0 0
3 16 16
4 0 0
5 64 64
6 0 0
7 256 256
8 0 0
9 1024 1024
\end{console}


(g)
The characteristic equation of
\[
a_n = 
\begin{cases}
  a_{n-2}/4 &\text{if } n \geq 2 \\
  1        &\text{if } n = 0 \\
  0        &\text{if } n = 1
\end{cases}
\]
is
\[
x^2 - 1/4
\]
The roots are $r_1 = -1/2$, $r_2 = 1/2$.
Therefore the general closed form for $a_n$ is
\[
a_n = C_1 (-1/2)^n + C_2 (1/2)^n
= C_1 (-1)^n (1/2)^n + C_2 (1/2)^n
= ((-1)^n C_1 + C_2) (1/2)^n
\]
From the base cases
\begin{align*}
1 &= a_0 = C_1 + C_2 \\
0 &= a_1 = -C_1 + C_2 
\end{align*}
Therefore $C_1 = 1/2, C_2 = 1/2$
Hence
\[
a_n = ((-1)^n (1/2) + (1/2)) (1/2)^n = ((-1)^{n} + 1) (1/2)^{n + 1}
\]
for $n \geq 0$.
\qed

Check:
\begin{console}[fontsize=\footnotesize]
def an(n):
    if n == 0: return 1
    elif n == 1: return 0
    else: return  an(n - 2) / 4.0

def bn(n):
    return ((-1)**n + 1) * 0.5**(n + 1)

for n in range(10):
    print(n, an(n), bn(n))
\end{console}
\begin{console}[fontsize=\footnotesize]
0 1 1.0
1 0 0.0
2 0.25 0.25
3 0.0 0.0
4 0.0625 0.0625
5 0.0 0.0
6 0.015625 0.015625
7 0.0 0.0
8 0.00390625 0.00390625
9 0.0 0.0
\end{console}




%------------------------------------------------------------------------------
\newpage
\nextq Rosen 8th edition, section 8.2, question 4.

\SOLUTION
Solve these recurrence relations together with the initial
conditions given.

a)

$r^2 -r -6 = 0$

$r=-2,3$

$a_n = a_1(-2)^n + a_23^n$

$3 = a_1 + a_2$

$6= -2a_1 + 3a_2$

$a_1 = \frac{3}{5}$ $a_2 = \frac{12}{5}$

$a_n = (\frac{3}{5})(-2)^n + (\frac{12}{5})3^n$

b)

$r^2 - 6r +10 = 0$

$r=-2,5$

$a_n = a_12^n + a_25^n$

$2 = a_1 + a_2$

$1= -2a_1 + 5a_2$

$a_1 = 3$ $a_2 = -1$

$a_n = 3 \cdot 2^n - 5^n$

c) 

$r^2 - 6r +8 = 0$

$r=-2,4$

$a_n = a_12^n + a_24^n$

$4 = a_1 + a_2$

$10= -2a_1 + 4a_2$

$a_1 = 3$ $a_2 = 1$

$a_n = 3 \cdot 2^n - 4^n$

d) 

$r^2 - 2r +1 = 0$

$r=-1,1$

$a_n = a_11^n + a_21^n$

$4 = a_1 $

$1= a_1 + a_2$

$a_1 = 4$ $a_2 = -3$

$a_n = 4  - 3^n$

e)

$r^2 - 1 = 0$

$r=-1,1$

$a_n = a_1(-1)^n + a_21^n$

$5 = a_1 + a_2$

$-1= -a_1 + a_2$

$a_1 = 3$ $a_2 = 2$

$a_n = 3 \cdot (-1)^n +2$

(f)

$r^2 + 6r + 9 = 0$

$r=-3,-3$

$a_n = a_1(-3)^n + a_2(-3)^n$

$3 = a_1$

$-3= -3a_1 +-3a_2$

$a_1 = 3$ $a_2 = -2$

$a_n = 3(-3)^n - 2n(-3)^n$

(g)

$r^2 +4r -5 = 0$

$r=-5,1$

$a_n = a_1(-5)^n + a_21^n$

$2 = a_1 + a_2$

$8= -5a_1 + a_2$

$a_1 = -1$ $a_2 = 3$

$a_n = -(-5)^n + 3$







\input{08-02-04.tex}

%------------------------------------------------------------------------------
\newpage
\nextq Rosen 8th edition, section 8.2, question 12.

\SOLUTION

$r^3 -2r^2 - r + 2 =0$

$r = 1,-1,2$

$a_n = a_1 + a_2(-1)^n + a_32^n $

$3 = a_1 + a_2 + a_3$

$ 6 = a_1 - a_2 + 2a_3$

$ a-1 = 6, a_2 = -2, a_3 = -1$

$a_n = 6 - 2(-1)^n - 2^n$



\input{08-02-12.tex}

%------------------------------------------------------------------------------
\newpage
\nextq Rosen 8th edition, section 8.2, question 13.

\SOLUTION


The characteristic equation of
\[
a_n = 
\begin{cases}
  7a_{n-2} + 6a_{n-3} &\text{if } n \geq 3 \\
  9                 &\text{if } n = 0 \\
  10                &\text{if } n = 1 \\
  32                &\text{if } n = 2 \\
\end{cases}
\]
is
\[
x^3 - 7x - 6
\]
By trial and error, one of the roots is $-1$.
By long division, we get $x^3 - 7x - 6 = (x + 1)(x^2 - x - 6)$.
The roots are $r_1 = -2$, $r_2 = -1$, $r_3 = 3$.
Therefore the general closed form for $a_n$ is
\[
a_n = C_1 (-2)^n + C_2 (-1)^n + C_3 3^n
= (-1)^n(C_12^n + C_2) + C_3 3^n
\]
From the base cases
\begin{align*}
 9 &= a_0 = C_1 + C_2 + C_3 \tag{1}\\
10 &= a_1 = -2C_1 - C_2 + 3C_3 \tag{2} \\
32 &= a_2 = 4C_1 + C_2 + 9C_3 \tag{3}
\end{align*}
(1)+(2) and (2)+(3) give us
\begin{align*}
19 &= -C_1 + 4C_3 \tag{4} \\
21 &= C_1 + 6C_3 \tag{5}
\end{align*}
(4) + (5) gives us $40 = 10C_3$, i.e., $C_3 = 4$.
(4) then gives us $C_1 = -3$.
From (1), we get $C_2 = 8$.
Hence
\[
a_n = (-1)^n((-3)2^n + 8) + 4 \cdot 3^n
\]
for $n \geq 0$.
\qed

Check:
\begin{console}[fontsize=\footnotesize]
def an(n):
    if n == 0: return 9
    elif n == 1: return 10
    elif n == 2: return 32
    else: return 7 * an(n - 2) + 6 * an(n - 3) 

def bn(n):
    return (-1)**n*((-3)*2**n + 8) + 4 * 3**n

for n in range(10):
    print(n, an(n), bn(n))
\end{console}
\begin{console}[fontsize=\footnotesize]
0 9 9
1 10 10
2 32 32
3 124 124
4 284 284
5 1060 1060
6 2732 2732
7 9124 9124
8 25484 25484
9 80260 80260
\end{console}


%------------------------------------------------------------------------------
\newpage
\nextq Rosen 8th edition, section 8.2, question 23.

\SOLUTION


(a)
If $a_n = -2^{n+1}$, then
\begin{align*}
3a_{n-1} + 2^n
&= 3(-2^{(n-1)+1}) + 2^n \\
&= 3(-2^n) + 2^n \\
&= (3(-1) + 1) \cdot 2^n \\
&= -2 \cdot 2^n \\
&= -2^{n + 1} \\
&= a_n
\end{align*}
Hence
\[
a_n^{(p)} = -2^{n+1}
\]
is a (particular) solution of $a_n = 3a_{n - 1} + 2^n$.


(b)
The characteristic equation of
$a_n = 3a_{n - 1}$
is
\[
x - 3
\]
The root of the charactistic equation is $r_1 = 3$.
Therefore the general solution to
$a_n = 3a_{n - 1}$ is $C_1 3^n$, i.e.,
\[
a_n^{(h)} = C_1 3^n
\]
where $C_1$ is a constant.
Hence the general solution to
$a_n = 3a_{n - 1} + 2^n$ is
\[
a_n = a_n^{(h)} + a_n^{(p)} = C_1 3^n -2^{n+1}
\]
where $C_1$ is a constant.

(c)
Using the base case,
\[
1 = a_0 = C_1 3^0 -2^{0+1} = C_1 - 2
\]
Hence $C_1 = 3$.
Therefore
\[
a_n = 3 \cdot 3^n -2^{n+1}
\]
\qed

Check:
\begin{Verbatim}[frame=single,fontsize=\small]
def an(n):
    if n == 0:
        return 1
    else:
        return 3 * an(n - 1) + 2**n

def bn(n):
    return 3 * 3**n - 2**(n + 1)

for n in range(10):
    print(n, an(n), bn(n))     
\end{Verbatim}
has output
\begin{Verbatim}[frame=single,fontsize=\small]
0 1 1
1 5 5
2 19 19
3 65 65
4 211 211
5 665 665
6 2059 2059
7 6305 6305
8 19171 19171
9 58025 58025
\end{Verbatim}


%------------------------------------------------------------------------------
\newpage
\nextq Rosen 8th edition, section 8.2, question 24.

\SOLUTION


(a)
If $a_n = n 2^n$, then
\begin{align*}
  2 a_{n - 1} + 2^n
  &= 2 (n - 1) 2^{n - 1} + 2^n \\
  &= (n - 1) 2^{n} + 2^n \\
  &= n 2^{n} \\
  &= a_n
\end{align*}
Hence
\[
a_n^{(p)} = n 2^n
\]
is a (particular) solution of $a_n = 2 a_{n - 1} + 2^n$.

(b)
The characteristic equation of
$a_n = 2 a_{n - 1}$
is
\[
x - 2
\]
The root of the charactistic equation is $r_1 = 2$.
Therefore the general solution to
$a_n = 3a_{n - 1}$ is $C_1 2^n$, i.e.,
\[
a_n^{(h)} = C_1 2^n
\]
where $C_1$ is a constant.
Hence the general solution to
$a_n = 3a_{n - 1} + 2^n$ is
\[
a_n = a_n^{(h)} + a_n^{(p)} = C_1 2^n + n2^n
\]
where $C_1$ is a constant.

(c)
Using the base case,
\[
2 = a_0 = C_1 2^0 + 0 = C_1
\]
Hence $C_1 = 2$.
Therefore
\[
a_n = 2 \cdot 2^n + n2^n = (2 + n) 2^n 
\]
\qed

Check:
\begin{Verbatim}[frame=single,fontsize=\small]
def an(n):
    if n == 0:
        return 2
    else:
        return 2 * an(n - 1) + 2**n

def bn(n):
    return (2 + n) * 2**n

for n in range(10):
    print(n, an(n), bn(n))     
\end{Verbatim}
has output
\begin{Verbatim}[frame=single,fontsize=\small]
0 2 2
1 6 6
2 16 16
3 40 40
4 96 96
5 224 224
6 512 512
7 1152 1152
8 2560 2560
9 5632 5632
\end{Verbatim}


%------------------------------------------------------------------------------
\newpage
\nextq Rosen 8th edition, section 8.2, question 25.

\SOLUTION


(a)
If $a_n = An + B$ is a particular solution of $a_n = 2 a_{n - 1} + n + 5$, then
\begin{align*}
  An + B
  &= 2 (A(n - 1) + B) + n + 5 \\ 
  &= (2A + 1) n + (-2A + 2B + 5) \\ 
\end{align*}
Hence
\begin{align*}
  A &= 2A + 1 \\
  B &= -2A + 2B + 5 
\end{align*}
i.e.,
\begin{align*}
  A &= - 1 \\
  B &= 2A - 5 
\end{align*}
Hence $A = -1$ and $B = -7$.
Therefore
\[
a^{(p)} = -n -7
\]
is a particular solution of $a_n = 2 a_{n - 1} + n + 5$.


(b)
The characteristic equation of $a_n = 2 a_{n - 1}$ is
\[
x - 2
\]
The root of the chracteristic equation is $r_1 = 2$.
Therefore the general solution of $a_n = 2a_{n - 1}$ is
\[
a^{(h)} = C_1 2^n
\]
Hence the general solution of $a_n = 2 a_{n - 1} + n + 5$ is
\[
a_n = a^{(h)} +  a^{(p)} = C_1 2^n - n - 7
\]


(c)
From $a_0 = 4$,
\[
4 = a_0 = C_1 2^0 - 0 - 7 = C_1 - 7
\]
i.e., $C_1 = 11$.
Hence the solution is
\[
a_n = 11 \cdot 2^n - n - 7
\]

Check:
\begin{Verbatim}[frame=single,fontsize=\small]
def an(n):
    if n == 0:
        return 4
    else:
        return 2 * an(n - 1) + n + 5

def bn(n):
    return 11 * 2**n - n - 7

for n in range(10):
    print(n, an(n), bn(n))     
\end{Verbatim}
has output
\begin{Verbatim}[frame=single,fontsize=\small]
0 4 4
1 14 14
2 35 35
3 78 78
4 165 165
5 340 340
6 691 691
7 1394 1394
8 2801 2801
9 5616 5616
\end{Verbatim}


%------------------------------------------------------------------------------
\newpage
\nextq Rosen 8th edition, section 8.2, question 25.

\SOLUTION


(a)
If $a_n = An + B$ is a particular solution of $a_n = 2 a_{n - 1} + n + 5$, then
\begin{align*}
  An + B
  &= 2 (A(n - 1) + B) + n + 5 \\ 
  &= (2A + 1) n + (-2A + 2B + 5) \\ 
\end{align*}
Hence
\begin{align*}
  A &= 2A + 1 \\
  B &= -2A + 2B + 5 
\end{align*}
i.e.,
\begin{align*}
  A &= - 1 \\
  B &= 2A - 5 
\end{align*}
Hence $A = -1$ and $B = -7$.
Therefore
\[
a^{(p)} = -n -7
\]
is a particular solution of $a_n = 2 a_{n - 1} + n + 5$.


(b)
The characteristic equation of $a_n = 2 a_{n - 1}$ is
\[
x - 2
\]
The root of the chracteristic equation is $r_1 = 2$.
Therefore the general solution of $a_n = 2a_{n - 1}$ is
\[
a^{(h)} = C_1 2^n
\]
Hence the general solution of $a_n = 2 a_{n - 1} + n + 5$ is
\[
a_n = a^{(h)} +  a^{(p)} = C_1 2^n - n - 7
\]


(c)
From $a_0 = 4$,
\[
4 = a_0 = C_1 2^0 - 0 - 7 = C_1 - 7
\]
i.e., $C_1 = 11$.
Hence the solution is
\[
a_n = 11 \cdot 2^n - n - 7
\]

Check:
\begin{Verbatim}[frame=single,fontsize=\small]
def an(n):
    if n == 0:
        return 4
    else:
        return 2 * an(n - 1) + n + 5

def bn(n):
    return 11 * 2**n - n - 7

for n in range(10):
    print(n, an(n), bn(n))     
\end{Verbatim}
has output
\begin{Verbatim}[frame=single,fontsize=\small]
0 4 4
1 14 14
2 35 35
3 78 78
4 165 165
5 340 340
6 691 691
7 1394 1394
8 2801 2801
9 5616 5616
\end{Verbatim}


%------------------------------------------------------------------------------
\newpage
\nextq Rosen 8th edition, section 8.2, question 26.

\SOLUTION


The characteristic equation of 
$a_n = 6 a_{n - 1} - 12 a_{n - 1} + 8 a_{n - 3} + F(n)$
is
\[
x^3 - 6 x^2 + 12 x - 8
\]
By trial and error, $2$ is a root.
By long division
\[
x^3 - 6 x^2 + 12 x - 8 = (x - 2)(x^2 - 4x + 4)
\]
Hence
\[
x^3 - 6 x^2 + 12 x - 8 = (x - 2)^3
\]
Hence the roots of $x^3 - 6 x^2 + 12 x - 8$ are $r_1 = r_2 = r_3 = 2$.
Therefore the general solution to homogeneous part of the
recurrence relation is
\[
a^{(h)}_n = C_1 2^n + C_2 2^n + C_1 n^2 2^n 
\]

(a)
The general form of
the closed form for $a_n = 6 a_{n - 1} - 12 a_{n - 1} + 8 a_{n - 3} + n^2$
is
\[
a_n = C_1 2^n + C_2 2^n + C_1 n^2 2^n + A + Bn + Cn^2
\]


(b)
The general form of
the closed form for $a_n = 6 a_{n - 1} - 12 a_{n - 1} + 8 a_{n - 3} + 2^n$
is
\[
a_n = C_1 2^n + C_2 n2^n + C_1 n^2 2^n + An^3 2^n
\]


(c)
The general form of
the closed form for $a_n = 6 a_{n - 1} - 12 a_{n - 1} + 8 a_{n - 3} + n2^n$
is
\[
a_n = C_1 2^n + C_2 n2^n + C_3 n^2 2^n + (A + Bn)n^3 2^n
\]
where $C_1, C_2, C_3, A, B$ are constants.


(d)
The general form of
the closed form for $a_n = 6 a_{n - 1} - 12 a_{n - 1} + 8 a_{n - 3} + (-2)^n$
is
\[
a_n = C_1 2^n + C_2 n2^n + C_3 n^2 2^n + A(-2)^n
\]
where $C_1, C_2, C_3, A$ are constants.


(e)
The general form of
the closed form for $a_n = 6 a_{n - 1} - 12 a_{n - 1} + 8 a_{n - 3} + n^2 2^n$
is
\[
a_n = C_1 2^n + C_2 n2^n + C_3 n^2 2^n + (A + Bn + Cn^2) n^3 2^n
\]
where $C_1, C_2, C_3, A, B, C$ are constants.


(f)
The general form of
the closed form for $a_n = 6 a_{n - 1} - 12 a_{n - 1} + 8 a_{n - 3} + n^3 (-2)^n$
is
\[
a_n = C_1 2^n + C_2 n2^n + C_3 n^2 2^n + (A + Bn + Cn^2 + Dn^3)(-2)^n
\]
where $C_1, C_2, C_3, A, B, C, D$ are constants.


(g)
The general form of
the closed form for $a_n = 6 a_{n - 1} - 12 a_{n - 1} + 8 a_{n - 3} + 3$
is
\[
a_n = C_1 2^n + C_2 n2^n + C_3 n^2 2^n + A
\]
where $C_1, C_2, C_3, A$ are constants.


%------------------------------------------------------------------------------
\newpage
\nextq Rosen 8th edition, section 8.2, question 28.

\SOLUTION

a)

$a_n = 2a_{n-1}$

$a_n^{(h)} = a2^n$

$p_2 = -2, 4p_2 = p_1$

$-2p_2 + 2p_1 - p_0 = 0$

$p = -8$ $p_0 = -12$

$a_n = a2^n -2n^2 -8n - 12$

b) 

$a=13$ 

$a_n = 13 \cdot 2^n - 8n - 12$
\input{08-02-28.tex}

%------------------------------------------------------------------------------
\newpage
\nextq Rosen 8th edition, section 8.2, question 30.

\SOLUTION

a)

$a_n = -5a_{n-1} - 6a_{n-2}$

$r^2 + 5r +6=0$ 

$r = -2,-3$

$a_n^{(h)} = a(-2)^n + \beta(-3)^n$

$c=16$

$a_n^{(p)} = 16 \cdot 4^n$

$a_n = a(-2)^n + \beta(-3)^n +4^{n+2}$

b)

$a_1 = 56$

$a_2 = 4a + 9 \beta + 256$

$a_n = (-2)^n + 2(-3) + 4^{n+2}$
\input{08-02-30.tex}

%------------------------------------------------------------------------------
\newpage
\input{instructions.tex}
\end{document}
