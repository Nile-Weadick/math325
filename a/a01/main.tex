\newcommand\COURSE{math325}
\newcommand\ASSESSMENT{q080403}
\newcommand\ASSESSMENTTYPE{Quiz}
\newcommand\POINTS{\textwhite{xxx/xxx}}

\makeatletter
\DeclareOldFontCommand{\rm}{\normalfont\rmfamily}{\mathrm}
\DeclareOldFontCommand{\sf}{\normalfont\sffamily}{\mathsf}
\DeclareOldFontCommand{\tt}{\normalfont\ttfamily}{\mathtt}
\DeclareOldFontCommand{\bf}{\normalfont\bfseries}{\mathbf}
\DeclareOldFontCommand{\it}{\normalfont\itshape}{\mathit}
\DeclareOldFontCommand{\sl}{\normalfont\slshape}{\@nomath\sl}
\DeclareOldFontCommand{\sc}{\normalfont\scshape}{\@nomath\sc}
\makeatother

\input{myquizpreamble}
\input{yliow}
\input{\COURSE}
\textwidth=6in

\renewcommand\TITLE{\ASSESSMENTTYPE \ \ASSESSMENT}

\newcommand\topmattertwo{
\topmatter
\score \\ \\
Open \texttt{main.tex} and enter answers (look for
\texttt{answercode}, \texttt{answerbox}, \texttt{answerlong}).
Turn the page for detailed instructions.
To rebuild and view pdf, in bash shell execute \texttt{make}.
To build a gzip-tar file, in bash shell execute \texttt{make s} and
you'll get \texttt{submit.tar.gz}.
}

\newcommand\tf{T or F or M}
\newcommand\answerbox[1]{\textbox{#1}}
\newcommand\codebox[1]{\begin{console}#1\end{console}}

\usepackage{pifont}
\newcommand{\cmark}{\textred{\ding{51}}}
\newcommand{\xmark}{\textred{\ding{55}}}

\newcounter{qc}
\newcommand\nextq{
%\newpage
\addtocounter{qc}{1}
Q{\theqc}.
}

\DefineVerbatimEnvironment%
 {answercode}{Verbatim}
 {frame=single,fontsize=\footnotesize}

\newenvironment{largebox}[1]{%
 \boxparone{#1}
}
{}

\usepackage{environ}
\let\oldquote=\quote
\let\endoldquote=\endquote
\let\quote\relax
\let\endquote\relax

% ADDED 2021/09/09
\renewcommand\boxpar[1]{
 \[
  \framebox[\textwidth][c] {
   \parbox[]{\dimexpr\textwidth - 0.25cm} {#1}
  }
 \]
}

\NewEnviron{answerlong}%
  {\vspace{-1mm} \global\let\tmp\BODY\aftergroup\doboxpar}

\newcommand\doboxpar{%
  \let\quote=\oldquote
  \let\endquote=\endoldquote
  \boxpar{\tmp}
}

\newenvironment{mcq}[7]%
{% begin code
#1 \dotfill{#2}
 \begin{tightlist}
 \item[(A)] #3
 \item[(B)] #4
 \item[(C)] #5
 \item[(D)] #6
 \item[(E)] #7 
 \end{tightlist}
}%
{% end code
} 

\renewcommand\EMAIL{}
\newcommand\score{%
\vspace{-0.6in}
\begin{flushright}
Score: \answerbox{\POINTS}
\end{flushright}
\vspace{-0.4in}
\hspace{0.7in}\AUTHOR
\vspace{0.2in}
}

\newcommand\blankline{\mbox{}\\ }

\newcommand\ANSWER{\textsc{Answer:}\vspace{-2mm}}


\renewcommand\AUTHOR{nweadick1@cougars.ccis.edu} % CHANGE TO YOURS

\begin{document}
\topmattertwo

In \LaTeX\, math notation is enclosed by the \verb!$! symbols.
For instance \verb!$x = a_{1} + b^{2}$! gives you $x = a_{1} + b^{2}$.
For emphasis you can write also write it as
\verb!\[ x = a_{1} + b^{2} \]!
to center your math:
\[
x = a_{1} + b^{2}
\]
For more information about \LaTeX\, go to my
website
\href{http://bit.ly/yliow0}{http://bit.ly/yliow0},
click on \verb!Yes!
you are one of my students,
then look for
\href{https://drive.google.com/file/d/0BzjYrK0VFuMWZm5xV0kyR3J2Zm8/view?usp=sharing}{latex.pdf}.)

%------------------------------------------------------------------------------
\newpage
\textsc{Example.}
Prove that
\begin{enumerate}[nosep]
\item[(a)] Reflexive: $|X| = |X|$
\item[(b)] Symmetric: If $|X| = |Y|$, then $|Y| = |X|$.
\item[(c)] Transitive: If $|X| \leq |Y|$ and $|Y| \leq |Z|$, then
$|X| \leq |Z|$.
\end{enumerate}

Recall the following definitions:
$|X| \leq |Y|$ means there is a 1--1 function $X \rightarrow Y$
and $|X| = |Y|$ means there there is a 1--1 and onto function $X \rightarrow Y$.

\SOLUTION

(a)

Let $X$ be a set and a function, s.t $f:X \rightarrow X$
$f(x)=x$, where $x$ is an element of $X$
$f$ is a 1-1 and onto function. 
since $x,x'$ are elements of $X$
$f(x) = f(x')$ implies $x=x'$

Therefore, $|X| = |X|$

(b) 

Let $f:X \rightarrow Y$ that is 1-1 and onto. Since $f$ is bijective it has an inverse function.
$f^-1: Y \rightarrow X$ , which is 1-1 and onto. 

Therefore, if $|X| = |Y|$ then, $|Y| = |X|$

(c)

Let $f:X \rightarrow Y$ that is 1-1 and $g:Y \rightarrow Z$ which is 1-1. and $h:X \rightarrow Z$.
then the composite function $gof$ = $h:X \rightarrow Z$

Therefore, If $|X| \leq |Y|$ and $|Y| \leq |Z|$, then $|X| \leq |Z|$.





(to be completed)


%------------------------------------------------------------------------------
\newpage
\nextq Rosen 8th edition, section 2.5, question 18.

Prove that for sets $A$ and $B$, if $|A| = |B|$, then $|P(A)| = |P(B)|$.

\textsc{Warning:} $A$ and $B$ need not be finite.

\SOLUTION

Let $f: A \rightarrow B$ which is 1-1. Since $|A| = |B|$ have the same cardinality, they will have the same number of subsets.

Therefore, if $|A| = |B|$, then $|P(A)| = |P(B)|$

(to be completed)


%------------------------------------------------------------------------------
\newpage
\nextq Rosen 8th edition, section 2.5, question 27.

Prove that the union of countably many countable sets is countable.
      
Let me explain what the question is asking.
Note that you already know what
is a \textit{countable set} -- it's a set $X$ such that $X$ is finite, say
$|X| = |\{1, 2, 3, ..., n\}|$, or $|X| = |\N|$.
And a union \textit{countably many sets} means you are trying to
compute the union of of a collection of sets and this \textit{collection} is
countable.
This means that you can label the sets of the collection in a countable way.
So either
(a) there is a finite number of sets (say $n$ of them) and the union is
\[
A_1 \cup A_2 \cup \cdots A_n = \bigcup_{i \in \{1, 2, 3, \cdots n\}} A_i
\]
i.e., the collection of sets $\{A_1, A_2 ..., A_n\}$ is finite
or
(b) there is infinitely countable number of sets and the union is
\[
A_1 \cup A_2 \cup \cdots = \bigcup_{i \in \N} A_i
\]
i.e., the collection of sets $\{A_1, A_2 ..., \}$ is infinitely countable.
An example of a uncountable union is when you can label the sets with $\R$.
For instance for each $x \in \R$, let $A_x = [x-1, x+1)$ and then you take
\[
\bigcup_{x \in \R} A_x
\]
Something like $\bigcup_{x \in [0,1)} A_x$ is also an uncountable union.
Therefore you need to prove this:
(a) If $\{A_1, A_2, ..., A_n\}$ is countable and each $A_i$ is countable, then
$\bigcup_{i \in \{1, 2, 3, \cdots n\}} A_i$ is also countable and
(b) If $\{A_1, A_2, ...\}$ is countable and each $A_i$ is countable, then
$\bigcup_{i \in \N} A_i$ is also countable.

Remember: \lq\lq countable" means (a) finite or (b) countably infinite.

\textsc{Note:} The solution is provided in the textbook.
Study the solution carefully and write it in your own words.
The solutions at the back of the textbook is usually very brief and
sometimes require clarifications and filling in some details.
Even better: See if you can prove it in a different, simpler, and clearer way.
Yes, there's a simpler proof that uses a proof that I talked about in class
for another theorem.
That's why studying and understanding proofs are important. 

\SOLUTION

Let the set of size $n$ s.t $X = \{x1,x2,...,xn\}$ and another set of size $n$ s.t $Y = \{y1,y2,...,yn\}$

You can construct such as set $ X \cup Y = \{x1,y1,x2,y2,...,xn,yn\}$ which is countable.

(to be completed)


%------------------------------------------------------------------------------
\newpage
\nextq Rosen 8th edition, section 2.5, question 37.

Prove that there are countably many programs that can be written in a
particular programming language.

(The proof is exactly the same if you change the questions to \lq\lq prove
that there are countable many novels that can be written in a particular
alphabetic language".)

\textsc{Note:} The solution is provided in the textbook.
Study the solution carefully and write it in your own words.
See if you can find a simpler adn clearer proof.

\SOLUTION

(to be completed)


%------------------------------------------------------------------------------
\newpage
\nextq Rosen 8th edition, section 2.5, question 38 (modified).

Prove that there are uncountably many functions from $\N$ to $\{ 0, 1 \}$.

Note: The previous questions says that there are \textit{countably}
many C++ programs.
This questions says that there are \textit{uncountably} many boolean functions
$\N \rightarrow \{0,1\}$ is uncountable.
I'll let you think about the (computer science/mathematical/engineering) consequence of these two statements.

\SOLUTION

(to be completed)


%------------------------------------------------------------------------------
\newpage
\textsc{Spoilers}

Here's an important hint:
There are two extremely important proof techniques when it comes to
the size of infinite sets:
the zigzag argument (I used this to prove $|\N| = |\Q|$)
and Cantor's diagonalization argument (I used this to prove $|\N| < |\R|$,
i.e., $\R$ is not countable).
Both methods can be used in this assignment, i.e.,
one method can be used in one of the three questions
and the other can be used in another question.

%------------------------------------------------------------------------------
\newpage

\textsc{Instructions}

In \verb!main.tex! change the email address in
\begin{console}
\renewcommand\AUTHOR{jdoe5@cougars.ccis.edu} 
\end{console}
yours.
In the bash shell, execute \lq\lq \verb!make!" to recompile \verb!main.pdf!.
Execute \lq\lq \verb!make v!" to view \verb!main.pdf!.
Execute \lq\lq \verb!make s!" to create \verb!submit.tar.gz! for submission.

For each question, you'll see boxes for you to fill.
You write your answers in \verb!main.tex! file.
For small boxes, if you see
\begin{console}[frame=single=single,fontsize=\small]
1 + 1 = \answerbox{}.
\end{console}
you do this:
\begin{console}[frame=single=single,fontsize=\small]
1 + 1 = \answerbox{2}.
\end{console}
\verb!answerbox! will also appear in
\lq\lq true/false" and \lq\lq multiple-choice"
questions.

For longer answers that needs typewriter font, if you see
\begin{console}[frame=single=single, fontsize=\small]
Write a C++ statement that declares an integer variable name x.
\begin{answercode}
\end{answercode}
\end{console}
you do this:
\begin{console}[frame=single=single, fontsize=\small]
Write a C++ statement that declares an integer variable name x.
\begin{answercode}
int x;
\end{answercode}
\end{console}
\verb!answercode! will appear in questions asking for
code, algorithm, and program output.
In this case, indentation and spacing is significant.
For program output, I do look at spaces and newlines.

For long answers (not in typewriter font) if you see
\begin{console}[frame=single=single, fontsize=\small]
What is the color of the sky?
\begin{answerlong}
\end{answerlong}
\end{console}
you can write
\begin{console}[frame=single=single, fontsize=\small]
What is the color of the sky?
\begin{answerlong}
The color of the sky is blue.
\end{answerlong}
\end{console}
For students beyond 245: You can put \LaTeX\ commands in
\verb!answerbox! and 
\verb!answerlong!.

A question that begins with \lq\lq T or F or M"
requires you to identify whether it is true or
false, or meaningless.
\lq\lq Meaningless" means something's wrong with the statement and
it is not well-defined.
Something like \lq\lq $1 +_2$" or \lq\lq $\{2\}^{\{3\}}$" is not
well-defined.
Therefore a question such as
\lq\lq Is $42 = 1 +_2$ true or false?" or
\lq\lq Is $42 = \{2\}^{\{3\}}$ true or false?"
does not make sense.
\lq\lq Is $P(42) = \{42\}$ true or false?" is meaningless because $P(X)$
is only defined if $X$ is a set.
For \lq\lq Is 1 + 2 + 3 true or false?", \lq\lq 1 + 2 + 3" is well--defined but
as a
\lq\lq numerical expression", not as a \lq\lq proposition", i.e.,
it cannot be true or false.
Therefore \lq\lq Is 1 + 2 + 3 true or false?" is also not a well-defined
question.

When writing results of computations, make sure it's simplified.
For instance write $2$ instead of $1 + 1$.
When you write down sets,
if the answer is $\{1\}$, I do not
want to see $\{1, 1\}$.

When writing a counterexample, always write the simplest.

Here are some examples (see \verb!instructions.tex! for details):

\begin{enumerate}

 \item \tf: 1 + 1 = 2 \dotfill\answerbox{T}
 
 \item \tf: 1 + 1 = 3 \dotfill\answerbox{F}
 
 \item \tf: $1 +^2 =$ \dotfill\answerbox{M}
 
 \item $1 + 2 =$ \answerbox{3}
 
 \item Write a C++ statement to declare an integer variable named
 \verb!x!.
 \begin{answercode}
int x;
 \end{answercode}

 \item Solve $x^2 - 1 = 0$.
 \begin{answerlong}
 Since $x^2 - 1 = (x-1)(x+1)$, $x^2 - 1 = 0$ implies $(x-1)(x+1)=0$.
 Therefore $x - 1 = 0$ or $x = -1$.
 Hence $x = 1$ or $x = -1$.
 \end{answerlong}

 \item
 \begin{mcq}
 {Which is true?}{\answerbox{C}}
 {$1+1=0$}
 {$1+1=1$}
 {$1+1=2$}
 {$1+1=3$}
 {$1+1=4$}
 \end{mcq}


\end{enumerate}

\end{document}
