\input{thispreamble.tex}

\renewcommand\AUTHOR{nweadick1@cougars.ccis.edu} % CHANGE TO YOURS

\begin{document}
\topmattertwo

In \LaTeX\, math notation is enclosed by the \verb!$! symbols.
For instance \verb!$x = a_{1} + b^{2}$! gives you $x = a_{1} + b^{2}$.
For emphasis you can write also write it as
\verb!\[ x = a_{1} + b^{2} \]!
to center your math:
\[
x = a_{1} + b^{2}
\]
For more information about \LaTeX\, go to my
website
\href{http://bit.ly/yliow0}{http://bit.ly/yliow0},
click on \verb!Yes!
you are one of my students,
then look for
\href{https://drive.google.com/file/d/0BzjYrK0VFuMWZm5xV0kyR3J2Zm8/view?usp=sharing}{latex.pdf}.)

%------------------------------------------------------------------------------
\newpage
\textsc{Example.}
Prove that
\begin{enumerate}[nosep]
\item[(a)] Reflexive: $|X| = |X|$
\item[(b)] Symmetric: If $|X| = |Y|$, then $|Y| = |X|$.
\item[(c)] Transitive: If $|X| \leq |Y|$ and $|Y| \leq |Z|$, then
$|X| \leq |Z|$.
\end{enumerate}

Recall the following definitions:
$|X| \leq |Y|$ means there is a 1--1 function $X \rightarrow Y$
and $|X| = |Y|$ means there there is a 1--1 and onto function $X \rightarrow Y$.

\SOLUTION

(a)

Let $X$ be a set and a function, s.t $f:X \rightarrow X$
$f(x)=x$, where $x$ is an element of $X$
$f$ is a 1-1 and onto function. 
since $x,x'$ are elements of $X$
$f(x) = f(x')$ implies $x=x'$

Therefore, $|X| = |X|$

(b) 

Let $f:X \rightarrow Y$ that is 1-1 and onto. Since $f$ is bijective it has an inverse function.
$f^-1: Y \rightarrow X$ , which is 1-1 and onto. 

Therefore, if $|X| = |Y|$ then, $|Y| = |X|$

(c)

Let $f:X \rightarrow Y$ that is 1-1 and $g:Y \rightarrow Z$ which is 1-1. and $h:X \rightarrow Z$.
then the composite function $gof$ = $h:X \rightarrow Z$

Since, this is a composite funct of two 1-1 functions then the composite is 1-1

Therefore, If $|X| \leq |Y|$ and $|Y| \leq |Z|$, then $|X| \leq |Z|$.





\input{example.tex}

%------------------------------------------------------------------------------
\newpage
\nextq Rosen 8th edition, section 2.5, question 18.

Prove that for sets $A$ and $B$, if $|A| = |B|$, then $|P(A)| = |P(B)|$.

\textsc{Warning:} $A$ and $B$ need not be finite.

\SOLUTION

Let $f: A \rightarrow B$ which is 1-1 and onto. 

Let $g: P(A) \rightarrow P(B)$

let $g$ be defined as $g(X) = f(X)$ 

Now, for all $X \subseteq A$, that is 1-1 and onto.

Therefore, if $|A| = |B|$, then $|P(A)| = |P(B)|$

\input{2-5-18.tex}

%------------------------------------------------------------------------------
\newpage
\nextq Rosen 8th edition, section 2.5, question 27.

Prove that the union of countably many countable sets is countable.
      
Let me explain what the question is asking.
Note that you already know what
is a \textit{countable set} -- it's a set $X$ such that $X$ is finite, say
$|X| = |\{1, 2, 3, ..., n\}|$, or $|X| = |\N|$.
And a union \textit{countably many sets} means you are trying to
compute the union of of a collection of sets and this \textit{collection} is
countable.
This means that you can label the sets of the collection in a countable way.
So either
(a) there is a finite number of sets (say $n$ of them) and the union is
\[
A_1 \cup A_2 \cup \cdots A_n = \bigcup_{i \in \{1, 2, 3, \cdots n\}} A_i
\]
i.e., the collection of sets $\{A_1, A_2 ..., A_n\}$ is finite
or
(b) there is infinitely countable number of sets and the union is
\[
A_1 \cup A_2 \cup \cdots = \bigcup_{i \in \N} A_i
\]
i.e., the collection of sets $\{A_1, A_2 ..., \}$ is infinitely countable.
An example of a uncountable union is when you can label the sets with $\R$.
For instance for each $x \in \R$, let $A_x = [x-1, x+1)$ and then you take
\[
\bigcup_{x \in \R} A_x
\]
Something like $\bigcup_{x \in [0,1)} A_x$ is also an uncountable union.
Therefore you need to prove this:
(a) If $\{A_1, A_2, ..., A_n\}$ is countable and each $A_i$ is countable, then
$\bigcup_{i \in \{1, 2, 3, \cdots n\}} A_i$ is also countable and
(b) If $\{A_1, A_2, ...\}$ is countable and each $A_i$ is countable, then
$\bigcup_{i \in \N} A_i$ is also countable.

Remember: \lq\lq countable" means (a) finite or (b) countably infinite.

\textsc{Note:} The solution is provided in the textbook.
Study the solution carefully and write it in your own words.
The solutions at the back of the textbook is usually very brief and
sometimes require clarifications and filling in some details.
Even better: See if you can prove it in a different, simpler, and clearer way.
Yes, there's a simpler proof that uses a proof that I talked about in class
for another theorem.
That's why studying and understanding proofs are important. 

\SOLUTION

Let there be countably many countable sets. s.t A1, A2, A3,....

$ A1 = \{a11,a12,a13,...\}$

$ A2 = \{a21,a22,a23,...\}$

$ A3 = \{a31,a32,a33,...\}$

You can construst such a set.

That the element can be listed $\{ai1,ai2,ai3,...\}$

$\bigcup_{i=1} A_i$ 

The elements can be listed aij with i + j = 2, then all terms aij
with i + j = 3, then all terms aij with i + j = 4\dots

Therefore, the union of countably many countable sets is countable.

\input{2-5-27.tex}

%------------------------------------------------------------------------------
\newpage
\nextq Rosen 8th edition, section 2.5, question 37.

Prove that there are countably many programs that can be written in a
particular programming language.

(The proof is exactly the same if you change the questions to \lq\lq prove
that there are countable many novels that can be written in a particular
alphabetic language".)

\textsc{Note:} The solution is provided in the textbook.
Study the solution carefully and write it in your own words.
See if you can find a simpler adn clearer proof.

\SOLUTION

There is number of potential strings for a finate alphabet. The number of strings is length n and n is positive.
Now, for a particular programming language there countab many program. We can say that since the set
 all programs would be a subset of all strings of a finite alphabet, which is countable.
Then, the set of all programs must be countable since it is a subset of a countable set. 

Therefore, there are countably many programs that can be written in a particular
programming language.
\input{2-5-37.tex}

%------------------------------------------------------------------------------
\newpage
\nextq Rosen 8th edition, section 2.5, question 38 (modified).

Prove that there are uncountably many functions from $\N$ to $\{ 0, 1 \}$.

Note: The previous questions says that there are \textit{countably}
many C++ programs.
This questions says that there are \textit{uncountably} many boolean functions
$\N \rightarrow \{0,1\}$ is uncountable.
I'll let you think about the (computer science/mathematical/engineering) consequence of these two statements.

\SOLUTION

To Prove that there are uncountably many functions from N to \{0, 1\}. 
I will suppose that there are countably many functions from N to \{0, 1\}.
and arrive at a contradiction. 

Assume, all the possible functions can be listed f1,f2,f3,\dots

We can form a new function that adds a rule changes what zero gets mapped to of any function.
Which creates a new function. If the element is mapped to 0 then map it to 1. 
If the element is mapped to 1 then map it 0. This can be applied to any element of the set $\N$.
This is a contradiction because you for any functon you can a single 
elemets map and get a new function. 

Therfore, here are uncountably many functions from $\N$ to $\{ 0, 1 \}$.






\input{2-5-38.tex}

%------------------------------------------------------------------------------
\newpage
\textsc{Spoilers}

Here's an important hint:
There are two extremely important proof techniques when it comes to
the size of infinite sets:
the zigzag argument (I used this to prove $|\N| = |\Q|$)
and Cantor's diagonalization argument (I used this to prove $|\N| < |\R|$,
i.e., $\R$ is not countable).
Both methods can be used in this assignment, i.e.,
one method can be used in one of the three questions
and the other can be used in another question.

%------------------------------------------------------------------------------
\newpage
\input{instructions.tex}
\end{document}
