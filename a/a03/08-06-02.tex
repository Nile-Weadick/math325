Let 
\begin{align*}
U &= \text{the set of applicants} \\
P_1(x) &= (x \text{ has altitude sickness}) \\
P_2(x) &= (x\text{ is not in good enough shape}) \\
P_3(x) &= (x\text{ has allergies}) 
\end{align*}
We are given the following information
\begin{align*}
|U| &= 1000 \\
N(P_1) &= 450 \\
N(P_2) &= 622 \\
N(P_3) &= 30 \\
N(P_1 P_2) &= 111 \\
N(P_1 P_3) &= 14 \\
N(P_2 P_3) &= 18 \\
N(P_1 N_2 P_3) &= 9
\end{align*}
The required number is $N(P_1' P_2' P_3')$.
By the principle of inclusion-exclusion,
\begin{align*}
N(P_1' P_2' P_3') 
&= |U| \\
&\hspace{0.5cm} - (N(P_1) + N(P_2) + N(P_3) \\  
&\hspace{0.5cm} + (N(P_1 P_2) + N(P_1 P_3) + N(P_2 P_3) \\  
&\hspace{0.5cm} - (N(P_1 P_2 P_3) \\ 
&= 1000 - (450 + 622 + 30) + (111 + 14 + 18) - 9 \\
&= 32
\end{align*}
Therefore the number of applicants that qualify is 32.

ANSWER:
\answerbox{32}
\qed
