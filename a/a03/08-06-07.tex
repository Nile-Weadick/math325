
[HINT: 
Let 
$U = \{x \in \Z \mid 2 \leq x \leq 10000\}$;
the integer 1 is excluded from $U$ since 1 is a second power,
third power, fourther power, etc. 
Define $A_1 = \{x \in U \mid x \text{ is a second power}\}$,
$A_2 = \{x \in U \mid x \text{ is a third power}\}$, 
$A_3 = \{x \in U \mid x \text{ is a fourth power}\}$, 
etc.
You hope that these sets and their intersections are easier to count.
If this is the case, then the inclusion-exclusion principle can be used.
For instance in the case of $A_1$,
we have $A_1 = \{2^2, 3^2, ..., 100^2\}$.
Therefore $|A_1| = 99$.]

Note that we only need to consider prime powers.
For instance the integer $16$ is a fourth-power and also a second-power.

Let $U = \{2, 3, \ldots, 10000\}$ and 
\begin{align*}
P_2(x)    &= (x \text{ is a second power}) \\
P_3(x)    &= (x \text{ is a third power}) \\
P_5(x)    &= (x \text{ is a 5-th power}) \\
P_7(x)    &= (x \text{ is a 7-th power}) \\
P_{11}(x) &= (x \text{ is a 11-th power})
\end{align*}
(Note: I'm excluding 1 because 1 is a $d$--th power for all 
positive integer $d$ which  means that if $1$ is included in $U$,
then all the $P_i$ will be non-empty and the there will be many 
non-zero
terms in the inclusion-exclusion computation.
To force many terms to be zero, I prefer to remove 1 first
from $U$ right at the beginning. 
If you don't, you will still get the same answer.)

Note that the $d$--th powers are
\[
1^d, 2^d, 3^d, ...
\]
The number of integer $d$--th power $\leq 10000$, is
\[
\floor{10000^{1/d}}
\]
Note that 
\begin{align*}
      10000^{1/d} < 2 
&\iff   10000 < 2^d \\
&\iff   \log_2 10000 < d \\
&\iff   \log_2 10000 < d \\
&\iff   13.287712... < d
\end{align*}
Therefore for $d \geq 14$, the only integer which is a $d$--th power
is 1.
Hence, on excluding 1, the number of integers in $U$ which are
$d$--th powers is
\[
\floor{10000^{1/d}} - 1
\]
and if 0 if $d > 13$.

By the inclusion-exclusion principle
\begin{align*}
&N(P_2 P_3 P_5 P_7 P_{11}) \\
&= |U|  \\
&\hspace{0.5cm} - (\floor{10000^{1/2}} - 1 +
                   \floor{10000^{1/3}} - 1 + 
                   \floor{10000^{1/5}} - 1 + 
                   \floor{10000^{1/7}} - 1 + 
                   \floor{10000^{1/11}} - 1 + \\
&\hspace{1cm}      + \floor{10000^{1/13}} - 1
                  ) \\
&\hspace{0.5cm} + (\floor{10000^{1/(2\cdot 3)}} - 1 +
                   \floor{10000^{1/(2\cdot 5)}} - 1
                  ) \\
&= 9999 - (99 + 20 + 5 + 2 + 1 + 1) + (3 + 1)\\
&= 9875
\end{align*}
Hence the number of integers from 1 to 10000 which are $d$--powers
for some $d > 1$ is 9875.

ANSWER:
\answerbox{9875}
\qed


\vspace{1cm}

\textsc{Note.}
Here's a quick-and-dirty check (i.e. it's not efficient).
The program computes the $d$--th powers which are at most 10000
for $d > 1$ 
\begin{Verbatim}[frame=single,fontsize=\footnotesize]
xs = []
for x in range(1, 10001):
    for d in range(2, 15):
        y = x**d
        if y <= 10000 and y not in xs:
            xs.append(y)
xs.sort()
print(len(xs))
print(xs)
\end{Verbatim}
The number of integers in \verb!xs! is 125.
Therefore the non-powers are $10000 - 125 = 9875$.
