
Let $U = \{1, 2, 3, ..., 200\}$.
Note that $\floor{\sqrt{200}} = \floor{14.14...} = 14$.
The primes less than or equal to 14 are 2, 3, 5, 7, 11, 13.

First we count the number of integers in $U$ which are
not divisible by 2, 3, 5, 7, 11, 13.
Define
\begin{align*}
P_p(x) = (x \text{ divisible by $p$})
\end{align*} 
for $p = 2, 3, 5, 7, 11$, and $13$.
We want to compute $N(P_2' P_3' P_5' P_7' P_{11}' P_{13}')$.
The required number is then
\[
N(P_2' P_3' P_5' P_7' P_{11}' P_{13}') + |\{2, 3, 5, 7, 11, 13\}| - 1
=
N(P_2' P_3' P_5' P_7' P_{11}' P_{13}') + 5
\]

By the principle of inclusion-exclusion,
\begin{align*}
&N(P_2' P_3' P_5' P_7' P_{11}' P_{13}') \\
&= 200  \\
&\hspace{0.5cm} - 
\left( 
\floor{\frac{200}{2}}
+ \floor{\frac{200}{3}} 
+ \floor{\frac{200}{5}} 
+ \floor{\frac{200}{7}} 
+ \floor{\frac{200}{11}} 
+ \floor{\frac{200}{13}} 
\right)  \\
&\hspace{0.5cm} + 
\biggl( 
\floor{\frac{200}{2 \cdot 3}}
+ \floor{\frac{200}{2 \cdot 5}} 
+ \floor{\frac{200}{2 \cdot 7}} 
+ \floor{\frac{200}{2 \cdot 11}} 
+ \floor{\frac{200}{2 \cdot 13}} 
+ \floor{\frac{200}{3 \cdot 5}} 
+ \floor{\frac{200}{3 \cdot 7}} 
+ \floor{\frac{200}{3 \cdot 11}} 
\\
&\hspace{1.25cm}
+ \floor{\frac{200}{3 \cdot 13}} 
+ \floor{\frac{200}{5 \cdot 7}}
+ \floor{\frac{200}{5 \cdot 11}} 
+ \floor{\frac{200}{5 \cdot 13}} 
+ \floor{\frac{200}{7 \cdot 11}} 
+ \floor{\frac{200}{7 \cdot 13}} 
+ \floor{\frac{200}{11 \cdot 13}} 
\biggr)  \\
&\hspace{0.5cm} -
\biggl( 
\floor{\frac{200}{2 \cdot 3 \cdot 5}}
+\floor{\frac{200}{2 \cdot 3 \cdot 7}}
+\floor{\frac{200}{2 \cdot 3 \cdot 11}}
+\floor{\frac{200}{2 \cdot 3 \cdot 13}}
+\floor{\frac{200}{2 \cdot 5 \cdot 7}}
+\floor{\frac{200}{2 \cdot 5 \cdot 11}}
\\
&\hspace{1.25cm}
+\floor{\frac{200}{2 \cdot 5 \cdot 13}}
+\floor{\frac{200}{3 \cdot 5 \cdot 7}}
+\floor{\frac{200}{3 \cdot 5 \cdot 11}}
+\floor{\frac{200}{3 \cdot 5 \cdot 13}} 
+\floor{\frac{200}{3 \cdot 7 \cdot 11}}
+\floor{\frac{200}{3 \cdot 7 \cdot 13}}
\\
&\hspace{1.25cm}
+\floor{\frac{200}{3 \cdot 11 \cdot 13}}
+\floor{\frac{200}{5 \cdot 7 \cdot 11}}
+\floor{\frac{200}{5 \cdot 11 \cdot 13}} \biggr)
\end{align*}
Note that the other terms are 0 since
\[
\floor{\frac{200}{2 \cdot 3 \cdot 5 \cdot 7}} = \floor{\frac{200}{210}} = 0
\]
Hence we have 
\begin{align*}
N(P_2' P_3' P_5' P_7' P_{11}' P_{13}') 
&= 200 - 267 + 132 - 24 \\
&= 41
\end{align*}

Therefore the number of primes less than or equal to 200 is
41 + 5 = 46.

ANSWER:
\answerbox{46}
\qed


\vspace{1cm}
\textsc{Note.}
You can (and should) of course write a simple function to count primes
up to 200:
\begin{Verbatim}[frame=single,fontsize=\footnotesize]
>>> def isprime(n):
...     for d in range(2, n):
...         if n % d == 0: return False
...     return True
... 
>>> count = 0
>>> for x in range(2, 201):
...     if isprime(x): count += 1
... 
>>> print(count)
46
\end{Verbatim}
The computation of the terms in the inclusion-exclusion can of course
be done with a program too.
\begin{Verbatim}[frame=single,fontsize=\footnotesize]
>>> p = [2,3,5,7,11,13]
>>> for x in p:
...     s += 200/x
... 
>>> print(s)
267
>>> s = 0
>>> for i in range(6):
...     for j in range(i+1,6):
...         s += 200/(p[i] * p[j])
... 
>>> print(s)
132
>>> s = 0
>>> for i in range(6):
...     for j in range(i+1,6):
...         for k in range(j+1,6):
...             s += 200/(p[i] * p[j] * p[k])
... 
>>> print(s)
24
>>> s = 0
>>> for i in range(6):
...     for j in range(i+1,6):
...         for k in range(j+1,6):
...             for l in range(l+1,6):
...                 s += 200/(p[i] * p[j] * p[k] * p[l])
... 
>>> print(s)
0
>>> 200 - 267 + 132 - 24
41
\end{Verbatim} 

If you want to checkout the primes you can do this:
\begin{Verbatim}[frame=single,fontsize=\footnotesize]
>>> ps = [x for x in range(2,201) if isprime(x)]
>>> print(len(ps))
46
>>> for x in ps:
...     print(x)
... 
2
3
5
7
11
13
17
19
23
29
31
37
41
43
47
53
59
61
67
71
73
79
83
89
97
101
103
107
109
113
127
131
137
139
149
151
157
163
167
173
179
181
191
193
197
199
>>> 
\end{Verbatim}
