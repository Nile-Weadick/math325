\input{thispreamble.tex}

\renewcommand\AUTHOR{nweadick1@cougars.ccis.edu} % CHANGE TO YOURS

\begin{document}
\topmattertwo

\textsc{Objectives}
\begin{enumerate}
\item Solve counting problems using principle of
inclusion-exclution.
\end{enumerate}
\vspace{1cm}



The following are practice problems for self-study:
\begin{enumerate}[nosep]
\li Rosen 8th edition, section 8.6: Odd numbered problems 1-25 except 15.
Solutions to many of the questions are included.
23 was covered in class.
Some questions (example 19, 24) are small theorems -- you are
encourage to try them and talk to me if you have questions.
\end{enumerate}
For this assignment the problems you need to solve are
\begin{enumerate}[nosep]
\li Rosen 8th edition, section 8.6: \underline{questions 4, 6, 10, 17}.
\end{enumerate}
Explain your work completely using the solutions provided as
guide and examples on how to write math properly.

In \LaTeX\, math notation is enclosed by the \verb!$! symbols.
For instance \verb!$x = a_{1} + b^{2}$! gives you $x = a_{1} + b^{2}$.
For emphasis you can write also write it as
\verb!\[ x = a_{1} + b^{2} \]!
to center your math:
\[
x = a_{1} + b^{2}
\]
For binomial coefficients, \verb!$\binom{5}{2}$! will give you 
$\binom{5}{2}$.
You can also look at the solutions, find something that you can use,
copy-and-paste, and modify.

For more information about \LaTeX\, go to my
website
\href{http://bit.ly/yliow0}{http://bit.ly/yliow0},
click on \verb!Yes!
you are one of my students,
then look for
\href{https://drive.google.com/file/d/0BzjYrK0VFuMWZm5xV0kyR3J2Zm8/view?usp=sharing}{latex.pdf}.)
Even easier: ask questions in CCCS discord.




For each question in section 8.6,
you must define your universe $U$ and
the propositional formulas $P_1, P_2, P_3$.
(Of course different problems have different numbers of
propositional formulas; you might need more or less.) 
You can use other subscripts such as $P_A, ...$
if you find it easier to read the solution that way.

After stating the above, you have to argue that the 
required number is $N(P_1' P_2' P_3')$.
In most cases, it should be obvious.
If that's the case you just say
\lq\lq The required number is clearly $N(P_1' P_2' P_3' ...)$''.
In some cases it might not be clear.
You h ave to judge for yourself.

After stating the above, you must state the version of P.I.E. you are
using. For instance you might want to say \lq\lq
By the principle of inclusion-exclusion,
\[
N(P_1' P_2' P_3') = |U| - ...
\]
Do NOT use $U$ or $P_i$'s if it's not defined.

After that you should compute the values $|U|$, $N(P_1)$, ..., $N(P_1P_2)$, ...
If the computation of $N(P_2)$ is similar to $N(P_1)$, then 
you may say \lq\lq Similarly $N(P_2) = ...$".
For easy and straightforward cases (example: the number is something from
Discrete I), you simply state the number. For instance:
\lq\lq We have $|U| = 2^5$.'' or \lq\lq Clearly $|U| = 2^5$.''
Learn to write \textit{full} sentences: 
writing \lq\lq $|U| = 2^5$'' out of nowhere
is very confusing and misleading.

Again you do not need to show trivial computations.
The general rule is this: 
If the result is from Discrete I (or previous classes), then you can use the
relevant formula right away without too much explanation.
If a result you want to use is not from previous classes,
then you need to explain how to change the computation to one
involving previous results.

Once all the quantities are stated/computated, you substitute
them into the P.I.E. that you have stated.

Finally state what you have found clearly.
For instance you might want to say \lq\lq Therefore
the number of students studying Biology and Physics is 115''.

Draw a box around your final answer if the question required
an explicit answer (i.e., if it's not a proof question).

%------------------------------------------------------------------------------
\newpage
\nextq Rosen 8th edition, section 8.6, question 1.

\SOLUTION

Let $U$ be the apples in the bushel.
Define
\begin{align*}
P_w(x) &= (\text{there are worms in $x$}) \\
P_b(x) &= (\text{there are bruises on $x$})
\end{align*}
The required number is the number of apples that can be sold
which is the number of apples without worms and without bruises, 
i.e., the required number is $N(P_w' P_b')$.

By the principle of inclusion-exclusion
\[
N(P_w' P_b') = |U| - (N(P_w) + N(P_b)) + N(P_wP_b)
\]
We are given 
\begin{align*}
|U| &= 100 \\
N(P_w) &= 20 \\
N(P_b) &= 15 \\
N(P_w P_b) &= 10
\end{align*}
Therefore
\[
N(P_w' P_b') = 100 - (20 + 15) + 10 = 75
\]
Hence the number of apples that can be sold is 75.

ANSWER:
\answerbox{75}
\qed





%------------------------------------------------------------------------------
\newpage
\nextq Rosen 8th edition, section 8.6, question 2.

\SOLUTION

Let 
\begin{align*}
U &= \text{the set of applicants} \\
P_1(x) &= (x \text{ has altitude sickness}) \\
P_2(x) &= (x\text{ is not in good enough shape}) \\
P_3(x) &= (x\text{ has allergies}) 
\end{align*}
We are given the following information
\begin{align*}
|U| &= 1000 \\
N(P_1) &= 450 \\
N(P_2) &= 622 \\
N(P_3) &= 30 \\
N(P_1 P_2) &= 111 \\
N(P_1 P_3) &= 14 \\
N(P_2 P_3) &= 18 \\
N(P_1 N_2 P_3) &= 9
\end{align*}
The required number is $N(P_1' P_2' P_3')$.
By the principle of inclusion-exclusion,
\begin{align*}
N(P_1' P_2' P_3') 
&= |U| \\
&\hspace{0.5cm} - (N(P_1) + N(P_2) + N(P_3) \\  
&\hspace{0.5cm} + (N(P_1 P_2) + N(P_1 P_3) + N(P_2 P_3) \\  
&\hspace{0.5cm} - (N(P_1 P_2 P_3) \\ 
&= 1000 - (450 + 622 + 30) + (111 + 14 + 18) - 9 \\
&= 32
\end{align*}
Therefore the number of applicants that qualify is 32.

ANSWER:
\answerbox{32}
\qed


%------------------------------------------------------------------------------
\newpage
\nextq Rosen 8th edition, section 8.6, question 3.

\SOLUTION


Let 
\[
U = \{(x_1, x_2, x_3) \in \N^3 \mid x_1 + x_2 + x_3 = 13 \}
\]
where $\N = \{0, 1, 2, 3, ...\}$.
Define
\begin{align*}
P_1(x_1, x_2, x_3) = (x_1 \geq 6) \\
P_2(x_1, x_2, x_3) = (x_2 \geq 6) \\
P_3(x_1, x_2, x_3) = (x_3 \geq 6) 
\end{align*}

The required number is the number of $(x_1, x_2, x_3)$
satisfying 
\[
x_1 + x_2 + x_3 = 13, \,\,\,\,\,
0 \leq x_1 < 6, \,\,\,\,\,
0 \leq x_2 < 6, \,\,\,\,\,
0 \leq x_3 < 6
\]
i.e., $N(P_1' P_2' P_3')$.
By the inclusion-exclusion principle, 
\begin{align*}
N(P_1' P_2' P_3')
&= |U| \\
&\hspace{0.5cm} - (N(P_1) + N(P_2) + N(P_3)) \\ 
&\hspace{0.5cm} + (N(P_1P_2) + N(P_1P_3) + N(P_2 P_3)) \\ 
&\hspace{0.5cm} - (N(P_1 P_2 P_3))
\end{align*}

We have
\[
|U| = \frac{(13 + 2)!}{13! 2!} = \binom{15}{2}
\]
(This is from Discrete I, i.e., it is the number of distribution
of 13 0's into 3 boxes, which is the same as the number of 
permutations of 13 0's and 2 1's).

$N(P_1)$ is the number of solutions to
\[
x_1 + x_2 + x_3 = 13, \,\,\, x_1 \geq 6, \,\,\, x_2 \geq 0, \,\,\, x_3 \geq 0
\]
which is the number of solutions to 
\[
(x_1 - 6) + x_2 + x_3 = 13 - 6, \,\,\, x_1 - 7 \geq 0, \,\,\, x_2 \geq 0, \,\,\, x_3 \geq 0
\]
which is the number of solutions to 
\[
x_1' + x_2 + x_3 = 7, \,\,\, x_1' \geq 0, \,\,\, x_2 \geq 0, \,\,\, x_3 \geq 0
\]
which is $\frac{(7 + 2)!}{7! 2!} = \binom{9}{2}$.
By symmetry, clearly $N(P_2) = N(P_3) = \binom{9}{2}$.


$N(P_1P_2)$ is the number of solutions to
\[
x_1 + x_2 + x_3 = 13, \,\,\, x_1 \geq 6, \,\,\, x_2 \geq 6, \,\,\, x_3 \geq 0
\]
which is the number of solutions to 
\[
(x_1 - 6) + (x_2 - 6) + x_3 = 13 - 6 - 6, \,\,\, x_1 - 6 \geq 0, \,\,\, x_2 - 6 \geq 0, \,\,\, x_3 \geq 0
\]
which is the number of solutions to 
\[
x_1' + x_2' + x_3 = 1, \,\,\, x_1' \geq 0, \,\,\, x_2 \geq 0, \,\,\, x_3 \geq 0
\]
which is $\binom{1 + 2}{1! 2!} = \binom{3}{1}$.
By symmetry, clearly $N(P_1P_3) = N(P_2P_3) = \binom{3}{1}$.

$N(P_1 P_2 P_3)$ is the number of solutions to
\[
x_1 + x_2 + x_3 = 13, \,\,\, x_1 \geq 6, \,\,\, x_2 \geq 6, \,\,\, x_3 \geq 6
\]
which is the number of solutions to 
\[
(x_1 - 6) + (x_2 - 6) + (x_3-6) = 13 - 6 - 6 - 6, \,\,\, x_1 - 6 \geq 0, \,\,\, x_2 - 6 \geq 0, \,\,\, x_3 - 6\geq 0
\]
which is the number of solutions to 
\[
x_1' + x_2' + x'_3 = -5, \,\,\, x_1' \geq 0, \,\,\, x'_2 \geq 0, \,\,\, x'_3 \geq 0
\]
which is clearly 0.


Therefore
\begin{align*}
N(P_1' P_2' P_3')
&= \binom{15}{2} - 3 \binom{9}{2}  + 3 \binom{3}{1} - 0 \\
&= \frac{15 \cdot 14}{2} - 3 \cdot \frac{8 \cdot 7}{2} + 3 \cdot 3 - 0\\
&= 6
\end{align*}
The number of solutions is 6.

ANSWER:
\answerbox{6}
\qed


%------------------------------------------------------------------------------
\newpage
\nextq Rosen 8th edition, section 8.6, question 4.

\SOLUTION
$ N = \binom{17+3}{17} = 1140$

$N(P1) = number of solutions with x1 >= 4 = \binom{16}{13} = 560$

$N(P2) = number of solutions with x2 >= 5 = \binom{15}{12} = 455$

$N(P3) = number of solutions with x3 >= 6 = \binom{14}{11} = 364$

$N(P4) = number of solutions with x4 >= 9 = \binom{11}{8} = 165$

$N(P1P2) = \binom{11}{8} = 165$

$N(P1P3) = \binom{10}{7} = 120$

$N(P1P4) = \binom{7}{4} = 35$

$N(P2P3) = \binom{9}{6} = 84$

$N(P2P4) = \binom{6}{3} = 20$

$N(P3P4) = \binom{4}{2} = 6$

$N(P1P2P3) = \binom{4}{2} = 6$

$N(P1P2P4) = 0$

$N(P2P3P4) = 0$

$N(P1P2P3P4) = 6$

$1140 - (560+455+354+165) + (165+120+35+84+20+6) - 6$

\answerbox{20}

\input{08-06-04.tex}

%------------------------------------------------------------------------------
\newpage
\nextq Rosen 8th edition, section 8.6, question 5.

\SOLUTION


Let $U = \{1, 2, 3, ..., 200\}$.
Note that $\floor{\sqrt{200}} = \floor{14.14...} = 14$.
The primes less than or equal to 14 are 2, 3, 5, 7, 11, 13.

First we count the number of integers in $U$ which are
not divisible by 2, 3, 5, 7, 11, 13.
Define
\begin{align*}
P_p(x) = (x \text{ divisible by $p$})
\end{align*} 
for $p = 2, 3, 5, 7, 11$, and $13$.
We want to compute $N(P_2' P_3' P_5' P_7' P_{11}' P_{13}')$.
The required number is then
\[
N(P_2' P_3' P_5' P_7' P_{11}' P_{13}') + |\{2, 3, 5, 7, 11, 13\}| - 1
=
N(P_2' P_3' P_5' P_7' P_{11}' P_{13}') + 5
\]

By the principle of inclusion-exclusion,
\begin{align*}
&N(P_2' P_3' P_5' P_7' P_{11}' P_{13}') \\
&= 200  \\
&\hspace{0.5cm} - 
\left( 
\floor{\frac{200}{2}}
+ \floor{\frac{200}{3}} 
+ \floor{\frac{200}{5}} 
+ \floor{\frac{200}{7}} 
+ \floor{\frac{200}{11}} 
+ \floor{\frac{200}{13}} 
\right)  \\
&\hspace{0.5cm} + 
\biggl( 
\floor{\frac{200}{2 \cdot 3}}
+ \floor{\frac{200}{2 \cdot 5}} 
+ \floor{\frac{200}{2 \cdot 7}} 
+ \floor{\frac{200}{2 \cdot 11}} 
+ \floor{\frac{200}{2 \cdot 13}} 
+ \floor{\frac{200}{3 \cdot 5}} 
+ \floor{\frac{200}{3 \cdot 7}} 
+ \floor{\frac{200}{3 \cdot 11}} 
\\
&\hspace{1.25cm}
+ \floor{\frac{200}{3 \cdot 13}} 
+ \floor{\frac{200}{5 \cdot 7}}
+ \floor{\frac{200}{5 \cdot 11}} 
+ \floor{\frac{200}{5 \cdot 13}} 
+ \floor{\frac{200}{7 \cdot 11}} 
+ \floor{\frac{200}{7 \cdot 13}} 
+ \floor{\frac{200}{11 \cdot 13}} 
\biggr)  \\
&\hspace{0.5cm} -
\biggl( 
\floor{\frac{200}{2 \cdot 3 \cdot 5}}
+\floor{\frac{200}{2 \cdot 3 \cdot 7}}
+\floor{\frac{200}{2 \cdot 3 \cdot 11}}
+\floor{\frac{200}{2 \cdot 3 \cdot 13}}
+\floor{\frac{200}{2 \cdot 5 \cdot 7}}
+\floor{\frac{200}{2 \cdot 5 \cdot 11}}
\\
&\hspace{1.25cm}
+\floor{\frac{200}{2 \cdot 5 \cdot 13}}
+\floor{\frac{200}{3 \cdot 5 \cdot 7}}
+\floor{\frac{200}{3 \cdot 5 \cdot 11}}
+\floor{\frac{200}{3 \cdot 5 \cdot 13}} 
+\floor{\frac{200}{3 \cdot 7 \cdot 11}}
+\floor{\frac{200}{3 \cdot 7 \cdot 13}}
\\
&\hspace{1.25cm}
+\floor{\frac{200}{3 \cdot 11 \cdot 13}}
+\floor{\frac{200}{5 \cdot 7 \cdot 11}}
+\floor{\frac{200}{5 \cdot 11 \cdot 13}} \biggr)
\end{align*}
Note that the other terms are 0 since
\[
\floor{\frac{200}{2 \cdot 3 \cdot 5 \cdot 7}} = \floor{\frac{200}{210}} = 0
\]
Hence we have 
\begin{align*}
N(P_2' P_3' P_5' P_7' P_{11}' P_{13}') 
&= 200 - 267 + 132 - 24 \\
&= 41
\end{align*}

Therefore the number of primes less than or equal to 200 is
41 + 5 = 46.

ANSWER:
\answerbox{46}
\qed


\vspace{1cm}
\textsc{Note.}
You can (and should) of course write a simple function to count primes
up to 200:
\begin{Verbatim}[frame=single,fontsize=\footnotesize]
>>> def isprime(n):
...     for d in range(2, n):
...         if n % d == 0: return False
...     return True
... 
>>> count = 0
>>> for x in range(2, 201):
...     if isprime(x): count += 1
... 
>>> print(count)
46
\end{Verbatim}
The computation of the terms in the inclusion-exclusion can of course
be done with a program too.
\begin{Verbatim}[frame=single,fontsize=\footnotesize]
>>> p = [2,3,5,7,11,13]
>>> for x in p:
...     s += 200/x
... 
>>> print(s)
267
>>> s = 0
>>> for i in range(6):
...     for j in range(i+1,6):
...         s += 200/(p[i] * p[j])
... 
>>> print(s)
132
>>> s = 0
>>> for i in range(6):
...     for j in range(i+1,6):
...         for k in range(j+1,6):
...             s += 200/(p[i] * p[j] * p[k])
... 
>>> print(s)
24
>>> s = 0
>>> for i in range(6):
...     for j in range(i+1,6):
...         for k in range(j+1,6):
...             for l in range(l+1,6):
...                 s += 200/(p[i] * p[j] * p[k] * p[l])
... 
>>> print(s)
0
>>> 200 - 267 + 132 - 24
41
\end{Verbatim} 

If you want to checkout the primes you can do this:
\begin{Verbatim}[frame=single,fontsize=\footnotesize]
>>> ps = [x for x in range(2,201) if isprime(x)]
>>> print(len(ps))
46
>>> for x in ps:
...     print(x)
... 
2
3
5
7
11
13
17
19
23
29
31
37
41
43
47
53
59
61
67
71
73
79
83
89
97
101
103
107
109
113
127
131
137
139
149
151
157
163
167
173
179
181
191
193
197
199
>>> 
\end{Verbatim}


%------------------------------------------------------------------------------
\newpage
\nextq Rosen 8th edition, section 8.6, question 6.

An integer is called squarefree if it is not divisible by
the square of a positive integer greater than 1. Find the
number of squarefree positive integers less than 100.

\SOLUTION

$99$

$- (\floor{\frac{99}{2^2}} + \floor{\frac{99}{3^2}} + \floor{\frac{99}{5^2}}+  \floor{\frac{99}{7^2}})$

$+ \floor{\frac{99}{2^2 \cdot 3^2}}$

$99 - (24 + 11 + 3 + 2) + (2)$

\answerbox{61}

\input{08-06-06.tex}

%------------------------------------------------------------------------------
\newpage
\nextq Rosen 8th edition, section 8.6, question 7.

\SOLUTION


[HINT: 
Let 
$U = \{x \in \Z \mid 2 \leq x \leq 10000\}$;
the integer 1 is excluded from $U$ since 1 is a second power,
third power, fourther power, etc. 
Define $A_1 = \{x \in U \mid x \text{ is a second power}\}$,
$A_2 = \{x \in U \mid x \text{ is a third power}\}$, 
$A_3 = \{x \in U \mid x \text{ is a fourth power}\}$, 
etc.
You hope that these sets and their intersections are easier to count.
If this is the case, then the inclusion-exclusion principle can be used.
For instance in the case of $A_1$,
we have $A_1 = \{2^2, 3^2, ..., 100^2\}$.
Therefore $|A_1| = 99$.]

Note that we only need to consider prime powers.
For instance the integer $16$ is a fourth-power and also a second-power.

Let $U = \{2, 3, \ldots, 10000\}$ and 
\begin{align*}
P_2(x)    &= (x \text{ is a second power}) \\
P_3(x)    &= (x \text{ is a third power}) \\
P_5(x)    &= (x \text{ is a 5-th power}) \\
P_7(x)    &= (x \text{ is a 7-th power}) \\
P_{11}(x) &= (x \text{ is a 11-th power})
\end{align*}
(Note: I'm excluding 1 because 1 is a $d$--th power for all 
positive integer $d$ which  means that if $1$ is included in $U$,
then all the $P_i$ will be non-empty and the there will be many 
non-zero
terms in the inclusion-exclusion computation.
To force many terms to be zero, I prefer to remove 1 first
from $U$ right at the beginning. 
If you don't, you will still get the same answer.)

Note that the $d$--th powers are
\[
1^d, 2^d, 3^d, ...
\]
The number of integer $d$--th power $\leq 10000$, is
\[
\floor{10000^{1/d}}
\]
Note that 
\begin{align*}
      10000^{1/d} < 2 
&\iff   10000 < 2^d \\
&\iff   \log_2 10000 < d \\
&\iff   \log_2 10000 < d \\
&\iff   13.287712... < d
\end{align*}
Therefore for $d \geq 14$, the only integer which is a $d$--th power
is 1.
Hence, on excluding 1, the number of integers in $U$ which are
$d$--th powers is
\[
\floor{10000^{1/d}} - 1
\]
and if 0 if $d > 13$.

By the inclusion-exclusion principle
\begin{align*}
&N(P_2 P_3 P_5 P_7 P_{11}) \\
&= |U|  \\
&\hspace{0.5cm} - (\floor{10000^{1/2}} - 1 +
                   \floor{10000^{1/3}} - 1 + 
                   \floor{10000^{1/5}} - 1 + 
                   \floor{10000^{1/7}} - 1 + 
                   \floor{10000^{1/11}} - 1 + \\
&\hspace{1cm}      + \floor{10000^{1/13}} - 1
                  ) \\
&\hspace{0.5cm} + (\floor{10000^{1/(2\cdot 3)}} - 1 +
                   \floor{10000^{1/(2\cdot 5)}} - 1
                  ) \\
&= 9999 - (99 + 20 + 5 + 2 + 1 + 1) + (3 + 1)\\
&= 9875
\end{align*}
Hence the number of integers from 1 to 10000 which are $d$--powers
for some $d > 1$ is 9875.

ANSWER:
\answerbox{9875}
\qed


\vspace{1cm}

\textsc{Note.}
Here's a quick-and-dirty check (i.e. it's not efficient).
The program computes the $d$--th powers which are at most 10000
for $d > 1$ 
\begin{Verbatim}[frame=single,fontsize=\footnotesize]
xs = []
for x in range(1, 10001):
    for d in range(2, 15):
        y = x**d
        if y <= 10000 and y not in xs:
            xs.append(y)
xs.sort()
print(len(xs))
print(xs)
\end{Verbatim}
The number of integers in \verb!xs! is 125.
Therefore the non-powers are $10000 - 125 = 9875$.


%------------------------------------------------------------------------------
\newpage
\nextq Rosen 8th edition, section 8.6, question 8.

\SOLUTION


Let $X = \{x_1, x_2, x_3, x_4, x_5, x_6, x_7\}$
and $Y = \{y_1, y_2, y_3, y_4, y_5\}$.
A function $f : X \rightarrow Y$ is onto if
\begin{align*}
& y_1 \text{ is in the image of $f$} \\
& y_2 \text{ is in the image of $f$} \\
& y_3 \text{ is in the image of $f$} \\
& y_4 \text{ is in the image of $f$} \\
& y_5 \text{ is in the image of $f$} 
\end{align*}
where the image of $f$ is the set of values attained by $f$, i.e.
the image of $f$ is $\{f(x_1), \ldots, f(x_7)\}$.

Define $U$ to be the set of all functions from $X$ to $Y$ and
\begin{align*}
P_1(f) = (y_1 \text{ is not in the image of $f$})  \\ 
P_2(f) = (y_2 \text{ is not in the image of $f$})  \\
P_3(f) = (y_3 \text{ is not in the image of $f$})  \\
P_4(f) = (y_4 \text{ is not in the image of $f$})  \\
P_5(f) = (y_5 \text{ is not in the image of $f$}) 
\end{align*}
Hence the required number is
\[
N(P_1' P_2' P_3' P_4' P_5')
\]
Note that $|U| = 5^7$.

$N(P_i)$ is the set of functions from $X$ to $Y$ where $y_i$ is
not in the image of the functions.
The number of such functions is $4^7$ since, excluding $y_i$,
there are 4 possible
values for each of the $7$ values of $X$.

$N(P_iP_j)$ (for $i < j$) 
is the number of functions from $X$ to $Y$
except that $y_i$ and $y_j$ are not in the image of the functions.
This means that for each value of $X$, there are 3 possible values.
Hence there are $3^7$ such functions.

Arguing in the same manner, 
\begin{align*}
N(P_i P_j P_k) &= 2^7 \\
N(P_i P_j P_k P_l) &= 1^7 \\
N(P_1 P_2 P_3 P_4 P_5) &= 0^7 
\end{align*}

By the inclusion-exclusion principle, 
\begin{align*}
N(P_1' P_2' P_3' P_4' P_5')
&= 5^7 
 - \binom{5}{1} 4^7
 + \binom{5}{2} 3^7
 - \binom{5}{3} 2^7
 + \binom{5}{4} 1^7
 - \binom{5}{5} 0^7 \\
&= 16800
\end{align*}
The required number is 16800.

ANSWER:
\answerbox{16800}
\qed



\vspace{1cm}
\textsc{Note.} 
Here's a quick check:
\begin{Verbatim}[frame=single,fontsize=\footnotesize]
>>> x1,x2,x3,x4,x5,x6,x7 = 1,2,3,4,5,6,7
>>> y1,y2,y3,y4,y5 = 1,2,3,4,5
>>> X = [x1,x2,x3,x4,x5]
>>> X = [x1,x2,x3,x4,x5,x6,x7]
>>> Y = [y1,y2,y3,y4,y5]
>>> 5**7
78125
>>> fs = []
>>> for a in Y:
...     for b in Y:
...         for c in Y:
...             for d in Y:
...                 for e in Y:
...                     for f in Y:
...                         for g in Y:
...                             fn = [(x1,a),(x2,b),(x3,c),
                                      (x4,d),(x5,e),(x6,f),(x7,g)]
...                             fs.append(fn)
... 
>>> len(fs)
78125
>>> count = 0
>>> for fn in fs:
...     image = []
...     for x,y in fn:
...         if y not in image: image.append(y)
...     image.sort()
...     if image == Y: count += 1
... 
>>> print(count)
16800
>>> 
\end{Verbatim}


%------------------------------------------------------------------------------
\newpage
\nextq Rosen 8th edition, section 8.6, question 9.

\SOLUTION


[HINT: Let the 6 toys be $T = \{t_1, t_2, \ldots, t_6\}$ and 
the three children be $C = \{c_1, c_2, c_3\}$.
Each distribution is a function $f : T \rightarrow C$.
The fact that each child gets at least one toy is the same as saying
the function $f$ is onto.]

The total number of ways is the number of onto functions from a set
of 6 elements to a set of 3.
Let $T$ be a set of 6 elements and $C = \{c_1, c_2, c_3\}$.
let $U = \{f : T \rightarrow C\}$.
Define
\begin{align*}
P_1(f) &= (\text{the image of $f$ does not contain $c_1$})\\
P_2(f) &= (\text{the image of $f$ does not contain $c_2$})\\
P_3(f) &= (\text{the image of $f$ does not contain $c_3$})
\end{align*}
We have $|U| = 3^6$.

$N(P_1)$ is the number of function from $T$ to $C - \{c_1\} = \{c_2, c_3\}$.
Therefore $N(P_1) = 2^6$.
Likewise $N(P_2) = N(P_3) = 2^6$.

$N(P_1 P_2)$ 
is the number of function from $T$ to $C - \{c_1, c_2\} = \{c_3\}$.
Therefore $N(P_1 P_2) = 1^6$.
Likewise $N(P_1 P_3) = N(P_2 P_3) = 1^6$.

$N(P_1 P_2 P_3)$ is the number of functions from $T$ to $C - \{c_1, c_2, c_3\}
 = \emptyset$. There are no such functions.
Therefore $N(P_1 P_2 P_3) = 0$.

Therefore, by the inclusion-exclusion principle, 
the total number of ways is
\[
3^6
- \binom{3}{1} 2^6
+ \binom{3}{2} 1^6
= 729 - 192 + 3
= 540
\] 
The required number is 540.

ANSWER: 
\answerbox{540}
\qed



%------------------------------------------------------------------------------
\newpage
\nextq Rosen 8th edition, section 8.6, question 10.
In how many ways can eight distinct balls be distributed
into three distinct urns if each urn must contain at least
one ball?

$3^8 - (C\binom{3}{1} \cdot 2^8) + (C\binom{3}{2} \cdot 1^8)$

\SOLUTION

\input{08-06-10.tex}

%------------------------------------------------------------------------------
\newpage
\nextq Rosen 8th edition, section 8.6, question 11.

\SOLUTION

????
%\input{08-06-11.tex}

%------------------------------------------------------------------------------
\newpage
\nextq Rosen 8th edition, section 8.6, question 12.

\SOLUTION

The following lists all permutations of 1234.
The derangements are marked with *

\begin{Verbatim}[frame=single,fontsize=\footnotesize]
1234
1243
1324
1342
1423
1432
2134 
2143 *
2314 
2341 *
2413 *
2431 
3124 
3142 *
3214 
3241 
3412 *
3421 *
4123 * 
4132 
4213 
4231 
4312 *
4321 *
\end{Verbatim}
Note that there are 9 derangements.

We can verify that the above exhaustive enumeration is correct
with a computation.
Let
\[
U = \{(x_1, x_2, x_3, x_4) \mid (x_1, x_2, x_3, x_4) \text{ is a permutation  of $1,2,3,4$} \}
\]
and define the following propositional formulas:
\begin{align*}
P_1((x_1, x_2, x_3, x_4)) &= (x_1 = 1) \\
P_2((x_1, x_2, x_3, x_4)) &= (x_2 = 2) \\
P_3((x_1, x_2, x_3, x_4)) &= (x_3 = 3) \\
P_4((x_1, x_2, x_3, x_4)) &= (x_4 = 4) \\
\end{align*}
The number of derangements is $N(P_1' P_2' P_3' P_4')$.
By the inclusion-exclusion principle
\begin{align*}
N(P_1' P_2' P_3' P_4')
&= 4! - \binom{4}{1}3! + \binom{4}{2}2! - \binom{4}{3}1! + \binom{4}{4}0! \\
&= 4! - \binom{4}{1}3! + \binom{4}{2}2! - \binom{4}{3}1! + \binom{4}{4}0! \\
&= 24 - 4 \cdot 6 + 6 \cdot 2 - 4 \cdot 1 + 1 \\
&= 9
\end{align*}


%------------------------------------------------------------------------------
\newpage
\nextq Rosen 8th edition, section 8.6, question 13.

\SOLUTION


The required number is
\begin{align*}
  D_7
  &= 7!\sum_{k=0}^7 (-1)^k\frac{1}{k!} \\
  &= 5040
  \left(
  1 - 1 + \frac{1}{2}  - \frac{1}{6} + \frac{1}{24} - \frac{1}{120}
  + \frac{1}{720} - \frac{1}{5040}
  \right)
  \\
  &= 1854
\end{align*}

\ANSWER
\answerbox{1854}
\qed


%------------------------------------------------------------------------------
\newpage
\nextq Rosen 8th edition, section 8.6, question 14.

\SOLUTION


The probability is
\[
\frac{D_{10}}{10!}
= \frac{10!}{10!} \sum_{k=0}^{10} (-1)^k \frac{1}{k!}
= \sum_{k=0}^{10} (-1)^k \frac{1}{k!}
\simeq 0.3678
\]

\ANSWER
\answerbox{approximately 0.3678}
\qed



%------------------------------------------------------------------------------
\newpage
\nextq Rosen 8th edition, section 8.6, question 15.

\SOLUTION

OMITTED SPRING2022
%\input{08-06-15.tex}

%------------------------------------------------------------------------------
\newpage
\nextq Rosen 8th edition, section 8.6, question 16.

\SOLUTION

This is the number of derangements of $n$ symbols, i.e., the required
number is
\[
D_n = n! \sum_{k=0}^n (-1)^k\frac{1}{k!}
\]
\qed


%------------------------------------------------------------------------------
\newpage
\nextq Rosen 8th edition, section 8.6, question 17.

(Show all work clearly.)

How many ways can the digits 0, 1, 2, 3, 4, 5, 6, 7, 8, 9 be
arranged so that no even digit is in its original position?

Digits {0,1,2,3,4,5,6,7,8,9} can be arranged 10! ways.

There are 5 even numbers in the list.

$10! - \binom{5}{1}9! + \binom{5}{2}8! - \binom{5}{3}7! + \binom{5}{4}6! -\binom{5}{5}5!$

$3628800-181440+403200-50400+3600-120$

\answerbox{2170680}

\SOLUTION

\input{08-06-17.tex}

%------------------------------------------------------------------------------
\newpage
\nextq Rosen 8th edition, section 8.6, question 22.

\SOLUTION


One can prove this using principle of inclusion-exclusion or we can use
Euler's theorem on $phi$-function (question 23 or from class notes) to obtain
\[
\phi(pq) = \phi(p) \phi(q) = (p-1)(q-1)
\]
\qed


%------------------------------------------------------------------------------
\newpage
\nextq Rosen 8th edition, section 8.6, question 23.

\SOLUTION


Euler's theorem for the Euler $\phi$--function was proven
in class.
\qed


%------------------------------------------------------------------------------
\newpage
\input{instructions.tex}
\end{document}
