
Let $X = \{x_1, x_2, x_3, x_4, x_5, x_6, x_7\}$
and $Y = \{y_1, y_2, y_3, y_4, y_5\}$.
A function $f : X \rightarrow Y$ is onto if
\begin{align*}
& y_1 \text{ is in the image of $f$} \\
& y_2 \text{ is in the image of $f$} \\
& y_3 \text{ is in the image of $f$} \\
& y_4 \text{ is in the image of $f$} \\
& y_5 \text{ is in the image of $f$} 
\end{align*}
where the image of $f$ is the set of values attained by $f$, i.e.
the image of $f$ is $\{f(x_1), \ldots, f(x_7)\}$.

Define $U$ to be the set of all functions from $X$ to $Y$ and
\begin{align*}
P_1(f) = (y_1 \text{ is not in the image of $f$})  \\ 
P_2(f) = (y_2 \text{ is not in the image of $f$})  \\
P_3(f) = (y_3 \text{ is not in the image of $f$})  \\
P_4(f) = (y_4 \text{ is not in the image of $f$})  \\
P_5(f) = (y_5 \text{ is not in the image of $f$}) 
\end{align*}
Hence the required number is
\[
N(P_1' P_2' P_3' P_4' P_5')
\]
Note that $|U| = 5^7$.

$N(P_i)$ is the set of functions from $X$ to $Y$ where $y_i$ is
not in the image of the functions.
The number of such functions is $4^7$ since, excluding $y_i$,
there are 4 possible
values for each of the $7$ values of $X$.

$N(P_iP_j)$ (for $i < j$) 
is the number of functions from $X$ to $Y$
except that $y_i$ and $y_j$ are not in the image of the functions.
This means that for each value of $X$, there are 3 possible values.
Hence there are $3^7$ such functions.

Arguing in the same manner, 
\begin{align*}
N(P_i P_j P_k) &= 2^7 \\
N(P_i P_j P_k P_l) &= 1^7 \\
N(P_1 P_2 P_3 P_4 P_5) &= 0^7 
\end{align*}

By the inclusion-exclusion principle, 
\begin{align*}
N(P_1' P_2' P_3' P_4' P_5')
&= 5^7 
 - \binom{5}{1} 4^7
 + \binom{5}{2} 3^7
 - \binom{5}{3} 2^7
 + \binom{5}{4} 1^7
 - \binom{5}{5} 0^7 \\
&= 16800
\end{align*}
The required number is 16800.

ANSWER:
\answerbox{16800}
\qed



\vspace{1cm}
\textsc{Note.} 
Here's a quick check:
\begin{Verbatim}[frame=single,fontsize=\footnotesize]
>>> x1,x2,x3,x4,x5,x6,x7 = 1,2,3,4,5,6,7
>>> y1,y2,y3,y4,y5 = 1,2,3,4,5
>>> X = [x1,x2,x3,x4,x5]
>>> X = [x1,x2,x3,x4,x5,x6,x7]
>>> Y = [y1,y2,y3,y4,y5]
>>> 5**7
78125
>>> fs = []
>>> for a in Y:
...     for b in Y:
...         for c in Y:
...             for d in Y:
...                 for e in Y:
...                     for f in Y:
...                         for g in Y:
...                             fn = [(x1,a),(x2,b),(x3,c),
                                      (x4,d),(x5,e),(x6,f),(x7,g)]
...                             fs.append(fn)
... 
>>> len(fs)
78125
>>> count = 0
>>> for fn in fs:
...     image = []
...     for x,y in fn:
...         if y not in image: image.append(y)
...     image.sort()
...     if image == Y: count += 1
... 
>>> print(count)
16800
>>> 
\end{Verbatim}
