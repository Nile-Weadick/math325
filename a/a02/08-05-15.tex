Let 
\begin{align*}
U &= \{ \text{bit string of length 8} \} \\
A & = \{x \in U \mid x \text{ contains 6 consecutive 0s} \}
\end{align*}
The required number is $|U| - |A|$.

A string in $A$ can be one of the three forms:
It is either
\[
B = \{000000 x y \mid x, y \text{ are bits}\}
\]
or
\[
C = \{x 000000 y \mid x, y \text{ are bits}\}
\]
or
\[
D = \{x y 000000 \mid x, y \text{ are bits}\}
\]
Therefore $|A| = |B \cup C \cup D|$.
By the principle of inclusion-exclusion, 
\[
|B \cup C \cup D|
= |B| + |C| + |D| - (|B \cap C| + |B \cap D| + |C \cap D|) + |B \cap C \cap D|
\]

By the multiplcation principle, 
$|B| = 2 \cdot 2 = 4$ since there are two ways to choose
$x$ and two ways to choose $y$. 
Likewise $|C| = |D| = 4$.

Now note that 
\[
B \cap C = \{000000y \mid y \text{ is a bit}\}
\]
Therefore $|B \cap C| = 2$.
Also, 
\[
C \cap D = \{x 0000000 \mid x \text{ is a bit}\}
\]
and therefore $|C \cap D| = 2$.
However
\[
B \cap D = \{ 00000000 \}
\]
i.e., $|B \cap D| = 1$.
We also have 
\[
B \cap C \cap D = \{00000000\}
\]
and therefore $|B \cap C \cap D| = 1$.

Therefore
\begin{align*}
|B \cup C \cup D|
&= |B| + |C| + |D| - (|B \cap C| + |B \cap D| + |C \cap D|) + |B \cap C \cap D|
\\
&= 4 + 4 + 4 - (2 + 2 + 1) + 1 \\
&= 8
\end{align*}

Therefore 
\[
|U| - |A| = 256 - |B \cup C \cup D| = 256 - 8 = 248
\]
Hence the number of bit strings of length 8 that do not contain 
6 consecutive 0s is 248.

ANSWER:
\answerbox{248}

\qed


\textsc{Question.}
What is instead of 6 consecutive 0s, I change the problem to 5 consecutive 0s?

\textsc{Question.}
What if I want to count
bit strings of length $n$ that does not contain
$k$ consecutives 0s? 
