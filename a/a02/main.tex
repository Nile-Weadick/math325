
\newcommand\COURSE{math325}
\newcommand\ASSESSMENT{q080403}
\newcommand\ASSESSMENTTYPE{Quiz}
\newcommand\POINTS{\textwhite{xxx/xxx}}

\makeatletter
\DeclareOldFontCommand{\rm}{\normalfont\rmfamily}{\mathrm}
\DeclareOldFontCommand{\sf}{\normalfont\sffamily}{\mathsf}
\DeclareOldFontCommand{\tt}{\normalfont\ttfamily}{\mathtt}
\DeclareOldFontCommand{\bf}{\normalfont\bfseries}{\mathbf}
\DeclareOldFontCommand{\it}{\normalfont\itshape}{\mathit}
\DeclareOldFontCommand{\sl}{\normalfont\slshape}{\@nomath\sl}
\DeclareOldFontCommand{\sc}{\normalfont\scshape}{\@nomath\sc}
\makeatother

\input{myquizpreamble}
\input{yliow}
\input{\COURSE}
\textwidth=6in

\renewcommand\TITLE{\ASSESSMENTTYPE \ \ASSESSMENT}

\newcommand\topmattertwo{
\topmatter
\score \\ \\
Open \texttt{main.tex} and enter answers (look for
\texttt{answercode}, \texttt{answerbox}, \texttt{answerlong}).
Turn the page for detailed instructions.
To rebuild and view pdf, in bash shell execute \texttt{make}.
To build a gzip-tar file, in bash shell execute \texttt{make s} and
you'll get \texttt{submit.tar.gz}.
}

\newcommand\tf{T or F or M}
\newcommand\answerbox[1]{\textbox{#1}}
\newcommand\codebox[1]{\begin{console}#1\end{console}}

\usepackage{pifont}
\newcommand{\cmark}{\textred{\ding{51}}}
\newcommand{\xmark}{\textred{\ding{55}}}

\newcounter{qc}
\newcommand\nextq{
%\newpage
\addtocounter{qc}{1}
Q{\theqc}.
}

\DefineVerbatimEnvironment%
 {answercode}{Verbatim}
 {frame=single,fontsize=\footnotesize}

\newenvironment{largebox}[1]{%
 \boxparone{#1}
}
{}

\usepackage{environ}
\let\oldquote=\quote
\let\endoldquote=\endquote
\let\quote\relax
\let\endquote\relax

% ADDED 2021/09/09
\renewcommand\boxpar[1]{
 \[
  \framebox[\textwidth][c] {
   \parbox[]{\dimexpr\textwidth - 0.25cm} {#1}
  }
 \]
}

\NewEnviron{answerlong}%
  {\vspace{-1mm} \global\let\tmp\BODY\aftergroup\doboxpar}

\newcommand\doboxpar{%
  \let\quote=\oldquote
  \let\endquote=\endoldquote
  \boxpar{\tmp}
}

\newenvironment{mcq}[7]%
{% begin code
#1 \dotfill{#2}
 \begin{tightlist}
 \item[(A)] #3
 \item[(B)] #4
 \item[(C)] #5
 \item[(D)] #6
 \item[(E)] #7 
 \end{tightlist}
}%
{% end code
} 

\renewcommand\EMAIL{}
\newcommand\score{%
\vspace{-0.6in}
\begin{flushright}
Score: \answerbox{\POINTS}
\end{flushright}
\vspace{-0.4in}
\hspace{0.7in}\AUTHOR
\vspace{0.2in}
}

\newcommand\blankline{\mbox{}\\ }

\newcommand\ANSWER{\textsc{Answer:}\vspace{-2mm}}


\renewcommand\AUTHOR{nweadick1@cougars.ccis.edu} % CHANGE TO YOURS

\begin{document}
\topmattertwo

\textsc{Objectives}
\begin{enumerate}
\item Solve counting problems using principle of
inclusion-exclution.
\end{enumerate}
\vspace{1cm}

The following are practice problems for self-study:
\begin{enumerate}[nosep]
\li Rosen 8th edition, section 8.5: Odd numbered problems 1-23.
Solution of questions 1-11,13,15 are included.
\end{enumerate}
For this assignment the problems you need to solve are
\begin{enumerate}[nosep]
\li Rosen 8th edition, section 8.5: \underline{questions 12, 14, 16}.
\end{enumerate}
Explain your work completely using the solutions provided as
guide and examples on how to write math properly.

In \LaTeX\, math notation is enclosed by the \verb!$! symbols.
For instance \verb!$x = a_{1} + b^{2}$! gives you $x = a_{1} + b^{2}$.
For emphasis you can write also write it as
\verb!\[ x = a_{1} + b^{2} \]!
to center your math:
\[
x = a_{1} + b^{2}
\]
For binomial coefficients, \verb!$\binom{5}{2}$! will give you 
$\binom{5}{2}$.
You can also look at the solutions, find something that you can use,
copy-and-paste, and modify.

For more information about \LaTeX\, go to my
website
\href{http://bit.ly/yliow0}{http://bit.ly/yliow0},
click on \verb!Yes!
you are one of my students,
then look for
\href{https://drive.google.com/file/d/0BzjYrK0VFuMWZm5xV0kyR3J2Zm8/view?usp=sharing}{latex.pdf}.)
Even easier: ask questions in CCCS discord.

%------------------------------------------------------------------------------
\newpage
\nextq Rosen 8th edition, section 8.5, question 1.

\SOLUTION

The relevant principle of inclusion-exclusion states
\[
|A_1 \cup A_2| = |A_1| + |A_2| - |A_1 \cap A_2| 
\]
We are given
\begin{align*}
|A_1| &= 12 \\
|A_2| &= 18 
\end{align*}

(a)
From the principle of inclusion-exclusion, we have
\begin{align*}
|A_1 \cup A_2| = 12 + 18 - 0 = 30
\end{align*}
ANSWER:
\answerbox{30}


(b)
From the principle of inclusion-exclusion, we have
\begin{align*}
|A_1 \cup A_2| = 12 + 18 - 1 = 29
\end{align*}
ANSWER:
\answerbox{29}

(c)
From the principle of inclusion-exclusion, we have
\begin{align*}
|A_1 \cup A_2| = 12 + 18 - 6 = 24
\end{align*}
ANSWER:
\answerbox{24}

(d) Since $A_1\subseteq A_2$, we have $A_1 \cap A_2 = A_1$.
Therefore
from the principle of inclusion-exclusion, we have
\begin{align*}
|A_1 \cup A_2| = |A_1| + |A_2| - |A_1 \cap A_2| = |A_1| + |A_2| - |A_1| = 
|A_2| = 18
\end{align*}
ANSWER:
\answerbox{18}

\qed

%------------------------------------------------------------------------------
\newpage
\nextq Rosen 8th edition, section 8.5, question 2.

\SOLUTION

Let 
\begin{align*}
C &= \{\text{student who took calculus}\} \\
D &= \{\text{student who took discrete mathematics}\}
\end{align*}
We are given
\begin{align*}
|C| &= 345 \\
|D| &= 212 \\
|C \cap D| &= 188
\end{align*}
The required number is $|C \cup D|$.
By the principle of inclusion-exclusion,
\[
|C \cup D| = |C| + |D| - |C \cap D|
\]
Therefore
\[
|C \cup D| = 345 + 212 - 188 = 369.
\]
Hence the required number is 369.

ANSWER:
\answerbox{369}

\qed

%------------------------------------------------------------------------------
\newpage
\nextq Rosen 8th edition, section 8.5, question 3.

\SOLUTION

Let
\begin{align*}
U &= \text{set of households in the survey} \\
T &= \{x \in U \mid x \text{ has at least one TV} \} \\
P &= \{x \in U \mid x \text{ has telephone service} \} 
\end{align*}
We are given
\begin{align*}
\frac{|T|}{|U|} &= 0.96 \\
\frac{|P|}{|U|} &= 0.98 \\
\frac{|T \cap P|}{|U|} &= 0.95 \\
\end{align*}
The required number is $|U - (T \cup P)|/|U|$.
Note that 
\begin{align*}
\frac{|U - (T \cup P)|}{|U|}
= \frac{|U|}{|U|} - \frac{|T \cup P|}{|U|} = 1 - \frac{|T \cup P|}{|U|} 
\end{align*}
The principle of inclusion-exclusion states
\[
|T \cup P| = |T| + |P| - |T \cap P|
\]
Therefore
\[
\frac{|T \cup P|}{|U|} = \frac{|T|}{|U|} + \frac{|P|}{|U|} 
- \frac{|T \cap P|}{|U|} 
= 0.96 + 0.98 - 0.95 = 0.99
\]
Hence the required number is 
\[
\frac{|U - (T \cup P)|}{|U|}
= 1 - \frac{|T \cup P|}{|U|} 
= 1 - 0.99 = 0.01
\]

ANSWER:
\answerbox{0.01}

\qed



%------------------------------------------------------------------------------
\newpage
\nextq Rosen 8th edition, section 8.5, question 4.

\SOLUTION

Let
\begin{align*}
P &= \text{set of owners that will buy a printer} \\
S &= \text{set of owners that will buy at least one software package} \\
\end{align*}
We are given
\begin{align*}
|P| &= 650000 \\
|S| &= 1250000 \\
|P \cup S| &= 1450000\\
\end{align*}
The required number is $|P \cap S|$.

By the principle of inclusion-exclusion, 
\[
|P \cup S| = |P| + |S| - |P \cap S|
\]
Therefore
\[
1450000 = 650000 + 1250000 - |P \cap S|
\]
Hence
\[
|P \cap S| = 650000 + 1250000 - 1450000 = 450000  
\]
The required number is 450000.

ANSWER:
\answerbox{450000}

\qed

%------------------------------------------------------------------------------
\newpage
\nextq Rosen 8th edition, section 8.5, question 5.

\SOLUTION

(a) The principle of inclusion-exclusion states
\begin{align*}
|A_1 \cup A_2 \cup A_3|
&= |A_1| + |A_2| + |A_3| \\
& \hspace{0.5cm} 
- \left( |A_1 \cap A_2| + |A_1 \cap A_3| + |A_2 \cap A_3| \right) \\
& \hspace{0.5cm} + |A_1 \cap A_2 \cap A_3|
\end{align*}
If the sets are pairwise disjoint, then
\[
|A_1 \cap A_2| = |A_1 \cap A_3| = |A_2 \cap A_3| =
|A_1 \cap A_2 \cap A_3| = 0
\]
Therefore
\[
|A_1 \cup A_2 \cup A_3|
= 100 + 100 + 100 - (0 + 0 + 0) + 0 = 300
\]

ANSWER:
\answerbox{300}


(b)
In this case we are given
\begin{align*}
|A_1 \cap A_2| &= |A_1 \cap A_3| = |A_2 \cap A_3| = 50 \\
|A_1 \cap A_2 \cap A_3| &= 0 
\end{align*}
Therefore from the principle of inclusion-exclusion we obtain
\begin{align*}
|A_1 \cup A_2 \cup A_3|
= 100 + 100 + 100 - (50 + 50 + 50) + 0 = 150
\end{align*}

ANSWER:
\answerbox{150}


(c)
In this case we are given
\begin{align*}
|A_1 \cap A_2| &= |A_1 \cap A_3| = |A_2 \cap A_3| = 50 \\
|A_1 \cap A_2 \cap A_3| &= 25 
\end{align*}
Therefore from the principle of inclusion-exclusion we obtain
\begin{align*}
|A_1 \cup A_2 \cup A_3|
= 100 + 100 + 100 - (50 + 50 + 50) + 25 = 175
\end{align*}

ANSWER:
\answerbox{175}


(d) If all sets are equal, then $A_1 \cup A_2 \cup A_3 = A_1$.
Hence
\[
|A_1 \cup A_2 \cup A_3| = |A_1| = 100
\]

ANSWER:
\answerbox{100}

\qed

%------------------------------------------------------------------------------
\newpage
\nextq Rosen 8th edition, section 8.5, question 6.

\SOLUTION

We are given
\begin{align*}
|A_1| &= 100 \\
|A_2| &= 1000 \\
|A_3| &= 10000
\end{align*}
The principle of inclusion-exclusion states
\begin{align*}
|A_1 \cup A_2 \cup A_3| &= |A_1| + |A_2| + |A_3| \\ 
                       & \hspace{0.5cm} 
                         - \left( |A_1 \cap A_2| + |A_1 \cap A_3| + |A_2 \cap A_3| \right) \\
                       & \hspace{0.5cm} + |A_1 \cap A_2 \cap A_3|
\end{align*}


(a)
With the assumption $A_1 \subseteq A_2 \subseteq A_3$, we have
\begin{align*}
A_1 \cap A_2          &= A_1 \\
A_1 \cap A_3          &= A_1 \\
A_2 \cap A_3          &= A_2 \\
A_1 \cap A_2 \cap A_3 &= A_1
\end{align*}
From the principle of inclusion-exclusion, we obtain
\begin{align*}
|A_1 \cup A_2 \cup A_3|
&= |A_1| + |A_2| + |A_3| - \left( |A_1| + |A_1| + |A_2| \right) + |A_1| \\
&= 100 + 1000 + 10000 - (100 + 100 + 1000) + 100 \\
&= 10000
\end{align*}
ANSWER:
\answerbox{10000}


(b) If the sets of pairwise disjoint,
\[
|A_1 \cap A_2| = |A_1 \cap A_3| = |A_2 \cap A_3| = |A_1 \cap A_2 \cap A_3| = 0
\]
From the principle of inclusion-exclusion, we obtain
\begin{align*}
|A_1 \cup A_2 \cup A_3| &= 100 + 1000 + 10000 - (0 + 0 + 0) + 0 \\
                        &= 11100
\end{align*}
ANSWER:
\answerbox{11100}


(c) We have
\begin{align*}
|A_1 \cap A_2|          &= |A_1 \cap A_3| = |A_2 \cap A_3| = 2 \\ 
|A_1 \cap A_2 \cap A_3| &= 1
\end{align*}
From the principle of inclusion-exclusion, we obtain
\begin{align*}
|A_1 \cup A_2 \cup A_3| &= 100 + 1000 + 10000 - (2 + 2 + 2) + 1 \\
                        &= 11095
\end{align*}
ANSWER:
\answerbox{11095}

\qed


%------------------------------------------------------------------------------
\newpage
\nextq Rosen 8th edition, section 8.5, question 7.

\SOLUTION

Let
\begin{align*}
U &= \text{set of all computer science students in the school } \\ 
C &= \{ x \in U \mid x \text{ took C}\} \\ 
L &= \{ x \in U \mid x \text{ took Linux}\} \\ 
J &= \{ x \in U \mid x \text{ took Java}\} \\ 
\end{align*}
We are given
\begin{align*}
|U| &= 2504 \\
|C| &= 345 \\
|L| &= 999 \\
|J| &= 1876 \\
|C \cap L| &= 231 \\
|J \cap L| &= 876 \\
|C \cap J| &= 290 \\
|C \cap J \cap L| &= 189 \\
\end{align*}
The required number is $|U - (C \cup J \cup L)|$.
By the principle of inclusion-exclusion:
\[
|C \cup J \cup L| 
= |C| + |J| + |L| - (|C \cap J| + |C \cap L| + |J \cap L|) 
+ |C \cap J \cap L| 
\] 
Therefore
\[
|C \cup J \cup L| 
= 345 + 999 + 1876 - (231 + 876 + 290) + 189 = 2012
\] 
Hence the required number is
\[ 
|U - (C \cup J \cup L)| 
= |U| - |C \cup J \cup L| = 2504 - 2012 = 492 
\]

ANSWER:
\answerbox{492}

\qed



%------------------------------------------------------------------------------
\newpage
\nextq Rosen 8th edition, section 8.5, question 8.

\SOLUTION

Let
\begin{align*}
U &= \text{set of college students in the survey} \\
B &= \{ x \in U \mid x \text{ likes broccoli} \} \\
C &= \{ x \in U \mid x \text{ likes cauliflower}  \} \\
S &= \{ x \in U \mid x \text{ likes brussel sprouts} \}
\end{align*}
We are given 
\begin{align*}
|U| &= 270 \\
|B| &= 94 \\
|C| &= 58 \\
|S| &= 64 \\
|B \cap C| &= 22 \\
|B \cap S| &= 26 \\
|C \cap S| &= 28 \\
|B \cap C \cap S| &= 14
\end{align*}
The required number is $|U - (B \cup C \cup S)|$.
By the principle of inclusion-exclusion:
\[
|B \cup C \cup S| 
= |B| + |C| + |S| - (|B \cap C| + |B \cap S| + |C \cap S|) 
+ |B \cap C \cap S| 
\] 
Therefore
\[
|B \cup C \cup S| 
= 94 + 58 + 64 - (22 + 26 + 28) + 14 = 154
\] 
Hence the required number is
\[ 
|U - (B \cup C \cup S)| 
= |U| - |B \cup C \cup S| = 170 - 154 = 16 
\]

ANSWER:
\answerbox{16}

\qed


%------------------------------------------------------------------------------
\newpage
\nextq Rosen 8th edition, section 8.5, question 9.

\SOLUTION


Let 
\begin{align*}
U &= \text{set of students in the college} \\
C &= \{ x \in U \mid x \text{ is in calculus} \} \\
D &= \{ x \in U \mid x \text{ is in discrete mathematics} \} \\
S &= \{ x \in U \mid x \text{ is in data structures} \} \\
P &= \{ x \in U \mid x \text{ is in programming languages} \} \\
\end{align*}
We are given
\begin{align*}
|C| &= 507 \\
|D| &= 292 \\
|S| &= 312 \\
|P| &= 344 \\
|C \cap S| &= 14 \\
|C \cap P| &= 213 \\
|D \cap S| &= 211 \\
|D \cap P| &= 43 \\
|C \cap D| &= 0 \\
|S \cap P| &= 0 \\
\end{align*}
Note that the intersection of any 3 of the sets or all 4 sets must be empty.
The required number is $|C \cup D \cup S \cup S \cup P|$.
By the principle of inclusion-exclusion,
\begin{align*}
|C\cup D \cup S \cup P|
&= |C| + |D| + |S| + |P| \\
&\hspace{0.5cm} - (|C \cap D| + |C \cap S| + |C \cap P| + |D \cap S| + |D \cap P| + |S \cap P|) \\
&\hspace{0.5cm} + (|C \cap D \cap S| + |C \cap D \cap P| + |C \cap S \cap P| + |D \cap S \cap P|) \\
&\hspace{0.5cm} - |C \cap D \cap S \cap P| \\
&= 507 + 292 + 312 + 344 - (14 + 213 + 211 + 43) + (0 + 0 + 0 + 0) - 0 \\
&= 974
\end{align*}

ANSWER:
\answerbox{974}

\qed



%------------------------------------------------------------------------------
\newpage
\nextq Rosen 8th edition, section 8.5, question 10.

\SOLUTION

Let 
\begin{align*}
U &= \{ 1, 2, 3, \ldots, 1000 \} \\
A_1 &= \{x \in U \mid x \text{ is divisible by 5} \} \\
A_2 &= \{x \in U \mid x \text{ is divisible by 7} \}
\end{align*}
The required number is $|U - (A_1 \cup A_2)|$.
Note that 
\begin{align*}
|U| &= 1000 \\
|A_1| &= \floor{ \frac{1000}{5} } = 200\\
|A_2| &= \floor{ \frac{1000}{7} } = 142\\
|A_1 \cap A_2| &= \{x \in U \mid x \text{ is divisible by 35} \} = \floor{ \frac{1000}{35} } = 28 
\end{align*}
By the principle of inclusion-exclusion, we have
\begin{align*}
|A_1 \cup A_2| 
&= |A_1| + |A_2| - |A_1 \cap A_2| \\
&= 200 + 142 - 28 \\
&= 314
\end{align*}
Therefore the required number is $|U - (A_1 \cup A_2)| = 1000 - 314 
= 686$.

ANSWER:
\answerbox{686}

\qed
\\

\textsc{Note.}
Of course since the numbers are so small, you should quickly
check this with some programming language:
\begin{console}[fontsize=\footnotesize]
>>> count = 0
>>> for i in range(1, 1001):
...     if i % 5 is not 0 and i % 7 is not 0:
...         count += 1
... 
>>> print(count)
686
>>> 
\end{console}

\textsc{Note.}
Note that for an integer $n$, $n$ is divisible by $k$ and $\ell$
iff $n$ is divisible by 
\[
\frac{k \ell}{\gcd(k, \ell)}
\]


%------------------------------------------------------------------------------
\newpage
\nextq Rosen 8th edition, section 8.5, question 11.

\SOLUTION


Let
\begin{align*}
  U &= \{1, 2, 3, ..., 1000\} \\
  A_1 &= \{x \in U \mid x \text{ divisible by 3} \} \\
  A_2 &= \{x \in U \mid x \text{ divisible by 17} \} \\
  A_3 &= \{x \in U \mid x \text{ divisible by 35} \} 
\end{align*}
The required number of $|U - (A_1 \cup A_2 \cup A_3)|$.
Note that
\begin{align*}
  |U| &= 1000 \\
  |A_1| &= \floor{ \frac{1000}{3} } = 333 \\
  |A_2| &= \floor{ \frac{1000}{17} } = 58 \\
  |A_3| &= \floor{ \frac{1000}{35} } = 28 \\
  |A_1 \cap A_2| &= \{x \in U \mid x \text{ divisible by 3, 17}\} = \floor{\frac{1000}{3 \cdot 17}} = 19 \\
  |A_1 \cap A_3| &= \{x \in U \mid x \text{ divisible by 3, 35}\} = \floor{\frac{1000}{3 \cdot 35}} = 9 \\
  |A_2 \cap A_3| &= \{x \in U \mid x \text{ divisible by 17, 35}\} = \floor{\frac{1000}{17 \cdot 35}} = 1 \\
  |A_1 \cap A_2 \cap A_3| &= \floor{ x \in U \mid x \text{ divisible by 3,17,35 }\} = \frac{1000}{3 \cdot 17 \cdot 35}} = 0 
\end{align*}
By the principle of inclusion-exclusion, we have
\begin{align*}
  | A_1 \cup A_2 \cup A_3 |
  &= (|A_1| + |A_2| + |A_3|) - (|A_1 \cap A_2| + |A_1 \cap A_3| + |A_2 \cap A_3|) + |A_1 \cap A_2 \cap A_3| \\
  &= (333 + 58 + 28) - (19 + 9 + 1) + 0 \\
  &= 390
\end{align*}
Therefore
\begin{align*}
  | U -  A_1 \cup A_2 \cup A_3 |
  = |U| -  |A_1 \cup A_2 \cup A_3|
  = 1000 - 390
  = 610
\end{align*}

  
ANSWER:
\answerbox{610}

\qed

\textsc{Note.}
Here's a quick check:
\begin{console}[fontsize=\footnotesize]
>>> count = 0
>>> for x in range(1, 1001):
...     if x % 3 != 0 and x % 17 != 0 and x % 35 != 0:
...         count += 1
... 
>>> print(count)
610
\end{console}


%------------------------------------------------------------------------------
\newpage
\nextq Rosen 8th edition, section 8.5, question 12.

\SOLUTION
Find the number of positive integers not exceeding
10,000 that are not divisible by 3, 4, 7, or 11.

Let there be four sets A,B,C,D.

$|A| = \floor{\frac{10000}{3}}$, $|B| = \floor{\frac{10000}{4}}$,
$|C| = \floor{\frac{10000}{7}}$,$|D| = \floor{\frac{10000}{11}}$,

$|A \cap B| = \floor{\frac{10000}{3 \cdot 4}}$,$|A \cap C| = \floor{\frac{10000}{3 \cdot 7}}$,$|A \cap D| = \floor{\frac{10000}{3 \cdot 11}}$,
$|B \cap C| = \floor{\frac{10000}{4 \cdot 7}}$,$|B \cap D| = \floor{\frac{10000}{4 \cdot 11}}$,$|C \cap D| = \floor{\frac{10000}{7 \cdot 11}}$,

$|A \cap B \cap C| = \floor{\frac{10000}{3 \cdot 4 \cdot 7}}$,$|A \cap B \cap D| = \floor{\frac{10000}{3 \cdot 4 \cdot 11}}$,$|B \cap C \cap D| = \floor{\frac{10000}{4 \cdot 7 \cdot 11}}$,

$|A \cap B \cap C \cap D| = \floor{\frac{10000}{3 \cdot 4 \cdot 7 \cdot 11}}$

Now, PIE.

$|A \cup B \cup C \cup D| = |A| + |B| + |C| + |D|$

$- (|A \cap B| + |A \cap C| + |A \cap D| + |B \cap C| + |B \cap D| + |C \cap D|$

$+ (|A \cap B \cap C| + |A \cap B \cap D| + |B \cap C \cap D|$

$- |A \cap B \cap C \cap D|$

$(3333 + 2500 + 1428 + 909) = 8170$

$-(833 + 476 + 303 + 357 + 227 + 129) = 2325$

$+(119 + 312 + 32) = 463$

$-(10)$

$10000 - 6298 = 3702$
\answerbox{3702}
\input{08-05-12.tex}

%------------------------------------------------------------------------------
\newpage
\nextq Rosen 8th edition, section 8.5, question 13.

\SOLUTION

Let 
\begin{align*}
U &= \{1, 2, 3, ..., 100\} \\
A &= \{x \in U \mid x \text{ is odd}\} \\
B &= \{x \in U \mid x \text{ is square}\} \\
\end{align*}
The required number is $|A \cup B|$.
The principle of inclusion-exclusion states
\[
|A \cup B| = |A| + |B| - |A \cup B|
\]

The number of integers in $U$ which are even is 
$\floor{\frac{100}{2}} = 50$.
Therefore 
\[
|A| = |U| - 50 = 50
\]
The squares in $B$ are $1^2, 2^2, 3^2, 4^2, 5^2, 6^2, 7^2, 8^2, 9^2, 10^2$.
Therefore $|B| = 10$.
(In general, the number of squares in 1, 2, 3, ..., $n$ is
$\floor{\sqrt{n}}$.)
Furthmoremore, the integers in 
$1^2$, $2^2$, $3^3$, $4^2$, $5^2$, $6^2$, $7^2$, $8^2$
$9^2$, $10^2$ which are odd are $1^1, 3^2, 5^2, 7^2, 9^2$.
Therefore 
\[
|A \cap B| = 5
\]

Therefore
\[
|A \cup B|
= |A| + |B| - |A \cap B| = 50 + 10 - 5 = 55
\]
Hence the required number is 55.

ANSWER:
\answerbox{55}

\qed



\textsc{Note.} This is easy to check with a program:
\begin{console}
import math

def is_square(n):
    x = int(math.sqrt(n))
    # test x + 1 in case of rounding errors
    return x ** 2 == n or (x + 1) ** 2 == n

count = 0
for i in range(1, 101):
    if i % 2 == 1 or is_square(i):
        count += 1

print count
\end{console}
The output is 55.

\textsc{Note.} Change the 100 to $N^2$ where $N$ is a positive integer.
Solve the problem.
Substitute 100 for $N$ in your solution and you should get 55.

\textsc{Question.}
Let $k > 0$ be an integer.
Why is the number of $k$--powers in $\{1, 2, 3, ..., N\}$
given $\floor{N^{1/k}}$?


%------------------------------------------------------------------------------
\newpage
\nextq Rosen 8th edition, section 8.5, question 14.

\SOLUTION

Find the number of positive integers not exceeding 1000
that are either the square or the cube of an integer.

Let there be two sets A,B

$|A| = \sqrt{1000}$, $|B| = \sqrt[3]{1000}$

$|A| = \floor{\sqrt{1000}} = 31$

$|B| = \floor{\sqrt[3]{1000}} = 10$

$|A \cap B| = \floor{\sqrt[3]{\sqrt{1000}}} = 3$

Now PIE,

$|A \cup B| = |A| + |B| - |A \cap B|$

$31 + 10 - 3 = 38$

\answerbox{38}

\input{08-05-14.tex}

%------------------------------------------------------------------------------
\newpage
\nextq Rosen 8th edition, section 8.5, question 15.

\SOLUTION

Let 
\begin{align*}
U &= \{ \text{bit string of length 8} \} \\
A & = \{x \in U \mid x \text{ contains 6 consecutive 0s} \}
\end{align*}
The required number is $|U| - |A|$.

A string in $A$ can be one of the three forms:
It is either
\[
B = \{000000 x y \mid x, y \text{ are bits}\}
\]
or
\[
C = \{x 000000 y \mid x, y \text{ are bits}\}
\]
or
\[
D = \{x y 000000 \mid x, y \text{ are bits}\}
\]
Therefore $|A| = |B \cup C \cup D|$.
By the principle of inclusion-exclusion, 
\[
|B \cup C \cup D|
= |B| + |C| + |D| - (|B \cap C| + |B \cap D| + |C \cap D|) + |B \cap C \cap D|
\]

By the multiplcation principle, 
$|B| = 2 \cdot 2 = 4$ since there are two ways to choose
$x$ and two ways to choose $y$. 
Likewise $|C| = |D| = 4$.

Now note that 
\[
B \cap C = \{000000y \mid y \text{ is a bit}\}
\]
Therefore $|B \cap C| = 2$.
Also, 
\[
C \cap D = \{x 0000000 \mid x \text{ is a bit}\}
\]
and therefore $|C \cap D| = 2$.
However
\[
B \cap D = \{ 00000000 \}
\]
i.e., $|B \cap D| = 1$.
We also have 
\[
B \cap C \cap D = \{00000000\}
\]
and therefore $|B \cap C \cap D| = 1$.

Therefore
\begin{align*}
|B \cup C \cup D|
&= |B| + |C| + |D| - (|B \cap C| + |B \cap D| + |C \cap D|) + |B \cap C \cap D|
\\
&= 4 + 4 + 4 - (2 + 2 + 1) + 1 \\
&= 8
\end{align*}

Therefore 
\[
|U| - |A| = 256 - |B \cup C \cup D| = 256 - 8 = 248
\]
Hence the number of bit strings of length 8 that do not contain 
6 consecutive 0s is 248.

ANSWER:
\answerbox{248}

\qed


\textsc{Question.}
What is instead of 6 consecutive 0s, I change the problem to 5 consecutive 0s?

\textsc{Question.}
What if I want to count
bit strings of length $n$ that does not contain
$k$ consecutives 0s? 


%------------------------------------------------------------------------------
\newpage
\nextq Rosen 8th edition, section 8.5, question 16.

\SOLUTION
How many permutations of the 26 letters of the English
alphabet do not contain any of the strings fish,rat or bird?

There are a total of 26! strings in all.

There are 23! strings that contain fish.

There are 24! strings that contain rat.

There are 23! strings that contain bird.

There are 21! strings that contain both fish and rat.

There no strings that contain rat and bird, due to the "r". 

Now PIE,

$26! - (23! + 24! + 23!) + (21!) $

\answerbox{$26! - (23! + 24! + 23!) + (21!)$}


\input{08-05-16.tex}

%------------------------------------------------------------------------------

\newpage

\textsc{Instructions}

In \verb!main.tex! change the email address in
\begin{console}
\renewcommand\AUTHOR{jdoe5@cougars.ccis.edu} 
\end{console}
yours.
In the bash shell, execute \lq\lq \verb!make!" to recompile \verb!main.pdf!.
Execute \lq\lq \verb!make v!" to view \verb!main.pdf!.
Execute \lq\lq \verb!make s!" to create \verb!submit.tar.gz! for submission.

For each question, you'll see boxes for you to fill.
You write your answers in \verb!main.tex! file.
For small boxes, if you see
\begin{console}[frame=single=single,fontsize=\small]
1 + 1 = \answerbox{}.
\end{console}
you do this:
\begin{console}[frame=single=single,fontsize=\small]
1 + 1 = \answerbox{2}.
\end{console}
\verb!answerbox! will also appear in
\lq\lq true/false" and \lq\lq multiple-choice"
questions.

For longer answers that needs typewriter font, if you see
\begin{console}[frame=single=single, fontsize=\small]
Write a C++ statement that declares an integer variable name x.
\begin{answercode}
\end{answercode}
\end{console}
you do this:
\begin{console}[frame=single=single, fontsize=\small]
Write a C++ statement that declares an integer variable name x.
\begin{answercode}
int x;
\end{answercode}
\end{console}
\verb!answercode! will appear in questions asking for
code, algorithm, and program output.
In this case, indentation and spacing is significant.
For program output, I do look at spaces and newlines.

For long answers (not in typewriter font) if you see
\begin{console}[frame=single=single, fontsize=\small]
What is the color of the sky?
\begin{answerlong}
\end{answerlong}
\end{console}
you can write
\begin{console}[frame=single=single, fontsize=\small]
What is the color of the sky?
\begin{answerlong}
The color of the sky is blue.
\end{answerlong}
\end{console}
For students beyond 245: You can put \LaTeX\ commands in
\verb!answerbox! and 
\verb!answerlong!.

A question that begins with \lq\lq T or F or M"
requires you to identify whether it is true or
false, or meaningless.
\lq\lq Meaningless" means something's wrong with the statement and
it is not well-defined.
Something like \lq\lq $1 +_2$" or \lq\lq $\{2\}^{\{3\}}$" is not
well-defined.
Therefore a question such as
\lq\lq Is $42 = 1 +_2$ true or false?" or
\lq\lq Is $42 = \{2\}^{\{3\}}$ true or false?"
does not make sense.
\lq\lq Is $P(42) = \{42\}$ true or false?" is meaningless because $P(X)$
is only defined if $X$ is a set.
For \lq\lq Is 1 + 2 + 3 true or false?", \lq\lq 1 + 2 + 3" is well--defined but
as a
\lq\lq numerical expression", not as a \lq\lq proposition", i.e.,
it cannot be true or false.
Therefore \lq\lq Is 1 + 2 + 3 true or false?" is also not a well-defined
question.

When writing results of computations, make sure it's simplified.
For instance write $2$ instead of $1 + 1$.
When you write down sets,
if the answer is $\{1\}$, I do not
want to see $\{1, 1\}$.

When writing a counterexample, always write the simplest.

Here are some examples (see \verb!instructions.tex! for details):

\begin{enumerate}

 \item \tf: 1 + 1 = 2 \dotfill\answerbox{T}
 
 \item \tf: 1 + 1 = 3 \dotfill\answerbox{F}
 
 \item \tf: $1 +^2 =$ \dotfill\answerbox{M}
 
 \item $1 + 2 =$ \answerbox{3}
 
 \item Write a C++ statement to declare an integer variable named
 \verb!x!.
 \begin{answercode}
int x;
 \end{answercode}

 \item Solve $x^2 - 1 = 0$.
 \begin{answerlong}
 Since $x^2 - 1 = (x-1)(x+1)$, $x^2 - 1 = 0$ implies $(x-1)(x+1)=0$.
 Therefore $x - 1 = 0$ or $x = -1$.
 Hence $x = 1$ or $x = -1$.
 \end{answerlong}

 \item
 \begin{mcq}
 {Which is true?}{\answerbox{C}}
 {$1+1=0$}
 {$1+1=1$}
 {$1+1=2$}
 {$1+1=3$}
 {$1+1=4$}
 \end{mcq}


\end{enumerate}

\end{document}
