\newcommand\COURSE{math325}
\newcommand\ASSESSMENT{q080403}
\newcommand\ASSESSMENTTYPE{Quiz}
\newcommand\POINTS{\textwhite{xxx/xxx}}

\makeatletter
\DeclareOldFontCommand{\rm}{\normalfont\rmfamily}{\mathrm}
\DeclareOldFontCommand{\sf}{\normalfont\sffamily}{\mathsf}
\DeclareOldFontCommand{\tt}{\normalfont\ttfamily}{\mathtt}
\DeclareOldFontCommand{\bf}{\normalfont\bfseries}{\mathbf}
\DeclareOldFontCommand{\it}{\normalfont\itshape}{\mathit}
\DeclareOldFontCommand{\sl}{\normalfont\slshape}{\@nomath\sl}
\DeclareOldFontCommand{\sc}{\normalfont\scshape}{\@nomath\sc}
\makeatother

\input{myquizpreamble}
\input{yliow}
\input{\COURSE}
\textwidth=6in

\renewcommand\TITLE{\ASSESSMENTTYPE \ \ASSESSMENT}

\newcommand\topmattertwo{
\topmatter
\score \\ \\
Open \texttt{main.tex} and enter answers (look for
\texttt{answercode}, \texttt{answerbox}, \texttt{answerlong}).
Turn the page for detailed instructions.
To rebuild and view pdf, in bash shell execute \texttt{make}.
To build a gzip-tar file, in bash shell execute \texttt{make s} and
you'll get \texttt{submit.tar.gz}.
}

\newcommand\tf{T or F or M}
\newcommand\answerbox[1]{\textbox{#1}}
\newcommand\codebox[1]{\begin{console}#1\end{console}}

\usepackage{pifont}
\newcommand{\cmark}{\textred{\ding{51}}}
\newcommand{\xmark}{\textred{\ding{55}}}

\newcounter{qc}
\newcommand\nextq{
%\newpage
\addtocounter{qc}{1}
Q{\theqc}.
}

\DefineVerbatimEnvironment%
 {answercode}{Verbatim}
 {frame=single,fontsize=\footnotesize}

\newenvironment{largebox}[1]{%
 \boxparone{#1}
}
{}

\usepackage{environ}
\let\oldquote=\quote
\let\endoldquote=\endquote
\let\quote\relax
\let\endquote\relax

% ADDED 2021/09/09
\renewcommand\boxpar[1]{
 \[
  \framebox[\textwidth][c] {
   \parbox[]{\dimexpr\textwidth - 0.25cm} {#1}
  }
 \]
}

\NewEnviron{answerlong}%
  {\vspace{-1mm} \global\let\tmp\BODY\aftergroup\doboxpar}

\newcommand\doboxpar{%
  \let\quote=\oldquote
  \let\endquote=\endoldquote
  \boxpar{\tmp}
}

\newenvironment{mcq}[7]%
{% begin code
#1 \dotfill{#2}
 \begin{tightlist}
 \item[(A)] #3
 \item[(B)] #4
 \item[(C)] #5
 \item[(D)] #6
 \item[(E)] #7 
 \end{tightlist}
}%
{% end code
} 

\renewcommand\EMAIL{}
\newcommand\score{%
\vspace{-0.6in}
\begin{flushright}
Score: \answerbox{\POINTS}
\end{flushright}
\vspace{-0.4in}
\hspace{0.7in}\AUTHOR
\vspace{0.2in}
}

\newcommand\blankline{\mbox{}\\ }

\newcommand\ANSWER{\textsc{Answer:}\vspace{-2mm}}


\renewcommand\AUTHOR{nweadick1@cougars.ccis.edu} % CHANGE TO YOURS

\begin{document}
\topmattertwo

\textsc{Objectives}
\begin{enumerate}
\item Compute recurrence relation.
\end{enumerate}
\vspace{1cm}



The following are practice problems for self-study:
\begin{enumerate}[nosep]
\li Rosen 8th edition, section 8.1: Odd numbered problems 1-31 except 15, 17,
23, 29.
\end{enumerate}
Some solutions are provided.
For this assignment the problems you need to solve are
\begin{enumerate}[nosep]
\li Rosen 8th edition, section 8.1: \underline{questions 4, 8, 12, 24}.
\end{enumerate}
Explain your work completely using the solutions provided as
guide and examples on how to write math properly.

In \LaTeX\, math notation is enclosed by the \verb!$! symbols.
For instance \verb!$x = a_{1} + b^{2}$! gives you $x = a_{1} + b^{2}$.
For emphasis you can write also write it as
\verb!\[ x = a_{1} + b^{2} \]!
to center your math:
\[
x = a_{1} + b^{2}
\]
For binomial coefficients, \verb!$\binom{5}{2}$! will give you 
$\binom{5}{2}$.
You can also look at the solutions, find something that you can use,
copy-and-paste, and modify.

For more information about \LaTeX\, go to my
website
\href{http://bit.ly/yliow0}{http://bit.ly/yliow0},
click on \verb!Yes!
you are one of my students,
then look for
\href{https://drive.google.com/file/d/0BzjYrK0VFuMWZm5xV0kyR3J2Zm8/view?usp=sharing}{latex.pdf}.)
Even easier: ask questions in CCCS discord.





Draw a box around your final answer if the question required
an explicit answer (i.e., if it's not a proof question).

%------------------------------------------------------------------------------
\newpage
\nextq Rosen 8th edition, section 8.1, question 1.

\SOLUTION

Let $T_n$ be the number of moves to solve the $n$--disks tower of Hanoi
problem.
The recurrence relation is
\[
T_n = 2 T_{n-1} + 1 \tag{*}
\]
with base case of $T_1 = 1$.

Let $P(n)$ be the statement
\[
P(n) = \biggl( T_n = 2^n - 1 \biggr)
\]
for $n \geq 1$.
We will prove that $P(n)$ holds for all $n \geq 1$ using
induction.

\text{Base case.}
For $n = 1$, $T_1 = 1$
and $2^1 - 1 = 2 - 1 = 1$.
Hence $T_1 = 2^1 - 1$. Therefore $P(1)$ holds.

\text{Inductive case.}
Suppose $P(n)$ holds, i.e., 
\[
T^n = 2^n - 1
\]
We want to show that $P(n + 1)$ holds as well, i.e., we want to show
\[
T^{n + 1} = 2^{n + 1} - 1
\]
From the recurrence relation $(*)$,
\[
T_{n + 1} = 2T_n + 1
\]
Therefore
\[
T_{n + 1} = 2(2^n - 1) + 1 = 2^{n + 1} - 2 + 1 = 2^{n+1} - 1
\]
and hence $P(n + 1)$ holds.

Therefore by induction, $P(n)$ holds for all $n \geq 1$, i.e.,
\[
T_n = 2^n - 1
\]
for all $n \geq 1$.
\qed





%------------------------------------------------------------------------------
\newpage
\nextq Rosen 8th edition, section 8.1, question 2.

\SOLUTION

Let $a_n$ be the number of permutations on $n$ distinct symbols.

The task $T$ of forming a permutation is
the same as the task $T_1$ of writing down the first symbol
following by the task $T_2$ of writing down $n - 1$ symbols distinct
from the symbol written down by task $T_1$.
By the multiplication principle,
the number of ways to perform $T$ is the number of ways to perform
$T_1$ multiplied by the number of ways to perform $T_2$, i.e.,
\[
a_n = n \cdot a_{n - 1}
\]

Clearly $a_1 = 1$.
To compute a closed form for $a_n$, note that
\begin{align*}
  a_n
  &= n a_{n - 1} \\
  &= n (n - 1) a_{n - 2} \\
  &= n (n - 2) a_{n - 3} \\
  &= ... \\
  &= n (n - 2)(n - 3)\cdots(n - (k - 1)) a_{n - k} \\
\end{align*}
When $n - k = 1$, we have $k = n - 1$ and 
\begin{align*}
  a_n
  &= n (n - 2)(n - 3)\cdots 2 \cdot a_1 \\
  &= n (n - 2)(n - 3)\cdots 2 \cdot 1 \\
\end{align*}

Hence
\[
a_n = n!
\]
for $n \geq 1$.
\qed


%------------------------------------------------------------------------------
\newpage
\nextq Rosen 8th edition, section 8.1, question 3.

\SOLUTION


(a)
Let the one dollor coin, one dollar bill, five dollar fill be denoted by
$a=1, b=1, c=1$.
A deposit is a sequence of $a,b,c$ values such that the sum is $n$.
For instance the deposit
\[
a b a c
\]
is a deposit for $a + b + a + c = 1 + 1 + 1 + 5 = 8$.

A deposit for $n$ dollar is a sequence of the form
\[
\begin{cases}
a p \\
b p' \\
c p'' \\
\end{cases}
\]
where
$p$ is a pattern for the deposit of $n - 1$ dollars,
$p'$ is a pattern for the deposit of $n - 1$ dollars,
$p''$ is a pattern for the deposit of $n - 5$ dollars,
Hence if $a_n$ denotes the number of ways to deposit $n$ dollars,
we have the recurrence relation
\[
a_n =  a_{n-1} + a_{n - 1} + a_{n - 5} = 2 a_{n - 1} + a_{n - 5}
\]
Clearly
\begin{enumerate}[nosep]
\li $a_1 = 2$: the patterns are $a$ and $b$
\li $a_2 = 4$: the patterns are $aa, ab, ba, bb$
\li $a_3 = 8$: the patterns are $aaa, aab, ..., bbb$, i.e., sequence of length 3 using two symbols
\li $a_4 = 16$: the patterns are $aaaa, aaab, ..., bbbb$ i.e., sequence of length 4 using two symbols
\li $a_5 = 33$: the patterns are $aaaaa, aaaab, ..., bbbbb, c$, i.e., sequence of length 5 using two symbols and $c$
\end{enumerate}

Hence we have
\[
a_n =
\begin{cases}
2 a_{n-1} + a_{n - 5} & \text{ if } n \geq 6 \\
2^n                 & \text{ if } 1 \leq n \geq 4 \\
33                  & \text{ if } n = 5 \\

\end{cases}
\]

(b)
\begin{align*}
a_6 &= 2 a_5 + a_1 = 2 \cdot 33 + 2 = 68 \\
a_7 &= 2 a_6 + a_2 = 2 \cdot 68 + 4 = 72 \\
a_8 &= 2 a_7 + a_3 = 2 \cdot 72 + 8 = 152 \\
a_9 &= 2 a_8 + a_4 = 2 \cdot 152 + 16 = 320 \\
a_{10} &= 2 a_9 + a_5 = 2 \cdot 320 + 33 = 673
\end{align*}
\ANSWER \answerbox{673}

\qed


%------------------------------------------------------------------------------
\newpage
\nextq Rosen 8th edition, section 8.1, question 4.
A country uses as currency coins with values of 1 peso,
2 pesos, 5 pesos, and 10 pesos and bills with values of
5 pesos, 10 pesos, 20 pesos, 50 pesos, and 100 pesos.
Find a recurrence relation for the number of ways to pay
a bill of n pesos if the order in which the coins and bills
are paid matters.
\SOLUTION

\input{08-01-04.tex}

%------------------------------------------------------------------------------
\newpage
\nextq Rosen 8th edition, section 8.1, question 8.

\SOLUTION

\input{08-01-08.tex}

%------------------------------------------------------------------------------
\newpage
\nextq Rosen 8th edition, section 8.1, question 12.

\SOLUTION

\input{08-01-12.tex}

%------------------------------------------------------------------------------
\newpage
\nextq Rosen 8th edition, section 8.1, question 24.

\SOLUTION

\input{08-01-24.tex}

%------------------------------------------------------------------------------
\newpage

\textsc{Instructions}

In \verb!main.tex! change the email address in
\begin{console}
\renewcommand\AUTHOR{jdoe5@cougars.ccis.edu} 
\end{console}
yours.
In the bash shell, execute \lq\lq \verb!make!" to recompile \verb!main.pdf!.
Execute \lq\lq \verb!make v!" to view \verb!main.pdf!.
Execute \lq\lq \verb!make s!" to create \verb!submit.tar.gz! for submission.

For each question, you'll see boxes for you to fill.
You write your answers in \verb!main.tex! file.
For small boxes, if you see
\begin{console}[frame=single=single,fontsize=\small]
1 + 1 = \answerbox{}.
\end{console}
you do this:
\begin{console}[frame=single=single,fontsize=\small]
1 + 1 = \answerbox{2}.
\end{console}
\verb!answerbox! will also appear in
\lq\lq true/false" and \lq\lq multiple-choice"
questions.

For longer answers that needs typewriter font, if you see
\begin{console}[frame=single=single, fontsize=\small]
Write a C++ statement that declares an integer variable name x.
\begin{answercode}
\end{answercode}
\end{console}
you do this:
\begin{console}[frame=single=single, fontsize=\small]
Write a C++ statement that declares an integer variable name x.
\begin{answercode}
int x;
\end{answercode}
\end{console}
\verb!answercode! will appear in questions asking for
code, algorithm, and program output.
In this case, indentation and spacing is significant.
For program output, I do look at spaces and newlines.

For long answers (not in typewriter font) if you see
\begin{console}[frame=single=single, fontsize=\small]
What is the color of the sky?
\begin{answerlong}
\end{answerlong}
\end{console}
you can write
\begin{console}[frame=single=single, fontsize=\small]
What is the color of the sky?
\begin{answerlong}
The color of the sky is blue.
\end{answerlong}
\end{console}
For students beyond 245: You can put \LaTeX\ commands in
\verb!answerbox! and 
\verb!answerlong!.

A question that begins with \lq\lq T or F or M"
requires you to identify whether it is true or
false, or meaningless.
\lq\lq Meaningless" means something's wrong with the statement and
it is not well-defined.
Something like \lq\lq $1 +_2$" or \lq\lq $\{2\}^{\{3\}}$" is not
well-defined.
Therefore a question such as
\lq\lq Is $42 = 1 +_2$ true or false?" or
\lq\lq Is $42 = \{2\}^{\{3\}}$ true or false?"
does not make sense.
\lq\lq Is $P(42) = \{42\}$ true or false?" is meaningless because $P(X)$
is only defined if $X$ is a set.
For \lq\lq Is 1 + 2 + 3 true or false?", \lq\lq 1 + 2 + 3" is well--defined but
as a
\lq\lq numerical expression", not as a \lq\lq proposition", i.e.,
it cannot be true or false.
Therefore \lq\lq Is 1 + 2 + 3 true or false?" is also not a well-defined
question.

When writing results of computations, make sure it's simplified.
For instance write $2$ instead of $1 + 1$.
When you write down sets,
if the answer is $\{1\}$, I do not
want to see $\{1, 1\}$.

When writing a counterexample, always write the simplest.

Here are some examples (see \verb!instructions.tex! for details):

\begin{enumerate}

 \item \tf: 1 + 1 = 2 \dotfill\answerbox{T}
 
 \item \tf: 1 + 1 = 3 \dotfill\answerbox{F}
 
 \item \tf: $1 +^2 =$ \dotfill\answerbox{M}
 
 \item $1 + 2 =$ \answerbox{3}
 
 \item Write a C++ statement to declare an integer variable named
 \verb!x!.
 \begin{answercode}
int x;
 \end{answercode}

 \item Solve $x^2 - 1 = 0$.
 \begin{answerlong}
 Since $x^2 - 1 = (x-1)(x+1)$, $x^2 - 1 = 0$ implies $(x-1)(x+1)=0$.
 Therefore $x - 1 = 0$ or $x = -1$.
 Hence $x = 1$ or $x = -1$.
 \end{answerlong}

 \item
 \begin{mcq}
 {Which is true?}{\answerbox{C}}
 {$1+1=0$}
 {$1+1=1$}
 {$1+1=2$}
 {$1+1=3$}
 {$1+1=4$}
 \end{mcq}


\end{enumerate}

\end{document}
