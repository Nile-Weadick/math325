\input{thispreamble.tex}

\renewcommand\AUTHOR{nweadick1@cougars.ccis.edu} % CHANGE TO YOURS

\begin{document}
\topmattertwo

\textsc{Objectives}
\begin{enumerate}
\item Compute recurrence relation.
\end{enumerate}
\vspace{1cm}



The following are practice problems for self-study:
\begin{enumerate}[nosep]
\li Rosen 8th edition, section 8.1: Odd numbered problems 1-31 except 15, 17,
23, 29.
\end{enumerate}
Some solutions are provided.
For this assignment the problems you need to solve are
\begin{enumerate}[nosep]
\li Rosen 8th edition, section 8.1: \underline{questions 4, 8, 12, 24}.
\end{enumerate}
Explain your work completely using the solutions provided as
guide and examples on how to write math properly.

In \LaTeX\, math notation is enclosed by the \verb!$! symbols.
For instance \verb!$x = a_{1} + b^{2}$! gives you $x = a_{1} + b^{2}$.
For emphasis you can write also write it as
\verb!\[ x = a_{1} + b^{2} \]!
to center your math:
\[
x = a_{1} + b^{2}
\]
For binomial coefficients, \verb!$\binom{5}{2}$! will give you 
$\binom{5}{2}$.
You can also look at the solutions, find something that you can use,
copy-and-paste, and modify.

For more information about \LaTeX\, go to my
website
\href{http://bit.ly/yliow0}{http://bit.ly/yliow0},
click on \verb!Yes!
you are one of my students,
then look for
\href{https://drive.google.com/file/d/0BzjYrK0VFuMWZm5xV0kyR3J2Zm8/view?usp=sharing}{latex.pdf}.)
Even easier: ask questions in CCCS discord.





Draw a box around your final answer if the question required
an explicit answer (i.e., if it's not a proof question).

%------------------------------------------------------------------------------
\newpage
\nextq Rosen 8th edition, section 8.1, question 1.

\SOLUTION

Let $T_n$ be the number of moves to solve the $n$--disks tower of Hanoi
problem.
The recurrence relation is
\[
T_n = 2 T_{n-1} + 1 \tag{*}
\]
with base case of $T_1 = 1$.

Let $P(n)$ be the statement
\[
P(n) = \biggl( T_n = 2^n - 1 \biggr)
\]
for $n \geq 1$.
We will prove that $P(n)$ holds for all $n \geq 1$ using
induction.

\text{Base case.}
For $n = 1$, $T_1 = 1$
and $2^1 - 1 = 2 - 1 = 1$.
Hence $T_1 = 2^1 - 1$. Therefore $P(1)$ holds.

\text{Inductive case.}
Suppose $P(n)$ holds, i.e., 
\[
T^n = 2^n - 1
\]
We want to show that $P(n + 1)$ holds as well, i.e., we want to show
\[
T^{n + 1} = 2^{n + 1} - 1
\]
From the recurrence relation $(*)$,
\[
T_{n + 1} = 2T_n + 1
\]
Therefore
\[
T_{n + 1} = 2(2^n - 1) + 1 = 2^{n + 1} - 2 + 1 = 2^{n+1} - 1
\]
and hence $P(n + 1)$ holds.

Therefore by induction, $P(n)$ holds for all $n \geq 1$, i.e.,
\[
T_n = 2^n - 1
\]
for all $n \geq 1$.
\qed





%------------------------------------------------------------------------------
\newpage
\nextq Rosen 8th edition, section 8.1, question 2.

\SOLUTION

Let $a_n$ be the number of permutations on $n$ distinct symbols.

The task $T$ of forming a permutation is
the same as the task $T_1$ of writing down the first symbol
following by the task $T_2$ of writing down $n - 1$ symbols distinct
from the symbol written down by task $T_1$.
By the multiplication principle,
the number of ways to perform $T$ is the number of ways to perform
$T_1$ multiplied by the number of ways to perform $T_2$, i.e.,
\[
a_n = n \cdot a_{n - 1}
\]

Clearly $a_1 = 1$.
To compute a closed form for $a_n$, note that
\begin{align*}
  a_n
  &= n a_{n - 1} \\
  &= n (n - 1) a_{n - 2} \\
  &= n (n - 2) a_{n - 3} \\
  &= ... \\
  &= n (n - 2)(n - 3)\cdots(n - (k - 1)) a_{n - k} \\
\end{align*}
When $n - k = 1$, we have $k = n - 1$ and 
\begin{align*}
  a_n
  &= n (n - 2)(n - 3)\cdots 2 \cdot a_1 \\
  &= n (n - 2)(n - 3)\cdots 2 \cdot 1 \\
\end{align*}

Hence
\[
a_n = n!
\]
for $n \geq 1$.
\qed


%------------------------------------------------------------------------------
\newpage
\nextq Rosen 8th edition, section 8.1, question 3.

\SOLUTION


(a)
Let the one dollor coin, one dollar bill, five dollar fill be denoted by
$a=1, b=1, c=1$.
A deposit is a sequence of $a,b,c$ values such that the sum is $n$.
For instance the deposit
\[
a b a c
\]
is a deposit for $a + b + a + c = 1 + 1 + 1 + 5 = 8$.

A deposit for $n$ dollar is a sequence of the form
\[
\begin{cases}
a p \\
b p' \\
c p'' \\
\end{cases}
\]
where
$p$ is a pattern for the deposit of $n - 1$ dollars,
$p'$ is a pattern for the deposit of $n - 1$ dollars,
$p''$ is a pattern for the deposit of $n - 5$ dollars,
Hence if $a_n$ denotes the number of ways to deposit $n$ dollars,
we have the recurrence relation
\[
a_n =  a_{n-1} + a_{n - 1} + a_{n - 5} = 2 a_{n - 1} + a_{n - 5}
\]
Clearly
\begin{enumerate}[nosep]
\li $a_1 = 2$: the patterns are $a$ and $b$
\li $a_2 = 4$: the patterns are $aa, ab, ba, bb$
\li $a_3 = 8$: the patterns are $aaa, aab, ..., bbb$, i.e., sequence of length 3 using two symbols
\li $a_4 = 16$: the patterns are $aaaa, aaab, ..., bbbb$ i.e., sequence of length 4 using two symbols
\li $a_5 = 33$: the patterns are $aaaaa, aaaab, ..., bbbbb, c$, i.e., sequence of length 5 using two symbols and $c$
\end{enumerate}

Hence we have
\[
a_n =
\begin{cases}
2 a_{n-1} + a_{n - 5} & \text{ if } n \geq 6 \\
2^n                 & \text{ if } 1 \leq n \geq 4 \\
33                  & \text{ if } n = 5 \\

\end{cases}
\]

(b)
\begin{align*}
a_6 &= 2 a_5 + a_1 = 2 \cdot 33 + 2 = 68 \\
a_7 &= 2 a_6 + a_2 = 2 \cdot 68 + 4 = 72 \\
a_8 &= 2 a_7 + a_3 = 2 \cdot 72 + 8 = 152 \\
a_9 &= 2 a_8 + a_4 = 2 \cdot 152 + 16 = 320 \\
a_{10} &= 2 a_9 + a_5 = 2 \cdot 320 + 33 = 673
\end{align*}
\ANSWER \answerbox{673}

\qed


%------------------------------------------------------------------------------
\newpage
\nextq Rosen 8th edition, section 8.1, question 4.
A country uses as currency coins with values of 1 peso,
2 pesos, 5 pesos, and 10 pesos and bills with values of
5 pesos, 10 pesos, 20 pesos, 50 pesos, and 100 pesos.
Find a recurrence relation for the number of ways to pay
a bill of n pesos if the order in which the coins and bills
are paid matters.

First 4 initial conditions

$a_1 = 1 , a_2 = 2, a_3 = 3, a_4 =5 $

if $a_n$ denotes the number of ways to deposit n pesos, we have the recurrence relation

for $n \geqslant 100$

$\{a_n = a_{n-1} + a_{n-2} + 2a_{n-5} + 2a_{n-10} 
+ a_{n-20} + a_{n-50} + a_{n-100}\}$

\SOLUTION

\input{08-01-04.tex}

%------------------------------------------------------------------------------
\newpage
\nextq Rosen 8th edition, section 8.1, question 8.
a) Find a recurrence relation for the number of bit strings
of length n that contain three consecutive 0s.
b) What are the initial conditions?
c) How many bit strings of length seven contain three
consecutive 0s?

(a) let $a_n$ be the number of bit strings of length n containing three consecutive 0s.

we have the recurrence relation

for $n \geqslant 3$

$\{a_n = a_{n-1} + a_{n-2} + a_{n-3} + 2^{n-3}\}$

\answerbox{$\{a_n = a_{n-1} + a_{n-2} + a_{n-3} + 2^{n-3}\}$}

(b) First 3 initial conditions are , 

\answerbox{$a_1 = 0, a_2 = 0, a_3 = 1$}

(c) 

$a_4 = 1 + 0 + 0 + 2^1 = 3$

$a_5 = 3 + 1 + 0 + 2^2 = 8$

$a_6 = 8 + 3 + 1 + 2^3 = 20$

$\{a_7 = a_{6} + a_{5} + a_{4} + 2^{4}\}$

$\{a_7 = 20 + 8 + 3 + 2^{4}\} = 47$

\answerbox{47}

\SOLUTION

\input{08-01-08.tex}

%------------------------------------------------------------------------------
\newpage
\nextq Rosen 8th edition, section 8.1, question 12.
a) Find a recurrence relation for the number of ways to
climb n stairs if the person climbing the stairs can take
one, two, or three stairs at a time.
b) What are the initial conditions?
c) In how many ways can this person climb a flight of
eight stairs?

(a)let $a_n$ be the number of ways to
climb n stairs if the person climbing the stairs can take
one, two, or three stairs at a time.

we have the recurrence relation

\answerbox{for $n \geqslant 3$ \{$a_n = a_{n-1} + a_{n-2} a_{n-3}$\}}

(b) The initial conditions are 

\answerbox{$a_0 = 1, a_1 = 1 , a_2 = 2, a_3 = 4$ }

(c) 

$a_4 = a_{3} + a_{2} + a_{1} = 4 + 2 + 1 = 7$

$a_5 = a_{4} + a_{3} + a_{2} = 7 + 4 + 2 = 13$

$a_6 = a_{5} + a_{4} + a_{3} = 13 + 7 + 4 = 24$

$a_7 = a_{6} + a_{5} + a_{4} = 24 + 13 + 7 = 44$

Now, $a_8 = a_{7} + a_{6} + a_{5} = 44 + 24 + 13 = 81$

\answerbox{81}


\SOLUTION

\input{08-01-12.tex}

%------------------------------------------------------------------------------
\newpage
\nextq Rosen 8th edition, section 8.1, question 24.
Find a recurrence relation for the number of bit sequences
of length n with an even number of 0s.

let $a_n $ be the number of bit sequences
of length n with an even number of 0s.

we have the recurrence relation

\answerbox{for $n \geqslant 2$ \{$a_n = a_{n-1} + 2^{n-1} - a_{n-1}$\}}

\SOLUTION

\input{08-01-24.tex}

%------------------------------------------------------------------------------
\newpage
\input{instructions.tex}
\end{document}
