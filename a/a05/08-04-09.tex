(a)
We have
\begin{align*}
  (1 + x^5 + x^{10} + x^{15} + \cdots)^3
  &= (1 + (x^5) + (x^5)^2 + (x^5)^{3} + \cdots))^3 \\
  &= \left( \sum_{n = 0} (x^5)^n \right)^3 \\
  &= \left( \frac{1}{1 - x^5} \right)^3 \\
  &= \sum_{n = 0}^\infty \binom{3 + n - 1}{n} (x^5)^n \\
  &= \sum_{n = 0}^\infty \binom{n + 2}{n} x^{5n} \\
  &= \sum_{n = 0}^\infty \binom{n + 2}{(n + 2) - n} x^{5n} \\
  &= \sum_{n = 0}^\infty \binom{n + 2}{2} x^{5n} \\
\end{align*}
In the above power series, the coefficient of $x^{10}$
is the coefficient of $x^{5n}$ when $n = 2$, which is 
$\binom{2 + 2}{2} = \binom{4}{2} = 4 \cdot 3 /2 = 6$.

(\textsc{Note.} Alternatively, this is also the number of solutions to
$a + b + c = 10$ where
$a=0,5,10,15,...$,
$b=0,5,10,15,...$,
$c=0,5,10,15,...$.)

(b)
We have
\begin{align*}
  (x^3 + x^4 + x^5 + x^6 + x^7 + \cdots)^3
  &= (x^3(1 + x + x^2 + x^3 + x^4 + \cdots)^3 \\
  &= \left( x^3 \frac{1}{1 - x} \right)^3 \\
  &= x^9 \left( \frac{1}{1 - x} \right)^3 \\
  &= x^9 \sum_{n=0}^\infty \binom{3 + n - 1}{n} x^n \\
  &= x^9 \sum_{n=0}^\infty \binom{n + 2}{n} x^n \\
  &= x^9 \sum_{n=0}^\infty \binom{n + 2}{(n + 2) - n} x^n \\
  &= x^9 \sum_{n=0}^\infty \binom{n + 2}{2} x^n \\
  &= \sum_{n=0}^\infty \binom{n + 2}{2} x^{n+9} \\
  &= \sum_{k=9}^\infty \binom{k - 7}{2} x^{k} & \text{(let $k = n + 9$)}
\end{align*}
In the above power series, the coefficient of $x^{10}$
is the coefficient of $x^{k}$ when $k = 10$, which is 
$\binom{3}{2} = 3$.


(\textsc{Note.} Alternatively, this is also the number of solutions to
$a + b + c = 10$ where
$a,b,c$ are all $\geq 3$.)

(c)
We have
\begin{align*}
&(x^4 + x^5 + x^6)(x^3 + x^4 + x^5 + x^6 + x^7)(1 + x + x^2 +
  x^3 + x^4 + \cdots) \\
  &= x^4(1 + x + x^2)x^3(1 + x + x^2 + x^3 + x^4)\frac{1}{1 - x} \\
  &= x^7 \frac{1 - x^3}{1 - x} \frac{1 - x^5}{1 - x} \frac{1}{1 - x} \\
  &= x^7 (1 - x^3)(1 - x^5) \left( \frac{1}{1 - x} \right)^3\\
  &= x^7 (1 - x^3 - x^5 + x^8) \sum_{n=0}^\infty \binom{3 + n - 1}{n} x^n\\
  &= (x^7 - x^{10} - x^{12} + x^{16}) \sum_{n=0}^\infty \binom{n + 2}{n} x^n\\
  &= (x^7 - x^{10} - x^{12} + x^{16}) \sum_{n=0}^\infty \binom{n + 2}{2} x^n
\end{align*}
The coefficient of $x^{10}$ is
\[
1 \cdot \binom{3 + 2}{2} - 1 \cdot \binom{0+2}{2}
= \binom{5}{2} - \binom{2}{2} = \frac{5 \cdot 4}{2} - 1 = 9 
\]

(\textsc{Note.} Alternatively, this is also the number of solutions to
$a + b + c = 10$ where
$4 \leq a \leq 6$,
$3 \leq b \leq 7$,
$0 \leq c$.)

(d)
We have
\begin{align*}
&(x^2 + x^4 + x^6 + x^8 + \cdots)(x^3 + x^6 + x^9 + \cdots)(x^4 +
  x^8 + x^{12} + \cdots) \\
  &= x^2(1 + x^2 + x^4 + x^6 + \cdots)x^3(1 + x^3 + x^6 + \cdots)x^4(1 +
  x^4 + x^{8} + \cdots) \\
  &= x^{2+3+4}(1 + x^2 + \cdots)(1 + x^3 + \cdots)(1 + x^4  + \cdots) \\
  &= x^{9}(1 + x^2 + \cdots)(1 + x^3 + \cdots)(1 + x^4  + \cdots)
\end{align*}
The term $a_{10}x^{10}$ in the above is the product $x^9$ and the
term of $x^1$ in
$(1 + x^2 + \cdots)(1 + x^3 + \cdots)(1 + x^4  + \cdots)$.
Clearly the coefficient of
$x^1$ in $(1 + x^2 + \cdots)(1 + x^3 + \cdots)(1 + x^4  + \cdots)$ is $0$.
Hence the coefficient of $x^{10}$ in 
$(x^2 + x^4 + x^6 + x^8 + \cdots)(x^3 + x^6 + x^9 + \cdots)(x^4 +
x^8 + x^{12} + \cdots)$
is $0$.


(\textsc{Note.} Alternatively, this is also the number of solutions to
$a + b + c = 10$ where
$a=2,4,6,...$,
$b=3,6,9,...$,
$c=4,8,12,...$.)

(e)
The term with $x^{10}$ in
\[
(1+x^2+x^4+x^6+x^8+ \cdots)(1+x^4+x^8+x^{12} + \cdots)(1 +
x^6 + x^{12} + x^{18} + \cdots)
\]
is given by
\[
1 \cdot x^4 \cdot x^6 + x^2 \cdot x^8 \cdot x^0 + x^4 \cdot x^0 \cdot x^6
+ x^6 \cdot x^4 \cdot x^0 + x^{10} \cdot x^0 \cdot x^0 = 5x^{10}
\]
Therefore the required coefficient is $5$.
This is also the number of solutions to
\[
a + b + c = 10
\]
where $a=0,2,4,6,...$ and $b=0,4,8,12,...$ and $c = 0,6,12,18,...$.
The following are the solution:
\begin{enumerate}[nosep]
\item $(a,b,c) = (0,4,6)$
\item $(a,b,c) = (2,8,0)$
\item $(a,b,c) = (4,0,6)$
\item $(a,b,c) = (6,4,0)$
\item $(a,b,c) = (10,0,0)$
\end{enumerate}

\textsc{Note.}
Whether you should manipulate the power series to get the
required coefficient or convert to a counting problem to get the
coefficient depends on the scenario.
Sometimes you don't know which is better until you try.
For instance for (e), you can try to manipulate the power series
and see if you can get the answer.
