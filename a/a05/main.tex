\input{thispreamble.tex}

\renewcommand\AUTHOR{nweadick1@cougars.ccis.edu} % CHANGE TO YOURS

\begin{document}
\topmattertwo

\textsc{Objectives}
\begin{enumerate}[nosep]
\item Use generating functions to compute closed forms for linear recurrences.
\item Use generating functions to solve counting problems.
\item Use generating functions to compute/prove identities.
\end{enumerate}
\vspace{1cm}



The following are practice problems for self-study:
\begin{enumerate}[nosep]
\li Rosen 8th edition, section 8.4: Odd numbered problems 1-57 except 43,
47, 49, 51.
41 was mentioned in my notes (and in class).
46 is in my notes.
53-57 are in my notes.
\end{enumerate}
Some solutions are provided.
For this assignment the problems you need to solve are
\begin{enumerate}[nosep]
\li Rosen 8th edition, section 8.4:
\underline{questions 8, 10, 14, 18, 24, 34, 38}.
\end{enumerate}
Explain your work completely using the solutions provided as
guide and examples on how to write math properly.

In \LaTeX\, math notation is enclosed by the \verb!$! symbols.
For instance \verb!$x = a_{1} + b^{2}$! gives you $x = a_{1} + b^{2}$.
For emphasis you can write also write it as
\verb!\[ x = a_{1} + b^{2} \]!
to center your math:
\[
x = a_{1} + b^{2}
\]
For binomial coefficients, \verb!$\binom{5}{2}$! will give you 
$\binom{5}{2}$.
You can also look at the solutions, find something that you can use,
copy-and-paste, and modify.

For more information about \LaTeX\, go to my
website
\href{http://bit.ly/yliow0}{http://bit.ly/yliow0},
click on \verb!Yes!
you are one of my students,
then look for
\href{https://drive.google.com/file/d/0BzjYrK0VFuMWZm5xV0kyR3J2Zm8/view?usp=sharing}{latex.pdf}.)
Even easier: ask questions in CCCS discord.





Draw a box around your final answer if the question required
an explicit answer (i.e., if it's not a proof question).

%------------------------------------------------------------------------------
\newpage
\nextq Rosen 8th edition, section 8.4, question 1.

\SOLUTION

The generating function for 2,2,2,2,2,2 is
\begin{align*}
f(x) &= 2 + 2x + 2x^2 + 2x^3 + 2x^4 + 2x^5 \\
     &= 2(1 + x + x^2 + x^3 + x^4 + x^5) \\
     &= 2 \frac{1 - x^6}{1 - x}
\end{align*}
\qed





%------------------------------------------------------------------------------
\newpage
\nextq Rosen 8th edition, section 8.4, question 2.

\SOLUTION

The generating function for 1, 4, 16, 64, 256 is
\begin{align*}
f(x) &= 1 + 4x + 16x^2 + 64 x^3 + 256 x^4 \\
     &= 1 + 4x + 4^2x^2 + 4^3 x^3 + 4^4 x^4 \\
     &= 1 + 4x + (4x)^2 + (4x)^3 + (4x)^4 \\
     &= \frac{1 - (4x)^5}{1 - x} \\
     &= \frac{1 - 1024x^5}{1 - x}
\end{align*}
\qed


%------------------------------------------------------------------------------
\newpage
\nextq Rosen 8th edition, section 8.4, question 3.

\SOLUTION


(a)
The generating function for
$0, 2, 2, 2, 2, 2, 2, 0, 0, 0, 0, 0$,... is
\begin{align*}
f(x) &= 2x + 2x^2 + \cdots + 2x^6 \\
     &= 2x(1 + x + \cdots + x^5) \\
     &= 2x\frac{1 - x^6}{1 - x}
\end{align*}


(b)
The generating function for
$0, 0, 0, 1, 1, 1, 1, 1, 1,...$ is
\begin{align*}
f(x) &= x^3 + x^4 + x^5 + \cdots \\
     &= x^3(1 + x + x^2 + \cdots) \\
     &= x^3\frac{1}{1 - x} \\
     &= \frac{x^3}{1 - x}
\end{align*}


(c)
The generating function for
$0, 1, 0, 0, 1, 0, 0, 1, 0, 0, 1, ...$ is
\begin{align*}
f(x) &= x + x^4 + x^7 + x^{10} + \cdots \\
     &= x(1 + x^3 + x^6 + x^{9} + \cdots) \\
     &= x(1 + (x^3) + (x^3)^2 + (x^{3})^3 + \cdots) \\
     &= x\frac{1}{1 - (x^3)} \\
     &= \frac{x}{1 - x^3}
\end{align*}


(d)
The generating function for
$2, 4, 8, 16, 32, 64, 128, 256,...$ is
\begin{align*}
f(x) &= 2 + 4x + 8x^2 + 16x^3 + 32x^{4} + \cdots \\
     &= 2(1 + 2x + 4x^2 + 8x^{3} + 16x^4 + \cdots) \\
     &= 2(1 + (2x) + (2x)^2 + (2x)^3 + \cdots) \\
     &= 2\frac{1}{1 - (2x)} \\
     &= \frac{2}{1 - 2x}
\end{align*}


(e)
The generating function for
$\binom{7}{0}, \binom{7}{1}, \binom{7}{2}, ..., \binom{7}{7}, 0, 0, 0, ...$ is
\begin{align*}
f(x) &= \binom{7}{0} + \binom{7}{1}x + \binom{7}{2}x^2 + \cdots
        + \binom{7}{7}x^7 \\
     &= \sum_{i=0}^7 \binom{7}{i} x^i \\
     &= (1 + x)^7
\end{align*}


(f)
The generating function for
$2, -2, 2, -2, 2, -2, 2, -2, ...$ is
\begin{align*}
f(x) &= 2 + (-2)x + 2x^2 + (-2)x^3 + \cdots \\
     &= 2 ( 1 + (-1)x + 1x^2 + (-1)x^3 + \cdots ) \\
     &= 2 ( 1 + (-x) + (-x)^2 + (-x)^3 + \cdots ) \\
     &= 2 \frac{1}{1 - (-x)} \\
     &= \frac{2}{1 + x} 
\end{align*}


(g)
The generating function for
$1, 1, 0, 1, 1, 1, 1, 1, 1, 1, ...$is
\begin{align*}
f(x) &= 1 + 1x + 0x^2 + 1x^3 + 1x^4 + 1x^5 + \cdots \\
     &= 1 + x + (x^3 + x^4 + x^5 + \cdots ) \\
     &= 1 + x + x^3(1 + x + x^2 + \cdots ) \\
     &= 1 + x + x^3\frac{1}{1 - x} \\
     &= 1 + x + \frac{x^3}{1 - x} \\
     &= \frac{(1 + x)(1 - x) + x^3}{1 - x} \\
     &= \frac{1 - x^2 + x^3}{1 - x} \\
\end{align*}


%[1∕(1−x)]−x^2 = (1 - x^2(1 - x))/(1 - x) = (1 - x^2 + x^3)/(1 - 2)


(g)
The generating function for
$0, 0, 0, 1, 2, 3, 4, ...$ is
\begin{align*}
f(x) &= 1x^3 + 2x^4 + 3x^5 + 4x^6 + \cdots \\
     &= x^3(1 + 2x + 3x^2 + 4x^3 + \cdots) \\
     &= x^3\frac{d}{dx}(x + x^2 + x^3 + x^4 + \cdots) \\
     &= x^3\frac{d}{dx}(1 + x + x^2 + x^3 + x^4 + \cdots) \\
     &= x^3\frac{d}{dx}\frac{1}{1 - x} \\
     &= x^3\frac{-1}{(1 - x)^2} \\
     &= -\frac{x^3}{(1 - x)^2} \\
\end{align*}


%------------------------------------------------------------------------------
\newpage
\nextq Rosen 8th edition, section 8.4, question 4.

\SOLUTION


(a) The generating function is
\[
f(x)
= \sum_{i=0}^6 (-1)x^i
= -\sum_{i=0}^6 x^i
= -\frac{1 - x^7}{1 - x} 
= \frac{x^7 - 1}{1 - x} 
\]

(b) The generating function is
\begin{align*}
  f(x) &= 1 + 3x + 9x^2 + 27x^3 + 81x^4 + 243x^5 + 729x^6 + \cdots \\
  &= 1 + (3x) + (3x)^2 + (3x)^3 + (3x)^4 + (3x)^5 + (3x)^6 + \cdots \\
  &= \frac{1}{1 - 3x}
\end{align*}

(c) The generating function is
\begin{align*}
  f(x)
  &= 0 + 0x + 3x^2 + (-3)x^3 + 3x^4 + (-3)x^5 + 3x^6 + (-3)x^7 + \cdots\\
  &= 3x^2 (1 - x + x^2 - x^3 + x^4 - x^5 \cdots) \\
  &= 3x^2 (1 + (-x) + (-x)^2 + (-x)^3 + (-x)^4 + (-x)^5 \cdots) \\
  &= 3x^2 \frac{1}{1 - (-x)} \\
  &= \frac{3x^2}{1 + x} 
\end{align*}


(d) The generating function is
\begin{align*}
  f(x)
  &= x + \sum_{i=0}^\infty x^i \\
  &= x + \frac{1}{1 - x} \\
  &= \frac{1 - x + 1}{1 - x} \\
  &= \frac{2 - x}{1 - x}
\end{align*}

(e)
The generating function is
\begin{align*}
  f(x)
  &= \sum_{i=0}^7 2^i\binom{7}{i} x^i \\
  &= \sum_{i=0}^7 \binom{7}{i} (2x)^i \\
  &= (1 + 2x)^7
\end{align*}


(f)
The generating function is
\begin{align*}
  f(x)
  &= (-3) + 3x + (-3)x^2 + 3x^3 + (-3)x^4 + 3x^5 + \cdots \\
  &= (-3) \sum_{i=0}^\infty (-x)^i \\
  &= (-3) \frac{1}{1 - (-x)} \\
  &= - \frac{3}{1 + x} 
\end{align*}

(g)
The generating function is
\begin{align*}
  f(x)
  &= 0 + 1x + (-2)x^2 + 4x^3 + (-8)x^4 + 16x^5 + (-32)x^6 + 64x^7 + \cdots \\
  &= \sum_{i=0}^\infty (-2)^{i-1}x^i \\
  &= \sum_{i=0}^\infty \frac{1}{-2}(-2)^{i}x^i \\
  &= -\frac{1}{2} \sum_{i=0}^\infty (-2x)^i \\
  &= -\frac{1}{2} \frac{1}{1 - (-2x)} \\
  &= -\frac{1}{2(1 + 2x)} 
\end{align*}

(h)
The generating function is
\begin{align*}
  f(x)
  &= \sum_{i=0}^\infty x^{2i} \\
  &= \sum_{i=0}^\infty (x^2)^i \\
  &= \frac{1}{1 - x^2} 
\end{align*}


%------------------------------------------------------------------------------
\newpage
\nextq Rosen 8th edition, section 8.4, question 5.

\SOLUTION


(a)
The generating function is
\begin{align*}
  f(x) &= \sum_{i=0}^\infty 5x^i \\
  &= 5 \sum_{i=0}^\infty x^i \\
  &= 5 \frac{1}{1 - x} \\
  &= \frac{5}{1 - x} \\
\end{align*}

(b)
The generating function is
\begin{align*}
  f(x) &= \sum_{i=0}^\infty 3^nx^i \\
  &= 5 \sum_{i=0}^\infty (3x)^i \\
  &= \frac{1}{1 - 3x} 
\end{align*}

(c)
The generating function is
\begin{align*}
  f(x) &= \sum_{i=3}^\infty 2x^i \\
  &= 2 \sum_{i=3}^\infty x^i \\
  &= 2 \sum_{k=0}^\infty x^{k+3} & & \text{(let $k = i - 3$)} \\
  &= 2 \sum_{k=0}^\infty x^3x^{k} \\
  &= 2x^3 \sum_{k=0}^\infty x^{k} \\
  &= 2x^3 \frac{1}{1 - x} \\
  &= \frac{2x^3}{1 - x} 
\end{align*}


(d)
The generating function is
\begin{align*}
  f(x) &= \sum_{n=0}^\infty (2n + 3) x^n \\
  &= \sum_{n=0}^\infty 2n x^n + \sum_{n=0}^\infty 3 x^n \\
  &= 2 \sum_{n=1}^\infty n x^n + 3 \sum_{n=0}^\infty x^n \\
  &= 2x \sum_{n=1}^\infty n x^{n-1} + 3 \frac{1}{1 - x} \\
  &= 2x \sum_{n=1}^\infty \frac{d}{dx} x^{n} + 3 \frac{1}{1 - x} \\
  &= 2x \frac{d}{dx} \sum_{n=0}^\infty x^{n} + \frac{3}{1 - x} \\
  &= 2x \frac{d}{dx} \frac{1}{1 - x} + \frac{3}{1 - x} \\
  &= 2x \frac{-1}{(1 - x)^2} + \frac{3}{1 - x} \\
  &= \frac{-2x}{(1 - x)^2} + \frac{3}{1 - x} \\
  &= \frac{-2x}{(1 - x)^2} + \frac{3(1 + x)}{(1 - x)(1 + x)} \\
  &= \frac{-2x + 3(1 + x)}{(1 - x)^2} \\
  &= \frac{3 - x}{(1 - x)^2} 
\end{align*}


(e)
The generating function is
\begin{align*}
  f(x) &= \sum_{n=0}^\infty \binom{8}{n} x^n \\
\end{align*}


%------------------------------------------------------------------------------
\newpage
\nextq Rosen 8th edition, section 8.4, question 6.

\SOLUTION


(a)
The generating function is
\begin{align*}
  f(x) &= \sum_{i=0}^\infty (-1)x^i \\
  &= - \sum_{i=0}^\infty x^i \\
  &= - \frac{1}{1 - x} \\
  &= \frac{1}{x - 1} 
\end{align*}

(b)
The generating function is
\begin{align*}
  f(x) &= \sum_{n=1}^\infty 2^nx^n \\
  &= \sum_{n=0}^\infty (2x)^n \\
  &= \frac{1}{1 - 2x} 
\end{align*}

(c)
The generating function is
\begin{align*}
  f(x) &= \sum_{n=0}^\infty (n-1)x^n \\
  &= \sum_{n=0}^\infty nx^n - \sum_{n=0}^\infty x^n \\
  &= \sum_{n=1}^\infty nx^{n} - \frac{1}{1 - x} \\
  &= x\sum_{n=1}^\infty nx^{n-1} - \frac{1}{1 - x} \\
  &= x \sum_{n=1}^\infty \frac{d}{dx} x^{n} - \frac{1}{1 - x} \\
  &= x \frac{d}{dx} \sum_{n=1}^\infty x^{n} - \frac{1}{1 - x} \\
  &= x \frac{d}{dx} \frac{1}{1 - x} - \frac{1}{1 - x} \\
  &= x \frac{-1}{(1 - x)^2} - \frac{1}{1 - x} \\
  &= \frac{-x - (1 - x)}{(1 - x)^2}  \\
  &= -\frac{1}{(1 - x)^2}  
\end{align*}


(d)
The generating function is
\begin{align*}
  f(x) &= \sum_{n=0}^\infty \frac{1}{(n+1)!} x^n \\
  &= \frac{1}{x}\sum_{n=0}^\infty \frac{1}{(n+1)!} x^{n+1} \\
  &= \frac{1}{x}\sum_{k=1}^\infty \frac{1}{k!} x^{k} & & \text{(let $k = n + 1$)}\\
  &= \frac{1}{x}
  \left(
  \sum_{k=0}^\infty \frac{1}{k!} x^{k} - 1
  \right)\\
  &= \frac{1}{x}
  \left(
  e^x - 1
  \right)\\
  &= \frac{e^x - 1}{x}
\end{align*}


%------------------------------------------------------------------------------
\newpage
\nextq Rosen 8th edition, section 8.4, question 7.

\SOLUTION

(a)
Using binomial theorem, we have
\begin{align*}
  (3x - 4)^3 &= \sum_{n=0}^3 \binom{3}{n} (3x)^n(-4)^{3 - n} \\
             &= \sum_{n=0}^3 \binom{3}{n} 3^n(-1)^{3 - n}4^{3 - n}x^n \\
             &= \sum_{n=0}^3 (-1)^{n + 1}\binom{3}{n} 4^3 \left(\frac{3}{4}\right)^n x^n 
\end{align*}
Therefore
\[
a_n =
\begin{cases}
  \binom{3}{n} (-1)^{n - 1}4^3 \left(\frac{3}{4}\right)^n & \text{if $n = 0,1,2,3$}\\
  0 & \text{otherwise}
\end{cases}
\]


(b)
Using binomial theorem, we have
\begin{align*}
(x^3 + 1)^3 &= \sum_{n=0}^3 \binom{3}{n} (x^3)^n \\
            &= \sum_{n=0}^3 \binom{3}{n} x^{3n} \\
\end{align*}
Therefore
\[
a_n =
\begin{cases}
  \binom{3}{n/3} & \text{if $n = 0, 3, 6, 9$}\\ 
  0              & \text{otherwise}
\end{cases}
\]


(c)
From
\[
  \frac{1}{1 - 5x} = \sum_{n=0}^\infty (5x)^n = \sum_{n=0}^\infty 5^n x^n
\]
we get
\[
a_n = 5^n
\]
for $n \geq 0$.


(d)
From
\begin{align*}
  \frac{x^3}{1 + 3x} &= x^3 \frac{1}{1 - (-3x)} \\
                     &= x^3 \sum_{n=0}^\infty (-3x)^n \\
                     &= x^3 \sum_{n=0}^\infty (-3)^n x^n \\
                     &= \sum_{n=0}^\infty (-3)^n x^{n+3} \\
                     &= \sum_{k=3}^\infty (-3)^{k-3} x^{k} & \text{(let $k = n + 3$)}\\
\end{align*}
we get
\[
a_n =
\begin{cases}
  0         & \text{if $n = 0, 1, 2$} \\
  (-3)^{n-3} & \text{if $n \geq 3$}
\end{cases}
\]


(e)
From
\begin{align*}
  x^2 + 3x + 7 + \frac{1}{1 - x^2}
  &= x^2 + 3x + 7 + \sum_{n=0}^\infty (x^2)^n \\
  &= x^2 + 3x + 7 + \sum_{n=0}^\infty x^{2n} \\
  &= x^2 + 3x + 7 + \left(1 + x^2 + \sum_{n=2}^\infty x^{2n} \right) \\
  &= 2x^2 + 3x + 8 + \sum_{n=2}^\infty x^{2n} 
\end{align*}
we get
\[
a_n =
\begin{cases}
  8 & \text{if $n = 0$} \\
  3 & \text{if $n = 1$} \\
  2 & \text{if $n = 2$} \\
  1 & \text{if $n \geq 4$ is even} \\
  0 & \text{if $n \geq 3$ is odd} \\
\end{cases}
\]


(f)
From
\begin{align*}
  \frac{x^4}{1 - x^4} - x^3 - x^2 - x - 1
  &= x^4 \frac{1}{1 - x^4} - x^3 - x^2 - x - 1 \\
  &= x^4 \sum_{n=0}^\infty (x^4)^n - x^3 - x^2 - x - 1 \\
  &= x^4 \sum_{n=0}^\infty x^{4n} - x^3 - x^2 - x - 1 \\
  &= \sum_{n=0}^\infty x^{4(n + 1)} - x^3 - x^2 - x - 1 \\
\end{align*}
we get
\[
a_n =
\begin{cases}
  -1 & \text{if $n = 0,1,2,3$} \\
  1 & \text{if $n = 4k$ for $k\geq 1$} \\
  0 & \text{otherwise}
\end{cases}
\]

(g)
From
\begin{align*}
  \frac{x^2}{(1 - x)^2} &= x^2 \frac{1}{(1 - x)^2} \\
  &= x^2 \sum_{n=0}^\infty \binom{2 + n - 1}{n} x^n \\
  &= x^2 \sum_{n=0}^\infty \binom{n + 1}{n} x^n \\
  &= x^2 \sum_{n=0}^\infty \binom{n + 1}{(n + 1) - n} x^n \\
  &= x^2 \sum_{n=0}^\infty \binom{n + 1}{1} x^n \\
  &= x^2 \sum_{n=0}^\infty (n + 1) x^n \\
  &= \sum_{n=0}^\infty (n + 1) x^{n+2} \\
  &= \sum_{k=2}^\infty (k - 1) x^{k} & \text{(let $k = n + 2$)}
\end{align*}
we get
\[
a_n
\begin{cases}
  0 & \text{if } n = 0,1 \\
  n-1 & \text{if } n \geq 2 
\end{cases}
\]


(g)
From
\begin{align*}
  2e^{2x} &= 2 \sum_{n=0}^\infty \frac{1}{n!} x^n \\
         &= \sum_{n=0}^\infty \frac{2}{n!} x^n 
\end{align*}
we get
\[
a_n = \frac{2}{n!}
\]
for $n \geq 0$.


%------------------------------------------------------------------------------
\newpage
\nextq Rosen 8th edition, section 8.4, question 8.

\SOLUTION

For each of these generating functions, provide a closed
formula for the sequence it determines.

(a) = $(x^2+1)^3$ = by multiplying out, we get $x^6 + 3x^4 + 3x2 + 1$
\answerbox{$a_0 = 1, a_2 = 3, a_4=3, a_6 = 1$}

(b) = $(3x-1)^3$ = by multiplying out, we get $27x^3-27x^2+9x-1$
\answerbox{$a_0 =-1,a_1=9,a_2=-27,a_3=27$}

(c) = $1/(1-2x^2)$ = \answerbox{$x^{2n}$, ODD coef's are 0.}

(d) = $x^2/(1-x)^3$ = $x^2 \cdot \binom{n+2}{2}x^n$ = $\binom{n+2}{2}x^{n+2}$ = $\binom{n}{2}x^n$
\answerbox{$\binom{n}{2}$ if $n \geqslant 2$, $0$ if $n=1,2$}

(e) = $x-1+(1/(1-3x))$ = $a_n = $
\answerbox{$3^n$,$a_0=-1=3^0=0, a_1= 1+3^1= 4$}

(f) = 
\answerbox{$a_n$ = $(-1)^n3n$ for $n \geqslant 3$ 
$a_n$ = $(-1)^n\binom{n+2}{2}$ for $n < 3$}

(g) = $x/(1+x+x^2)$ = \answerbox{$a_n = 0$ for n is a multiple of 3, 
$a_n =1$, for n = 1 > multiple of 3, 
$a_n =-1$, for n = 2 > multiple of 3}

(h) = $e^{3x^2} - 1$ = \answerbox{$a_0 = 0$, $a_n = 0$ for n is ODD, $a_{2n} = 3^n /m!$ for n is EVEN}
\input{08-04-08.tex}

%------------------------------------------------------------------------------
\newpage
\nextq Rosen 8th edition, section 8.4, question 9.

\SOLUTION

(a)
We have
\begin{align*}
  (1 + x^5 + x^{10} + x^{15} + \cdots)^3
  &= (1 + (x^5) + (x^5)^2 + (x^5)^{3} + \cdots))^3 \\
  &= \left( \sum_{n = 0} (x^5)^n \right)^3 \\
  &= \left( \frac{1}{1 - x^5} \right)^3 \\
  &= \sum_{n = 0}^\infty \binom{3 + n - 1}{n} (x^5)^n \\
  &= \sum_{n = 0}^\infty \binom{n + 2}{n} x^{5n} \\
  &= \sum_{n = 0}^\infty \binom{n + 2}{(n + 2) - n} x^{5n} \\
  &= \sum_{n = 0}^\infty \binom{n + 2}{2} x^{5n} \\
\end{align*}
In the above power series, the coefficient of $x^{10}$
is the coefficient of $x^{5n}$ when $n = 2$, which is 
$\binom{2 + 2}{2} = \binom{4}{2} = 4 \cdot 3 /2 = 6$.

(\textsc{Note.} Alternatively, this is also the number of solutions to
$a + b + c = 10$ where
$a=0,5,10,15,...$,
$b=0,5,10,15,...$,
$c=0,5,10,15,...$.)

(b)
We have
\begin{align*}
  (x^3 + x^4 + x^5 + x^6 + x^7 + \cdots)^3
  &= (x^3(1 + x + x^2 + x^3 + x^4 + \cdots)^3 \\
  &= \left( x^3 \frac{1}{1 - x} \right)^3 \\
  &= x^9 \left( \frac{1}{1 - x} \right)^3 \\
  &= x^9 \sum_{n=0}^\infty \binom{3 + n - 1}{n} x^n \\
  &= x^9 \sum_{n=0}^\infty \binom{n + 2}{n} x^n \\
  &= x^9 \sum_{n=0}^\infty \binom{n + 2}{(n + 2) - n} x^n \\
  &= x^9 \sum_{n=0}^\infty \binom{n + 2}{2} x^n \\
  &= \sum_{n=0}^\infty \binom{n + 2}{2} x^{n+9} \\
  &= \sum_{k=9}^\infty \binom{k - 7}{2} x^{k} & \text{(let $k = n + 9$)}
\end{align*}
In the above power series, the coefficient of $x^{10}$
is the coefficient of $x^{k}$ when $k = 10$, which is 
$\binom{3}{2} = 3$.


(\textsc{Note.} Alternatively, this is also the number of solutions to
$a + b + c = 10$ where
$a,b,c$ are all $\geq 3$.)

(c)
We have
\begin{align*}
&(x^4 + x^5 + x^6)(x^3 + x^4 + x^5 + x^6 + x^7)(1 + x + x^2 +
  x^3 + x^4 + \cdots) \\
  &= x^4(1 + x + x^2)x^3(1 + x + x^2 + x^3 + x^4)\frac{1}{1 - x} \\
  &= x^7 \frac{1 - x^3}{1 - x} \frac{1 - x^5}{1 - x} \frac{1}{1 - x} \\
  &= x^7 (1 - x^3)(1 - x^5) \left( \frac{1}{1 - x} \right)^3\\
  &= x^7 (1 - x^3 - x^5 + x^8) \sum_{n=0}^\infty \binom{3 + n - 1}{n} x^n\\
  &= (x^7 - x^{10} - x^{12} + x^{16}) \sum_{n=0}^\infty \binom{n + 2}{n} x^n\\
  &= (x^7 - x^{10} - x^{12} + x^{16}) \sum_{n=0}^\infty \binom{n + 2}{2} x^n
\end{align*}
The coefficient of $x^{10}$ is
\[
1 \cdot \binom{3 + 2}{2} - 1 \cdot \binom{0+2}{2}
= \binom{5}{2} - \binom{2}{2} = \frac{5 \cdot 4}{2} - 1 = 9 
\]

(\textsc{Note.} Alternatively, this is also the number of solutions to
$a + b + c = 10$ where
$4 \leq a \leq 6$,
$3 \leq b \leq 7$,
$0 \leq c$.)

(d)
We have
\begin{align*}
&(x^2 + x^4 + x^6 + x^8 + \cdots)(x^3 + x^6 + x^9 + \cdots)(x^4 +
  x^8 + x^{12} + \cdots) \\
  &= x^2(1 + x^2 + x^4 + x^6 + \cdots)x^3(1 + x^3 + x^6 + \cdots)x^4(1 +
  x^4 + x^{8} + \cdots) \\
  &= x^{2+3+4}(1 + x^2 + \cdots)(1 + x^3 + \cdots)(1 + x^4  + \cdots) \\
  &= x^{9}(1 + x^2 + \cdots)(1 + x^3 + \cdots)(1 + x^4  + \cdots)
\end{align*}
The term $a_{10}x^{10}$ in the above is the product $x^9$ and the
term of $x^1$ in
$(1 + x^2 + \cdots)(1 + x^3 + \cdots)(1 + x^4  + \cdots)$.
Clearly the coefficient of
$x^1$ in $(1 + x^2 + \cdots)(1 + x^3 + \cdots)(1 + x^4  + \cdots)$ is $0$.
Hence the coefficient of $x^{10}$ in 
$(x^2 + x^4 + x^6 + x^8 + \cdots)(x^3 + x^6 + x^9 + \cdots)(x^4 +
x^8 + x^{12} + \cdots)$
is $0$.


(\textsc{Note.} Alternatively, this is also the number of solutions to
$a + b + c = 10$ where
$a=2,4,6,...$,
$b=3,6,9,...$,
$c=4,8,12,...$.)

(e)
The term with $x^{10}$ in
\[
(1+x^2+x^4+x^6+x^8+ \cdots)(1+x^4+x^8+x^{12} + \cdots)(1 +
x^6 + x^{12} + x^{18} + \cdots)
\]
is given by
\[
1 \cdot x^4 \cdot x^6 + x^2 \cdot x^8 \cdot x^0 + x^4 \cdot x^0 \cdot x^6
+ x^6 \cdot x^4 \cdot x^0 + x^{10} \cdot x^0 \cdot x^0 = 5x^{10}
\]
Therefore the required coefficient is $5$.
This is also the number of solutions to
\[
a + b + c = 10
\]
where $a=0,2,4,6,...$ and $b=0,4,8,12,...$ and $c = 0,6,12,18,...$.
The following are the solution:
\begin{enumerate}[nosep]
\item $(a,b,c) = (0,4,6)$
\item $(a,b,c) = (2,8,0)$
\item $(a,b,c) = (4,0,6)$
\item $(a,b,c) = (6,4,0)$
\item $(a,b,c) = (10,0,0)$
\end{enumerate}

\textsc{Note.}
Whether you should manipulate the power series to get the
required coefficient or convert to a counting problem to get the
coefficient depends on the scenario.
Sometimes you don't know which is better until you try.
For instance for (e), you can try to manipulate the power series
and see if you can get the answer.


%------------------------------------------------------------------------------
\newpage
\nextq Rosen 8th edition, section 8.4, question 10.

\SOLUTION

(a) = $1/(1-x^3)$ = $\binom{n+2}{2}x^n$
\answerbox{$\binom{5}{2} = 10$}

(b) = $\binom{n+2}{2}x^n$
\answerbox{$\binom{5}{2} = 10$}

(c) = 

(d) = \answerbox{2}

(e) = \answerbox{0}

\input{08-04-10.tex}

%------------------------------------------------------------------------------
\newpage
\nextq Rosen 8th edition, section 8.4, question 14.

\SOLUTION

$x^n$ for $1/(1-x)^5$ = $ \binom{n+4}{4}$

$\binom{16}{4} - 5 \cdot \binom{12}{4} + 10 \cdot \binom{8}{4} - 10 \cdot \binom{4}{4}$

$1820-2475+700-10$

\answerbox{35}

\input{08-04-14.tex}
%------------------------------------------------------------------------------
\newpage
\nextq Rosen 8th edition, section 8.4, question 18.

\SOLUTION

$x^n$ for $1/(1-x)^3$ = $\binom{n+2}{2}$

$\binom{13}{2} - \binom{5}{2}$

\answerbox{68}
\input{08-04-18.tex}

%------------------------------------------------------------------------------
\newpage
\nextq Rosen 8th edition, section 8.4, question 24.

\SOLUTION

$ x^n $ for $1/(1-x)^2$ = $n+1$

$1 \cdot 3 + 2 \cdot 2 + 3 \cdot 1$

\answerbox{10}

\input{08-04-24.tex}
%------------------------------------------------------------------------------
\newpage
\nextq Rosen 8th edition, section 8.4, question 34.

\SOLUTION

$G(x)$ = $ 1/(1-4x)$

\answerbox{$a_k$ = $4^k$}

\input{08-04-34.tex}


%------------------------------------------------------------------------------
\newpage
\nextq Rosen 8th edition, section 8.4, question 38.

\SOLUTION

\input{08-04-38.tex}

%------------------------------------------------------------------------------
\newpage
\nextq Find a closed form for $0^3 + 1^3 + 2^3 + \cdots + n^3$.

\SOLUTION

\input{sum-of-cubes.tex}

%------------------------------------------------------------------------------
\newpage
\input{instructions.tex}
\end{document}
